\documentclass{emulateapj} 
 
\usepackage[dvipdf]{epsfig} 
\usepackage[dvips]{rotating}
\usepackage{subfigure}

\newcommand{\IRAS}{{\it IRAS}}
\newcommand{\HERSCHEL}{{\it Herschel}}
\newcommand{\SPITZER}{{\it Spitzer}}
\newcommand{\PLANCK}{{\it Planck}}
\newcommand{\AKARI}{{\it Akari}}
\newcommand{\COBE}{{\it COBE}}

\bibpunct{(}{)}{;}{a}{}{,} 
 
\shorttitle{\PLANCK~Dust Model}
 
\shortauthors{Meisner \& Finkbeiner} 

\begin{document}

\title{Two-component thermal dust emission model: application to the {\it PLANCK} HFI maps}
\author{Aaron M. Meisner\altaffilmark{1,2}}
\author{Douglas P. Finkbeiner\altaffilmark{1,2}}
\altaffiltext{1}{Department of Physics, Harvard University, 17 Oxford Street, 
Cambridge, MA 02138, USA; ameisner@fas.harvard.edu}
\altaffiltext{2}{Harvard-Smithsonian Center for Astrophysics, 60 Garden St, 
Cambridge, MA 02138, USA; dfinkbeiner@cfa.harvard.edu}

\begin{abstract}
We present full-sky maps of dust optical depth and temperature derived by 
fitting the \cite{FDS99} two-component dust emission model to
\PLANCK~217, 353, 545, 857 GHz and \cite{SFD} \IRAS~100$\mu$m data. Both the 
optical depth and temperature maps have angular resolution of $6.1'$. Such maps
 serve as an important alternative to and comparison for other dust 
emission models, especially the single modified blackbody (MBB) model in which 
$\beta$ and $T$ both vary spatially. We also expect the derived optical 
depth to provide valuable cross-checks for other dust-column related data 
products, specifically the \cite{SFD} map, which assumed 
$\beta = 2$ single-component dust emission and included a temperature map of 
inferior $\sim$$1.3^{\circ}$ resolution. We briefly compare our 
best-fit two-component spectral energy distributions to those of the 
\cite{planckdust} single-component parametrization.
%mention the enhanced temperature resolution relative to SFD
%Our optical depth map will also provide a useful reference for dust maps 
%based on stellar colors rather than infrared emission.
\end{abstract}

\section{Introduction}
%why dust important in astronomy/astrophysics
The presence of Galactic interstellar dust affects
astronomical observations over a wide range of wavelengths. In the mid-infrared
and far-infrared, Galactic dust emission is a major contributor
to the observed sky intensity. At optical and ultraviolet (UV) wavelengths, 
extinction by foreground dust grains attenuates the signal from extragalactic 
sources over the entire sky. Understanding the properties of dust in the 
interstellar medium (ISM) is valuable in its own right, as the ISM is the site 
of star formation, thereby playing a crucial role in galaxy evolution. Equally,
 or perhaps even more important to the practice of astronomy, however, is 
accurately accounting for dust as a foreground which reddens optical/UV 
observations and superimposes Galactic emission on low-frequency observations 
of the cosmic microwave background (CMB). 
%specify ``low frequency more precisely eventually
% somewhere mention re-radiation

%basic background about interstellar dust emission
Over the past decades, satellite observations have dramatically enhanced our
knowledge about infrared emission from the ISM. The \textit{Infrared Astronomy 
Satellite} (\IRAS), with its $\sim$4$'$ resolution, revolutionized the study of
 Galactic dust emission, first revealing the high-latitude ``infrared cirrus'' 
using 60$\mu$m and 100$\mu$m observations \citep{low84, wheelock94}. Later, the
Diffuse Infrared Background Experiment (DIRBE) aboard the \COBE~satellite 
provided complementary full-sky measurements at ten infrared wavelengths from 
1.25$\mu$m to 240$\mu$m with inferior 0.7$^{\circ}$ angular resolution, but 
with a reliable zero point \citep{boggess92}. \COBE/FIRAS \citep{firas} also 
provided full-sky infrared dust spectra at $7^{\circ}$ resolution in 213 narrow
frequency bins between 200 GHz and 2100 GHz.

\cite{FDS99} (hereafter FDS99) used FIRAS data to derive a globally best-fit 
model of dust emission applicable over a very broad range of frequencies. 
FDS99 showed that a model consisting of a single MBB with power law index 
$\beta$ could not accurately match the \IRAS/FIRAS/DIRBE spectrum at both the 
Wien and Rayleigh-Jeans extremes. To fit both ends of the spectrum, FDS99 
proposed a dust emission model consisting of two MBBs, each with a different
temperature and power law index. Physically, these two components
represent distinct dust grain species within the ISM, specifically 
amorphous silicates and carbonaceous grains.

re
The \PLANCK~2013 data release \citep{planck2013} represents an important 
opportunity to revisit foreground predictions in light of \PLANCK's superb, 
relatively artifact-free broadband data covering the entire sky and a wide 
range of frequencies. \cite{planckdust} has recently conducted a study modeling
\PLANCK~353 GHz, 545 GHz, 857 GHz and \IRAS~100$\mu$m emission with a single
MBB spectrum. Here we investigate the FDS99 two-component dust emission model 
as an alernative model for the dust spectral energy distribution (SED) composed
 of \PLANCK~and \IRAS~data. In doing so, we obtain $6.1'$ resolution maps of 
dust temperature and optical depth. Because we employ a model that has been 
validated with FIRAS data down to sub-mm wavelengths, we expect our best-fit 
parameters to be useful in constructing high-resolution predictions of dust 
emission over a very broad range of wavelengths. This includes low frequencies 
($<$350 GHz), which \cite{planckdust} caution their model may not adequately 
fit, and also measurements near the peak of the dust SED, for example 
\AKARI~140-160$\mu$m \citep{akari}. We also anticipate our derived optical 
depth map will serve as a useful comparison for extinction estimates based 
directly upon optical/NIR observations of stars.

%would be nice if i had a citation here for Eddie/Greg pan-starrs dust stuff

In $\S$\ref{sec:data} we discuss the data used throughout this study. In 
$\S$\ref{sec:prepro} we describe our preprocessing of the data to isolate 
thermal emission from Galactic dust. In $\S$\ref{sec:modeling} we describe the 
dust emission model we fit to the \PLANCK~and \IRAS~data. In 
$\S$\ref{sec:fitting} we provide details of the Markov chain Monte Carlo method
with which we have estimated the parameters of our model. In 
$\S$\ref{sec:results} we present the full-sky maps of dust temperature and 
optical depth thus obtained, and conclude in $\S$\ref{sec:conclusion}.

 \section{Data}
\label{sec:data}
All \PLANCK~data products utilized throughout this work are drawn from the 
\PLANCK~2013 release \citep{planck2013}. Specifically, we have made use 
of the 217 GHz, 353 GHz, 545 GHz and 857 GHz intensity maps which have been 
corrected for zodiacal light 
\citep[\texttt{R1.10\_nominal\_ZodiCorrected},][]{planckzodi}.

To incorporate measurements on the Wien side of the dust emission spectrum, 
we include 100$\mu$m data in our SED fits. In particular, we use the SFD 
reprocessing of \IRAS~100$\mu$m, which we will refer to as \verb|i100|, and at 
times by frequency as 3000 GHz. The \verb|i100| map has angular resolution of 
$6.1'$, and was constructed so as to contain only thermal 
emission from Galactic dust, with compact sources and zodiacal light removed, 
and its zero level tied to H\,\textsc{i}. We use the \verb|i100| map as-is, without any 
custom modifications.
%maybe clarify that throughout we use SFD i100 totally as-is, with no
% customization

%list other %important characteristics of i100 e.g. that it has been 
%zodi-corrected and relies on dirbe on scales larger than ???

\section{Preprocessing}
\label{sec:prepro}

The following subsections detail the processing steps we have applied to 
isolate thermal emission from Galactic dust in the \PLANCK~maps and prepare
these maps for our joint \IRAS/\PLANCK~SED fits.

\subsection{CMB Anisotropy Removal}
\label{sec:cmb}
We first addressed the CMB anisotropies before performing any of the 
interpolation/smoothing described in $\S$\ref{sec:ptsrc}/$\S$\ref{sec:smth}. 
The CMB anisotropies are effectively imperceptible in \PLANCK~857 GHz, but can 
be perceived at a low level in \PLANCK~545 GHz, and are prominent in 
\PLANCK~217 GHz and 353 GHz relative to the Galactic emission we wish to 
characterize, especially at high latitudes. To remove the CMB anisotropies, we 
have subtracted the Spectral Matching Independent Component Analysis 
(\verb|SMICA|) model from each of the \PLANCK~maps \citep{smica}, after 
applying appropriate unit conversions. Removal of the residual Solar dipole at 
low frequencies is discussed in $\S$\ref{sec:zp}.
% SMICA is subtracted from ALL Planck maps being used, including 545, 857 GHz

\subsection{Compact Sources}
\label{sec:ptsrc}
To mask compact sources, we use the SFD compact source mask, which was 
originally constructed to mask point sources and resolved compact objects
that are not part of the Galactic cirrus at 3000 GHz. Given our pixelization 
(see $\S$\ref{sec:pix}), 1.56\% of pixels are masked. We have inspected all of 
the maps used in this analysis and do not see any evidence that this compact 
source mask fails to adequately mask compact sources in the \PLANCK~data. In 
practice, we perform our dust SED fits in all pixels, even those flagged by 
the point source mask. Compact source affected pixels can be ignored or 
interpolated over downstream.
%maybe say that we're releasing a version of the map with point sources
%interpolated over by us as well

\subsection{Smoothing}
\label{sec:smth}
We wish to simultaneously fit \verb|i100| along with the four 
highest-frequency \PLANCK~bands. To properly combine these maps, they must have
 the same point spread function (PSF). \verb|i100|, with its $6.1'$ symmetric 
Gaussian PSF, is the lowest-resolution map we have included. To match PSFs, we 
have therefore smoothed each of the \PLANCK~maps under consideration to $6.1'$ 
resolution by considering each native map to have a symmetric Gaussian beam and
 smoothing by the appropriate symmetric Gaussian such that the resulting map 
has a FWHM of $6.1'$. The FWHM we assign to the native \PLANCK~maps is taken 
from \cite{planckbeam}, specifically $5.01'$ for 217 GHz, $4.86'$ for 353 GHz, 
$4.84'$ for 545 GHz, and $4.63'$ for 857 GHz.

\subsection{Zero-Point}
\label{sec:zp}
Although we wish to isolate and model thermal emission from Galactic dust, the
\PLANCK~maps contain additional components on large angular scales. At each 
frequency, there can exist an overall, constant offset that must be subtracted 
to set the zero level of Galactic dust. Additionally, faint residuals of the 
Solar dipole remain at low frequencies. At 353 GHz, 545 GHz, 857 GHz, we have 
adopted the zero-point offsets of \cite{planckdust} Table 1, which were derived
 by correlating \PLANCK~857 GHz versus H\,\textsc{i} at low column density, and
then by correlating the other \PLANCK~channels against the corrected 857 GHz 
map. We have not altered the \verb|i100| zero-point in any way. We have also 
removed the Solar dipole residual according to \cite{planckdust}, by 
subtracting a dipole pattern oriented toward 
$(l, b) = (263.99^{\circ}, 48.26^{\circ})$ and scaled according to their 
Table 1.

\begin{figure}
\begin{center}
\epsfig{file=scatter_857_217.eps, width=3.3in}
\caption{\label{fig:scatter} Scatter plot of \PLANCK~217 GHz versus \PLANCK~857
GHz. Left: before correcting for offset and residual Solar dipole. Right: after
correcting or the best-fit offset and dipole amplitude, as described in 
$\S$\ref{sec:zp}. The dashed red line shows the best-fit linear relationship
between 857 GHz and 217 GHz.}
\end{center}
\end{figure}

The offset and residual dipole parameters for \PLANCK~217 GHz were not 
available from \cite{planckdust} Table 1. In order to set the zero-point of 
Galactic emission in the 217 GHz map, we simultaneously fit the constant 
offset and residual dipole amplitude. Restricting to regions of low column 
density (H\,\textsc{i} $<$ 200 K km/s), we modeled the 217 GHz map as:

%in previous sentence, mention the specific cut on HI velocities considered

\begin{equation}
\mathcal{I}_{217, i} = a\mathcal{I}_{857, i} + b\mathcal{D}_{i} + c
\end{equation}

With $i$ denoting a single nside=64 HEALPix pixel in the maps 
$\mathcal{I}_{857}$, $\mathcal{I}_{217}$, and $\mathcal{D}$. Here 
$\mathcal{I}_{857}$ is the \PLANCK~857 GHz map with zodiacal emission, compact 
sources, and constant offset removed, smoothed to $1^{\circ}$ resolution. 
$\mathcal{I}_{217}$ is the corresponding $1^{\circ}$ resolution \PLANCK~217 
GHz map with zodiacal emission and compact sources removed. $\mathcal{D}$ is a
scaling of the Solar dipole with unit amplitude. Because $\sim$18,000 pixels
satisfy the low H\,\textsc{i} cut, we have an overconstrained linear model with
three parameters: a, b, and c. $a$ represents the slope of \PLANCK~217 GHz 
versus \PLANCK~857 GHz assuming they are linearly correlated, and is not
of any importance to our downstream analysis. $b$ represents the scaling of 
the Solar dipole that makes the 857 GHz vs. 217 GHz correlation tightest, and 
$c$ represents the constant offset necessary to make the zero level of the 217 
GHz map consistent with that of the 857 GHz map. Performing a linear least
squares fit with uniform weights and iterative outlier rejection, we find 
$a=4.33\times10^{-5}\ K/(MJy/sr)$, $b=-1.40\times10^{-5}\ K$, and 
$c=8.12\times10^{-5}\ K$. Note that $b$ represents 0.42\% of the Solar dipole 
amplitude, which is within the calibration uncertainty of the \PLANCK~217 GHz 
band. Figure \ref{fig:scatter} shows scatterplots of the 857 GHz versus 217 
GHz correlation before and after correcting for our best-fit offset and dipole 
residual. Not only is the improved correlation striking in this scatterplot, 
but the residual dipole also appears visually well-corrected in the 217 GHz 
map itself. Before performing thermal dust fits, we therefore subtract $c$ and 
$b\mathcal{D}$ from the 217 GHz map.

\subsection{Unit Conversions}

Some of the maps in native format are provided in thermodynamic $\delta T$, 
while others are provided in units of MJy/sr. Before performing any fits, 
we convert all measurements to MJy/sr following the guidelines of 
\cite{planckresponse}.

%i think it's table 5 that I specifically want to cite within the ``spectral
%response'' paper, but i need to check that those values are consistent
% with the ones i actually used from the other document ``HFI-unit conversion 
%and colour correction instructions v1.2''

\subsection{Molecular Emission}
One advantage of the FIRAS data used by FDS99 was its many narrow bins in 
frequency, which allowed those frequency ranges contaminated by strong 
molecular line emission to be discarded from the analysis. Unfortunately, with 
the broad \PLANCK~and \IRAS~bandpasses we do not have such a luxury. This means
 that we cannot truly separate out the thermal continuum of dust emission we 
wish to study from the molecular line emission. The most prominent molecular 
line emission in the Planck bands of interest arises from CO lines in the XX, 
YY bands. We investigated the possibility of subtracting out CO emission based 
on the template of \cite{planckco}. However, we found that the CO emission
rarely amounted to more than XX\% of the total intensity, and following
the treatment of \cite{planckdust}, $\S$2.1, we do not attempt to remove CO
emission.

% for CO line list see ~/wise/pro/planck_co_lines.pro

\begin{figure}
\begin{center}
\epsfig{file=sed.eps, width=3.3in}
\caption{\label{fig:sed} Planck SED for a single nside=2048 pixel in the 
Polaris region. Note that the two lowest-frequency data points (\PLANCK~100, 
143 GHz) were not used in our fit, while the three lowest-frequency data
points were not used in the \cite{planckdust} fit.}
\end{center}
\end{figure}

\begin{figure*}
\begin{center}
\epsfig{file=posteriors.eps, width=7.0in}
\caption{Gridded posterior PDFs for three nside=2048 pixels. Red crosses mark 
the best-fit parameters based on our Markov chain sampling of the posterior. 
The posterior distributions are in general extremely well-behaved, showing
no multimodality or other pathological qualities. Our MCMC parameter 
estimates coincide well with the peaks in the gridded posteriors. The 
colorscale is linear in $log(P)$, with black representing the maximum of 
$log(P)$ and white representing $max(log(P))-5$. Left: Low S/N pixel at 
high-latitude in Galactic north. Center: High S/N pixel in the Polaris region. 
Right: Low S/N pixel at high-latitude in the Galactic south.}
\end{center}
\end{figure*}

\section{Dust Emission Model}
\label{sec:modeling}

Our model is that of FDS99, which consists of two populations of dust 
grains emitting as MBBs, but with different temperatures and different 
emissivity power law indices $\beta$. The shape of this spectrum is given by:

\begin{equation}
M_{\nu} \propto f_{1}q_{1}(\frac{\nu}{\nu_{0}})^{\beta_1}B_{\nu}(T_1) + (1-f_{1})q_{2}(\frac{\nu}{\nu_0})^{\beta_2}B_{\nu}(T_2)
\end{equation}

With $\nu_0$ = 3000 GHz, $f_{1}$ = 0.0363, $\beta_1$ = 1.67, $\beta_2$ = 2.70, 
and $q_1/q_2$ = 13.0. For the parameters thus far specified, the two dust 
temperatures are further related by:

\begin{equation}
T_1 = 0.352T_2^{1.18}
\end{equation}

At each sky location, we have five independent intensity measurements (four ) 
FDS99 found typical values for $T_2$ and 
%basically just finish clarifying here that we want to start simple by fitting
%only two parameters for each pixel since we only have 5 intensity
% measurements per pixel in our baseline fits at 6.1 arcmin resoluiton

\section{Fitting Procedure}
\label{sec:fitting}

\subsection{Pixelization}
\label{sec:pix}
For the purpose of fitting, we break the sky into independent pixels of 
angular size $\sim$1.72$'$, defined by the HEALPix pixelization with nside=2048
\citep{healpix}. This is convenient because the Planck maps are provided in 
this format and this pixelization adequately samples the $6.1'$ resolution maps
under consideration. We reiterate that each pixel is fit independently of 
neighboring pixels.

\subsection{Finite Bandpass Correction}
The thermal dust emission model of $\S$\ref{sec:modeling} predicts the 
intensity $M_{\nu}$ in MJy/sr for any single frequency $\nu$. But in reality
we are comparing to broad bands typically hundreds of GHz wide. The shape of 
model spectrum $M_{\nu}$ is only dependent upon one parameter: the hot
dust temperature $T_2$. We have therefore computed bandpass correction factors
$b_{\nu_i}$ such that $M_{\nu_i}b_{\nu_i}$ is the predicted observation for the
\PLANCK~or \IRAS~band of interest. We have used the \PLANCK~transmission 
curves provided in the file \verb|HFI_RIMO_R1.10.fits| \citep{planckresponse}. 
For 3000 GHz, we have used the \IRAS~bandpass, following the treatment of
\cite{planckdust}. We note that it is not clear which bandpass should
properly be ascribed to the \IRAS/DIRBE composite 100$\mu$m maps under
consideration. % (here SFD \verb|i100|) and that of \cite{planckdust}).

% check that RIMO file i've used is actually the most up-to-date version

\subsection{Markov Chains}
As explained in $\S$\ref{sec:modeling}, the two free parameters in our 
per-pixel dust emission SED are the hot dust temperature $T_2$, which 
determinesthe shape of the SED, and the spectrum normalization. In practice,
the normalization is controlled via $\tilde{I}_{545}$, the amplitude of the 
model SED in MJy/sr at $\nu$ = 545 GHz. For each pixel, we run a 
Metropolis-Hastings Markov chain sampling the posterior probability as a 
function of the two parameters $T_2$ and $\tilde{I}_{545}$. For each pixel, we 
are sampling the posterior given by:

\begin{equation}
P(\tilde{I}_{545}, T_2|\{I\}) \propto \mathcal{L}(\{I\}|\tilde{I}_{545}, T_2)P(T_2)
\end{equation}

The likelihood function is given by:
\begin{equation}
\mathcal{L}(\{I\}|\tilde{I}_{545}, T_2) = \displaystyle\prod\limits_{i}\mathcal{N}(M_{\nu_{i}}(\tilde{I}_{545}, T_2)|I_{\nu_{i}}, \sigma_{\nu_i})
\end{equation}

%be more careful about notation re: color corrections

Where the product runs over $\nu_i$ in $\{217,\ 353,\ 545,\ 857,\ 3000\}$ GHz. 
For each pixel, the errors on each intensity measurement $\sigma_{\nu_i}$ are 
given by:

\begin{equation}
\sigma_{\nu_i} = \sqrt{c^2_{\nu_i}I^2_{\nu_i} + c^2_{\nu_i}CMB^2_{\nu_i} + (\delta O_{\nu_i})^2 + n^2_{\nu_i}}
\end{equation}

This error budget is the same as that of \cite{planckdust} Equation 7. 
$c_{\nu_i}$ is the multiplicative calibration uncertainty for the intensity in 
each band. We have adopted the $c_{\nu_i}$ values listed in \cite{planckdust} 
Table 1. For the 217 GHz band, we have assigned $c_{217 GHz}$ = 0.54, from 
Table 11 of \cite{planckcalib}. $CMB$ represents the CMB intensity 
subtracted at the relevant pixel. $\delta0_{\nu_i}$ represents the uncertainty
in the offset used to tie \PLANCK~intensity to HI. We have adopted the 
 $\delta0_{\nu_i}$ values of \cite{planckdust} Table 1. $n_{\nu_i}$ represents 
the instrumental noise in the pixel of interest, and is taken to be the square 
root of the \verb|ii_cov| parameter that accompanies each \PLANCK~intensity 
map. In order to obtain reasonable fitting results at high latitude where the 
data is noisy, we include a prior on $T_2$:

%what about the offset uncertainty for 217 GHz

\begin{equation}
P(T_2) = \mathcal{N}(T_2|\bar{T}_2, \sigma_{\bar{T}_2})
\end{equation}

With $\bar{T}_2$ = 16.2 K and $\sigma_{\bar{T}_2}$ = 1.4 K. In practice we 
always perform computations using the logarithm of relevant probabilities.
%mention something about the proposal distribution at some point?

\begin{figure}
\begin{center}
\epsfig{file=tcomparison.eps, width=3.3in}
\caption{\label{fig:comparison} Comparison of SFD temperature, two-component 
model $T_2$, and \cite{planckdust} temperature (labeled $T_{R1.20}$) for a
 $10.5^{\circ}\times8.3^{\circ}$  region centered about 
$(l,b) = (111.6^{\circ}, 20.3^{\circ})$. Note the 
differing colorscales. Both models incorporating \PLANCK~data clearly show a 
major improvement in angular resolution relative to SFD.}
\end{center}
\end{figure}

For each pixel, we initialize the Markov chain to have $T_2$ = $\bar{T}_2$ and
. We run 500 steps of burn-in and then 2000 steps during which we keep track
of the proposed $T_{2, j}$ and $I_{545, j}$ values. We use these 2000 values 
for each parameter to compute estimates of each parameter,  $T_2$ = 
$\langle T_{2, j} \rangle$, $I_{545}$ = $\langle I_{545, j} \rangle$, and of 
each parameter's variance, $\sigma^2_{T_2}$ = $\langle T^2_{2, j} \rangle-\langle T_{2, j} \rangle ^2$ and $\sigma^2_{I_{545}}$=$\langle I^2_{545, j} 
\rangle-\langle I_{545, j} \rangle ^2$

\section{Results}
\label{sec:results}

\subsection{Calculating $\tau$}

\subsection{Data Release}
We are releasing nside=2048 HEALPix maps in Galactic coordinates summarizing 
the results of of our two-component dust fits.

%provide maps at multiple different nsides ?
%   -> if so, actually smooth the maps before rebinning, or leave 
%      undersampled?
% provide uncertainty maps for tau and T2 ??

\subsection{Limitations}
Perhaps the dominant limitation of our fits is the presence of cosmic
infrared background anisotropies (CIBA) bleeding through into our maps of 
Galactic dust temperature and optical depth in high-latitude regions 
\citep{ciba}.
% CIBA !!!
% IRAS missing so that parts of sky not truly 6.1' resolution map

\subsection{Comparison to Planck Collaboration Dust Map}
%striping in IRAS/IRIS versus in SFD IRAS ... planck dust map obvious striping
An important difference between our maps and those of \cite{planckdust} is our 
use of SFD \verb|i100| instead of IRIS 100$\mu$m \citep{IRIS}. Residual 
\IRAS~striping at levels higher than that remaining in SFD \verb|i100| is 
clearly visible in the \cite{planckdust} maps.

%we actually have 6.1' arcminute resolution
It is true that we have smoothed the input intensity maps for this study 
to $6.1'$ FWHM, which is nominally lower resolution than the $5'$ 
\cite{planckdust} maps. However, as is apparent from figure 
\ref{fig:comparison}, the fact that \cite{planckdust} modeled $\beta$ at 
$0.5^{\circ}$ resolution has effectively blurred their dust temperature map.


\subsubsection{Omitting 217 GHz}
The \cite{planckdust} result used only 353, 545, 857, and 3000 GHz. When we 
repeat our analysis ignoring 217 GHz, so that we are using the same
set of frequencies, we find that no major changes arise. This actually
should be done/investigated though.

\section{Conclusion}
\label{sec:conclusion}

\begin{figure*} [ht]
\begin{center}
\epsfig{file=results.eps, width=7.0in}
\caption{Our best-fit $T_2$, binned to 27.5$'$ resolution}
\end{center}
\end{figure*}

This material is based upon work supported by the National Science Foundation 
Graduate Research Fellowship under Grant No.Based on observations obtained with
 Planck (http://www.esa.int/Planck), an ESA science mission with instruments 
and contributions directly funded by ESA Member States, NASA, and Canada. This 
research made use of the NASA Astrophysics Data System (ADS) and the IDL 
Astronomy User's Library at Goddard. \footnote{Available at 
\texttt{http://idlastro.gsfc.nasa.gov}}

\bibliographystyle{apj}
\bibliography{twocomp.bib}

\end{document}
