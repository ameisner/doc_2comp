\documentclass{emulateapj} 
 
\usepackage[dvipdf]{epsfig} 
\usepackage[dvips]{rotating}
\usepackage{subfigure}
\usepackage{mathrsfs}

\newcommand{\IRAS}{{\it IRAS}}
\newcommand{\HERSCHEL}{{\it Herschel}}
\newcommand{\SPITZER}{{\it Spitzer}}
\newcommand{\PLANCK}{{\it Planck}}
\newcommand{\AKARI}{{\it Akari}}
\newcommand{\COBE}{{\it COBE}}
\newcommand{\WISE}{{\it WISE}}

\bibpunct{(}{)}{;}{a}{}{,} 
 
\shorttitle{\PLANCK~Dust Model}
 
\shortauthors{Meisner \& Finkbeiner} 

\begin{document}

\title{Two-component thermal dust emission model: application to the 
{\it PLANCK} HFI maps}
\author{Aaron M. Meisner\altaffilmark{1,2}}
\author{Douglas P. Finkbeiner\altaffilmark{1,2}}
\altaffiltext{1}{Department of Physics, Harvard University, 17 Oxford Street, 
Cambridge, MA 02138, USA; ameisner@fas.harvard.edu}
\altaffiltext{2}{Harvard-Smithsonian Center for Astrophysics, 60 Garden St, 
Cambridge, MA 02138, USA; dfinkbeiner@cfa.harvard.edu}

\begin{abstract}
We present full-sky, $6.1'$ resolution maps of dust optical depth and 
temperature derived by fitting the \cite{FDS99} two-component dust emission 
model to the \PLANCK~217, 353, 545, and 857 GHz maps along with  
\IRAS~100$\mu$m data. This parametrization of the far infrared thermal dust 
spectrum as the sum of two modified blackbodies serves as an important 
alternative to the commonly adopted single modified blackbody dust emission 
model. Because our two-component model matches the dust spectrum near its 
peak, accounts for the flattening of the dust spectrum at millimeter 
wavelengths, and specifies dust temperature at 6.1$'$ FWHM, our model 
provides accurate, high-resolution thermal dust emission foreground predictions
 from 100 GHz to 3000 GHz. We also calibrate the derived optical depth to 
reddening at visible wavelengths, and compare the resulting reddening estimates
with those based on stellar spectra, as well as those of \cite{SFD} and 
\cite{planckdust} based on infrared emission.

%We also expect the derived optical depth to provide 
%valuable cross-checks for other dust-column related 
%data products, specifically the \cite{SFD} map, which assumed a single 
%modified blackbody with $\nu^2$ emissivity and included a temperature map of 
%inferior $\sim$$1.3^{\circ}$ resolution. We compare our best-fit two-component
% spectral energy distributions to those of the \cite{planckdust} 
% single-component parametrization.


%specify early on that this is 'Galactic dust', 'dust' is kind of a vague
% term ... on the other hand the Planck dust paper is pretty lax about this
%MENTION USE IN MAKING IR FOREGROUND PREDICTIONS
%mention more explicitly the enhanced temperature resolution relative to SFD??
%Our optical depth map will also provide a useful reference for dust maps 
%based on stellar colors rather than infrared emission.
\end{abstract}

\section{Introduction}
%why dust important in astronomy/astrophysics
The presence of Galactic interstellar dust affects
astronomical observations over a wide range of wavelengths. In the mid-infrared
and far-infrared, Galactic dust emission contributes significantly
to the total observed sky intensity. At optical and ultraviolet (UV) 
wavelengths, extinction by foreground dust grains attenuates the signal from 
extragalactic sources over the entire sky. Observations of interstellar dust 
emission/absorption can improve our understanding of the composition and 
physical conditions that prevail in the interstellar medium (ISM), an 
environment which plays a crucial role in Galactic evolution and star 
formation. Equally, or perhaps even more important to the practice of 
astronomy, however, is accurately accounting for dust as a foreground which 
reddens optical/UV observations of stars/galaxies and superimposes Galactic 
emission on low-frequency observations of the cosmic microwave background 
(CMB). 
%specify ``low frequency more precisely eventually
% somewhere mention re-radiation

Over the past decades, satellite observations have dramatically enhanced our
knowledge about infrared emission from the ISM. The \textit{Infrared Astronomy 
Satellite} (\IRAS), with its $\sim$4$'$ resolution, revolutionized the study of
 Galactic dust emission, first revealing the high-latitude ``infrared cirrus'' 
using 60$\mu$m and 100$\mu$m observations \citep{low84, wheelock94}. Later, the
Diffuse Infrared Background Experiment (DIRBE) aboard the \COBE~satellite 
provided complementary full-sky measurements at ten infrared wavelengths from 
1.25$\mu$m to 240$\mu$m, boasting a reliable zero point despite inferior 
$\sim$0.7$^{\circ}$ angular resolution \citep{boggess92}. \COBE/FIRAS 
\citep{firas} also provided full-sky infrared dust spectra at $7^{\circ}$ 
resolution in 213 narrow frequency bins between 30 GHz and 2850 GHz.

% DIRBE beam solid angles:
% http://lambda.gsfc.nasa.gov/product/cobe/dirbe_products.cfm

\citet[hereafter FDS99]{FDS99} used these FIRAS data to derive a globally 
best-fit model of dust emission applicable over a very broad range of 
frequencies. FDS99 showed that no model consisting of a single modified 
blackbody (MBB) could accurately match the FIRAS/DIRBE spectrum at both the 
Wien and Rayleigh-Jeans extremes. To fit the thermal dust spectrum between 100 
and 3000 GHz, FDS99 therefore proposed an emission model consisting of two 
MBBs, each with a different temperature and emissivity power law index. 
Physically, these two components might represent distinct dust grain species 
within the ISM, namely amorphous silicates and carbonaceous grains. By 
combining this best-fit two-component model with a custom reprocessing of 
\IRAS~100$\mu$m, FDS99 provided widely used foreground predictions with 
$6.1'$ FHWM, limited largely by their $1.3^{\circ}$ resolution DIRBE-based 
temperature correction.

% ^^final sentence about what FDS99 meant for foreground predictions (and how
% those predictions actually worked)

%did FDS99 actually use DIRBE (pretty sure they did)

% FDS used 100 GHz to 3000 GHz in their two-component fits
%mention which grain species is large vs. small

% should mention somewhere (probably in above paragraph, the range of
% frequencies over which FDS99 model was shown to be applicable

The \PLANCK~2013 data release \citep{planck2013} represents an important 
opportunity to revisit foreground predictions in light of \PLANCK's superb, 
relatively artifact-free broadband data covering the entire sky and a wide 
range of frequencies. \cite{planckdust} has recently conducted a study modeling
\PLANCK~353 GHz, 545 GHz, 857 GHz and \IRAS~100$\mu$m emission with a single
MBB spectrum. Here we investigate the FDS99 two-component dust emission model 
as an alernative parametrization for the dust spectral energy distribution 
(SED) composed  of \PLANCK~and \IRAS~data. In doing so, we obtain maps of dust 
temperature and optical depth, both at $6.1'$ resolution. Because we employ a 
model that has been validated with FIRAS data down to sub-mm wavelengths, we 
expect our best-fit parameters to be useful in constructing high-resolution 
predictions of dust emission over a very broad range of wavelengths. This 
includes low frequencies ($<$350 GHz), which \cite{planckdust} caution their 
model may not adequately fit, and also measurements near the peak of the dust 
SED, for example \AKARI~140-160$\mu$m \citep{akari}. We also anticipate our 
derived optical depth map will serve as a useful cross-check for extinction 
estimates based directly upon optical observations of stars
\citep[e.g.][]{green14} and as a baseline for next-generation dust extinction 
maps incorporating high-resolution, full-sky infrared data sets such as 
\WISE~\citep{wright10, meisner14} and \AKARI.

%would be nice if i had a citation here for Eddie/Greg pan-starrs dust stuff

In $\S$\ref{sec:data} we introduce the data used throughout this study. In 
$\S$\ref{sec:prepro} we describe our preprocessing of the \PLANCK~maps to 
isolate thermal emission from Galactic dust. In $\S$\ref{sec:modeling} we 
describe the dust emission model we fit to the \PLANCK~and \IRAS~data. In 
$\S$\ref{sec:fitting} we provide details of the Markov chain Monte Carlo (MCMC)
 method with which we have estimated the parameters of our model. In 
$\S$\ref{sec:results} we present the full-sky maps of dust temperature and 
optical depth thus obtained, and conclude in $\S$\ref{sec:conclusion}.

 \section{Data}
\label{sec:data}
All \PLANCK~data products utilized throughout this work are drawn from the 
\PLANCK~2013 release \citep{planck2013}. Specifically, we have made use 
of all six of the zodiacal light corrected HFI intensity maps
\citep[\texttt{R1.10\_nominal\_ZodiCorrected},][]{planckzodi}. Our 
full-resolution (6.1$'$ FWHM) SED fits neglect the two lowest HFI frequencies, 
100 and 143 GHz, as these have FWHM of 9.66$'$ and 7.27$'$ respectively.

To incorporate measurements on the Wien side of the dust emission spectrum, 
we include 100$\mu$m data in our SED fits. In particular, we use the 
\citet[henceforth SFD]{SFD} reprocessing of \IRAS~100$\mu$m, which we will 
refer to as \verb|i100|, and at times by frequency as 3000 GHz. The \verb|i100|
 map has angular resolution of $6.1'$, and was constructed so as to contain 
only thermal emission from Galactic dust, with compact sources and zodiacal 
light removed, and its zero level tied to H\,\textsc{i}. We use the \verb|i100|
map as is, without any custom modifications.
%maybe clarify that throughout we use SFD i100 totally as-is, with no
% customization

%list other %important characteristics of i100 e.g. that it has been 
%zodi-corrected and relies on dirbe on s

\section{Preprocessing}
\label{sec:prepro}

The following subsections detail the processing steps we have applied to 
isolate thermal emission from Galactic dust in the \PLANCK~maps and prepare
these maps for our joint \IRAS/\PLANCK~SED fits.

\subsection{CMB Anisotropy Removal}
\label{sec:cmb}
We first addressed the CMB anisotropies before performing any of the 
interpolation/smoothing described in $\S$\ref{sec:ptsrc}/$\S$\ref{sec:smth}. 
The CMB anisotropies are effectively imperceptible upon visual inspection 
in \PLANCK~857 GHz, but can be perceived at a low level in \PLANCK~545 GHz, and
are prominent from 100 to 353 GHz relative to the Galactic emission
we wish to characterize, especially at high latitudes. To remove the CMB 
anisotropies, we have subtracted the Spectral Matching Independent Component 
Analysis \citep[SMICA,][]{smica} model from each of the \PLANCK~maps, 
after applying appropriate unit conversions for the 545 and 857 GHz maps 
with native units of MJy/sr. Low-order corrections, particularly removal of the
residual Solar dipole, are discussed in $\S$\ref{sec:zp}.
% SMICA is subtracted from ALL Planck maps being used, including 545, 857 GHz
% WHAT IS THE ANGULAR RESOLUTION OF THE SMICA MAP, AND IS IT EVEN
% WELL-DEFINED??

\subsection{Compact Sources}
\label{sec:ptsrc}
%start by mentioning that SFD removed compact sources (outside of b +/- 5)
After subtracting the SMICA CMB model, we next proceeded to interpolate 
over compact sources, including both point sources and resolved galaxies. 
Removing compact sources at this stage is important as it prevents 
contamination of compact-source-free pixels in our downstream analyses which 
require smoothing of the \PLANCK~maps. SFD carefully removed point sources and 
galaxies from the \verb|i100| map everywhere outside of 
$|b|$$<$$5^{\circ}$. We do not perform any further modifications of the 
\verb|i100| map to account for compact sources. To mask compact sources in the 
\PLANCK~maps, we use the SFD compact source mask, which was originally 
constructed to mask point sources and resolved compact objects that are not 
part of the Galactic cirrus at 3000 GHz. Given our pixelization (see 
$\S$\ref{sec:pix}), 1.56\% of pixels are masked. 

%this compact source section still needs a lot of work

% probably best to refer to an appendix for details of the point source
% masking

\subsection{Smoothing}
\label{sec:smth}
For our full-resolution model, we wish to simultaneously fit \verb|i100| along 
with the four highest-frequency \PLANCK~bands. To properly combine these maps, 
they must have the same point spread function (PSF). \verb|i100|, with its 
$6.1'$ symmetric Gaussian beam, has the lowest angular resolution of the 
relevant maps. To match PSFs, we have therefore smoothed each of the 
\PLANCK~maps under consideration to \verb|i100| resolution by considering each 
native \PLANCK~map to have a symmetric Gaussian beam and smoothing by the 
appropriate symmetric Gaussian such that the resulting map has a  $6.1'$ FWHM. 
The FWHM we assign to the native \PLANCK~maps is taken from \cite{planckbeam}, 
specifically $5.01'$ for 217 GHz, $4.86'$ for 353 GHz, $4.84'$ for 545 GHz, and
 $4.63'$ for 857 GHz.

\subsection{Molecular Emission}
Because the FIRAS spectra consist of many narrow frequency bins, FDS99 were
able to discard the relatively small number of frequency intervals contaminated
by strong molecular line emission. Unfortunately, while the \PLANCK~data 
considered in this study are of high angular resolution, their broad bandpasses
do not offer the same luxury of spectral resolution in dealing with line 
emission. Therefore, we must subtract estimates of the molecular line 
contamination from each \PLANCK~band in order to best isolate the thermal 
continuum we wish to characterize. The most prominent molecular line emission 
in the \PLANCK~bands of interest arises from the three lowest CO rotational
transitions: J=1$\rightarrow$0 at 115 GHz, J=2$\rightarrow$1 at 230 GHz and 
J=3$\rightarrow$2 at 345 GHz, respectively affecting the \PLANCK~100, 217 and 
353 GHz bands. The J=1$\rightarrow$0 line also imparts a signal upon 
\PLANCK~143 GHz, but at a negligible level, $\sim$1000$\times$ fainter relative
to the dust continuum than J=1$\rightarrow$0 at 100 GHz.

%TODO : actually smooth the CO map to 6.1 arcmin
To correct for molecular emission, we employed the \PLANCK~Type 3 CO 
data product, which boasts the highest S/N 
among the available full-sky CO maps based on \PLANCK~HFI/LFI 
data \citep{planckco}. The native angular resolution of the Type 3 CO map is 
5.5$'$. We therefore began by smoothing the raw Type 3 CO map to $6.1'$, to 
match the PSF of the smoothed \PLANCK~intensity maps we wish to correct 
for molecular emission (for our full-resolution two-parameter fits, these are 
217 GHz and 353 GHz). 

%take a few sentences to mention how exactly the CO was scaled, i.e. conversion
% factors from K km/s and line ratio assumptions

We must apply the appropriate unit conversions to the $6.1'$ Type 3 CO
map before we can subtract it from the \PLANCK~intensity maps, which at the
frequencies of interest have native units of $K_{CMB}$. The Type 3 CO map
is provided in units of K$_{RJ}$ km/s of J=1$\rightarrow$0 emission. 
To convert this quantity to $K_{CMB}$, we assume that all of the CO emission 
arises from the $^{12}$CO isotope, and derived the \PLANCK-observed CO 
intensity in units of $K_{CMB}$ as follows:

\begin{equation}
I_{CO, \nu_i, N, N-1} = I_{3}F_{12CO, \nu_i, N, N-1} R_{N, N-1}
\end{equation}

% state what I_3 is 
Where $I_{CO, \nu_i, N, N-1}$ is the intensity in $K_{CMB}$ in \PLANCK~band 
$\nu_i$ due to the CO transition from J=$N$ to J=$(N$$-$1). $I_3$ represents 
the Type 3 CO amplitude in  K$_{RJ}$ km/s of J=1$\rightarrow$0 emission. The 
$F_{12CO, \nu_i, N, N-1}$ are conversion factors between K$_{RJ}$ km/s and 
$K_{CMB}$ for particular band/transition pairs. The relevant values adopted are
$F_{12CO, 100, 1, 0}$=1.478$\times$10$^{-5}$, 
$F_{12CO, 217, 2, 1}$=4.585$\times$$10^{-5}$, and 
$F_{12CO, 353, 3, 2}$=1.751$\times$$10^{-4}$.
$R_{N, N-1}$ represents the line ratio of the transition from J=$N$ to 
J=$(N$$-$1) relative to the J=1$\rightarrow$0. Thus, $R_{1,0}$=1, and we 
further adopt $R_{2,1}$=0.595 and $R_{3,2}$=0.297 based on \cite{planckco}. 
These line ratios are assumed to be constant over the entire sky. Formally, 
then, the CO contamination in band $\nu_i$ is given by:

\begin{equation}
I_{CO, \nu_i} = \sum\limits_{N} I_{CO, \nu_i, N, N-1}
\end{equation}

It turns out that, for each of the \PLANCK~bands in which CO emission is
non-negligible (100, 217 and 353 GHz), only a single $N$ contributes ($N$=1, 
$N$=2 and $N$=3, respectively). 

Unfortunately, the CO maps at $6.1'$ are rather noisy,
and the vast majority of the sky has completely negligible CO emission. Thus,
in order to avoid adding unnecessary noise outside of molecular cloud complexes
and at high latitudes, we have zeroed out low-signal regions of the CO Type 3
map. We identify  low-signal regions as those with $\mathcal{I}_3$$<$1 K$_{RJ}$
 km/s, where $\mathcal{I}_3$ is the Type 3 CO map smoothed to 0.25$^{\circ}$ 
resolution. This cut results in 90\% of the sky being unaffected by the CO 
correction, particularly the vast majority of the high Galactic latitude sky.

%did i ever spell out the acronyms HFI, LFI ????

%We investigated the possibility of subtracting out CO emission 
%based on the template of \cite{planckco}. However, we found that the CO 
%emission rarely amounted to more than XX\% of the total intensity, and 
%following the treatment of \cite{planckdust}, $\S$2.1, we do not attempt to 
%remove CO emission.

%We did not attempt to correct for higher J CO transitions beyond J=3->2

% for CO line list see ~/wise/pro/planck_co_lines.pro

\subsection{Zero-Point}
\label{sec:zp}

% for quantitative assessment of the zero-point difference due to
% flattening of 857 GHz vs. HI, see 
% ~/Desktop/dust/planck_hi/piecewise.pdf (on my Mac)

% -> at low HI, slope decreases by a factor of 1.88
%    the resulting difference in 857 GHz zero points is 0.373 MJy/sr

Although we wish to isolate and model thermal emission from Galactic dust, the
\PLANCK~maps contain additional components on large angular scales. At each 
frequency, there can exist an overall, constant offset that must be subtracted 
to set the zero level of Galactic dust. Additionally, faint residuals of the 
Solar dipole remain at low frequencies. We will address these issues by
separately solving two sub-problems: first, fixing the absolute zero level 
relative to external data, and second fitting per-band relative offsets 
and low order corrections by correlating the \PLANCK~bands against one another.
%this last sentence needs lots of work

\subsubsection{Absolute Zero Level}
\label{sec:zp_abs}
In \cite{planckdust}, the absolute zero level of infrared thermal dust emission
was set by requiring that \PLANCK~infrared emission tends to zero when 
H\,\textsc{i} is zero, assuming a linear correlation between these two 
measurements at low column density. However, this approach is somewhat
unsatisfying in that there appear to be different slopes of \PLANCK~857 GHz 
versus H\,\textsc{i} for different ranges of H\,\textsc{i} intensity. In 
particular, \PLANCK~857 GHz appears to ``flatten out'' at very low 
H\,\textsc{i}. More quantitavely, we have found using the LAB H\,\textsc{i} 
data \citep{lab} for $-72$$<$$v_{LSR}$$<$$+25$ that the best-fit slope for 
H\,\textsc{i}$<$70 K km/s is a factor of $\sim$1.9 lower than the best fit 
slope for 110 K km/s $<$H\,\textsc{i}$<$200 K km/s, and as a result the implied
zero-point offsets for \PLANCK~857 GHz differ by $\sim$0.37 MJy/sr.

\begin{figure}
\begin{center}
\epsfig{file=zp_857.eps, width=3.4in}
\caption{\label{fig:fdsref} Scatterplot of $\mathcal{F}_{857}$ versus 
$\mathcal{I}_{857}$, illustrating our absolute zero point determination by
comparison to the FDS99 prediction for \PLANCK~857 GHz.}
\end{center}
\end{figure}

\begin{figure}
\begin{center}
\epsfig{file=dipole_all.eps, width=3.4in}
\caption{\label{fig:dip}  Scatter plots of \PLANCK~143, 217, 353, and 545 GHz 
versus \PLANCK~857 GHz. Left: before correcting for offset and residual Solar 
dipole. Right: after correcting or the best-fit offset and dipole amplitude, as
described in $\S$\ref{sec:zp}. The dashed red line shows the best-fit linear 
relationship in each case.}
\end{center}
\end{figure}

\begin{figure*}
\begin{center}
\epsfig{file=quad_100.eps, width=6.5in}
\caption{\label{fig:quad} Summary of low-order corrections at 100 GHz.}
\end{center}
\end{figure*}

Because of these ambiguities in the relationship between 857 GHz and 
H\,\textsc{i} emission, we decided to instead constrain the 857 GHz zero level
by comparison to the FDS99 857 GHz prediction. This renders the \PLANCK~857 GHz
absolute zero level tied largely to the reliable FIRAS zero point, and to some 
lesser extent to H\,\textsc{i} through the \verb|i100| zero point. 
Specifically, we fit a linear model to the FDS99 predicted 857 GHz values as a 
function of \PLANCK~857 GHz. We used a version of the \PLANCK~857 GHz map with 
zodiacal light and point sources removed and smoothed to 1$^{\circ}$ 
resolution, which we will refer to as $\mathcal{I}_{857}$. For this analysis, 
we considered $\mathcal{I}_{857}$ to be the independent variable, as it has 
much higher S/N than the FDS99 prediction, henceforward referred to as 
$\mathcal{F}_{857}$. Note that $\mathcal{F}_{857}$ is not a monochromatic 
prediction, as we have incorporated the finite bandpass correction factors 
described in $\S$\ref{sec:bpcorr}, using the FDS99 temperature map to determine
the dust spectrum shape. We then rebin to $N_{side}$=64 and restrict to pixels 
with $\mathcal{I}_{857}$$<$2.15 MJy/sr. Since \PLANCK~857 GHz smoothed to 
degree resolution has very high S/N, we can safely perform such a cut on 
$\mathcal{I}_{857}$. Figure \ref{fig:fdsref} shows a scatterplot of 
$\mathcal{I}_{857}$ versus $\mathcal{F}_{857}$, with a moving median and linear
fit overplotted. The linear fit was performed with uniform weights and 
iterative outlier rejection. The best-fit linear model is given by 
$\mathcal{F}_{857}$=0.994$\mathcal{I}_{857}$$-$0.018. It is encouraging that 
the slope is quite close to unity, and the offset near zero.

The formal statistical errors on these parameters are small. Just as in the 
case of performing the regression versus  H\,\textsc{i}, systematic problems 
will dominate the uncertainty in the derived zero point offset. The systematics
likely to dominate our FDS-based zero level are imperfect zodiacal light 
corrections and the FDS99 temperature map. To quantify the level of systematic 
error, we split the sky into four quadrants, with the dividing boundaries being
 $b$=0$^{\circ}$ and $l$=0$^{\circ}$, $l$=180$^{\circ}$. We again restricted to
 $\mathcal{I}_{857}$$<$2.15 MJy/sr, and repeated the regression in each 
quadrant. The slopes varied from 0.948 to 0.993, with an rms of 0.0188, while 
the offsets varied from $-$0.056 MJy/sr to $+$0.089 MJy/sr, with an rms of 
0.0586 MJy/sr.

%mention that all of this assumes that residual CMB dipole
% and other low order-corrections are negligible at 857 GHz

% Maybe it's not necessary to mention the 545 GHz check ... although
% i should do the check eventually as to whether choosing 545 GHz
% for the absolute zero point is consistent with choosing 857 GHz
% to within the errors that i'm quoting

%We also repeated the full-sky FDS99 prediction versus \PLANCK~regression at 
%545 GHz. After 

%We also repeated the full-sky and quadrant regressions with the FDS99 
%predictions at 545 GHz (see bottom panel of Figure XX). We find the 
%slopes vary from XX to YY with an rms scatter of ZZ. The 545 GHz zero points
%have an rms scatter of SS, which is consistent with the 545 GHz zero 
%point rms implied by the scatter in 857 GHz zero point.

% actually evaluate the statistical errors !!!!

% what HI intensity does the 2.15 MJy/sr 857 GHz correspond to ?
%  ---> pretty high actually

\begin{deluxetable*}{rrrrrr} 
\tabletypesize{\scriptsize}
\tablecolumns{6} 
\tablewidth{0pc} 
\tablecaption{\label{table:offs} Planck/IRAS Map Properties \& Pre-processing} 
\tablehead{
\colhead{$\nu$ (GHz)} &
\colhead{Offset ($K_{CMB}$)} & 
\colhead{Dipole ($K_{CMB}$)} & 
\colhead{Quadrupole ($K_{CMB}$)} & 
\colhead{$c_{\nu}$} &
\colhead{FWHM ($'$)}
}
\startdata
100 & 1.69$\times$10$^{-5}$$\pm$3.61$\times$10$^{-7}$ & $-$1.08$\times$10$^{-5}$ & $-$7.17$\times$10$^{-6}$ &  0.0054 & 9.66 \\
143 & 3.58$\times$10$^{-5}$$\pm$7.58$\times$10$^{-7}$ & $-$1.08$\times$10$^{-5}$ & - & 0.0054 &  7.27 \\
217 & 7.79$\times$10$^{-5}$$\pm$2.60$\times$10$^{-6}$ & $-$1.40$\times$10$^{-5}$ & - & 0.0054 &  5.01 \\
353 & 2.76$\times$10$^{-4}$$\pm$1.95$\times$10$^{-5}$ & $-$3.08$\times$10$^{-5}$ & - & 0.012 & 4.86 \\
    & Offset (MJy/sr) & Dipole (MJy/sr) & Quadrupole (MJy/sr) & & \\ \cline{2-4} \\ [-2ex]
545 & 7.28$\times$10$^{-2}$$\pm$1.99$\times$10$^{-2}$ & 1.63$\times$10$^{-2}$  & - & 0.10 & 4.84 \\
857 & 1.82$\times$10$^{-2}$$\pm$6.02$\times$10$^{-2}$ &  - & - & 0.10 & 4.63 \\
3000 & 0.0$\pm$4.3$\times$10$^{-2}$ & - & - & 0.10 & 6.1
\enddata
\end{deluxetable*}

\subsubsection{Relative Zero Level}

In the course of this study we use not only \PLANCK~857 GHz, but also all of
the remaining \PLANCK~HFI bands, as well as \verb|i100|. To derive the 
absolute zero level offsets that must be applied to each of the five
lowest-frequency \PLANCK~bands, we perform a regression versus the 
\PLANCK~857 GHz map corrected for the absolute zero level offset derived in
 $\S$\ref{sec:zp_abs}. We assume no offset need be applied
to \verb|i100|, which has a zero level tied to H\,\textsc{i} by SFD.
% maybe need to restate the idea that we're deriving zero points by assuming
% that the band of interest must approach zero linearly as the reference
% (in this case 857 ghz) approaches zero.

The need for additional low-order corrections beyond a simple scalar offset 
became evident upon inspecting the HFI maps at 100-545 GHz. In particular, we 
noticed the presence of a residual dipole pattern, with an orientation 
consistent with that of the Solar dipole. To most precisely recover the 
amplitude of the residual dipole, it is necessary to have the 
highest-possible S/N in the independent variable of our regression. For 
this reason we have used 857 GHz as a reference for the 100-545 GHz bands, as 
opposed to the FDS99 predictions or H\,\textsc{i}. 

We have performed one regression per HFI band to simultaneously 
fit for the zero level offset, the slope relative to 857 GHz, and the residual 
dipole amplitude. For each 100-545 GHz HFI band, we restrict to regions of low
column density (H\,\textsc{i} $<$ 200 K\,km\,s$^{-1}$ for 
$-72$$<$$v_{LSR}$$<$$+25$ km\,s$^{-1}$) and fit the following model:

\begin{equation} \label{equ:dip}
\mathcal{I}_{\nu_i, p} = m\mathcal{I}_{857, p} + d\mathcal{D}_{p} + b
\end{equation}

With $p$ denoting a single nside=64 HEALPix pixel in the maps 
$\mathcal{I}_{857}$, $\mathcal{I}_{\nu_i}$, and $\mathcal{D}$. Here 
$\mathcal{I}_{857}$ is the \PLANCK~857 GHz map with zodiacal emission, compact 
sources, and the constant offset of $\S$\ref{sec:zp_abs} removed, smoothed to 
$1^{\circ}$ resolution. $\mathcal{I}_{\nu_i}$ is the corresponding $1^{\circ}$ 
resolution \PLANCK~HFI map with zodiacal emission and compact sources removed. 
In the context of Equation \ref{equ:dip}, $\nu_i \in$  
\{100, 143, 217, 353, 545\} GHz. Note that $\mathcal{I}_{\nu_i}$ is always in
the native units of the relevant \PLANCK~band. $\mathcal{D}$ is a scaling of 
the Solar dipole pattern oriented toward 
$(l, b) = (263.99^{\circ}, 48.26^{\circ})$, with unit amplitude. Because 
$\sim$18,000 pixels satisfy the low H\,\textsc{i} cut, we have an 
overconstrained linear model with three parameters: $m$, $d$, and $b$. $m$ 
represents the best-fit slope of \PLANCK~band $\nu_i$ versus \PLANCK~857 GHz 
assuming they are linearly related. $d$ gives the residual Solar dipole
amplitude, and its best-fit value represents the scaling of the Solar dipole 
that makes the \PLANCK~band $\nu_i$ versus 857 GHz correlation most tightly 
linear. $b$ represents the constant offset that must be subtracted from the 
band $\nu_i$ map to make its zero level consistent with that of the 857 GHz 
map. For each band $\nu_i$, we obtain estimates of $m$, $d$, and $b$ by 
performing a linear least squares fit with uniform weights and iterative 
outlier rejection. Figure \ref{fig:dip} shows scatterplots of the band $\nu_i$ 
versus 857 GHz correlation before and after correcting for the best-fit offset 
and residual dipole, for each $\nu_i \in$ \{143, 217, 353, 545\} GHz. Not only 
are the tightened correlations striking in these scatterplots, but the residual
dipole subtractions appear very successful in the two-dimensional band $\nu_i$ 
maps themselves. Before performing thermal dust fits, we therefore subtract the
best-fit $b$ and $d\mathcal{D}$ from each band $\nu_i$ map. The best-fit
offsets and residual dipole amplitudes are listed in Table \ref{table:offs}.

For \PLANCK~100 GHz, we found an additional low-order correction to be
necessary (see Figure \ref{fig:quad}). Therefore, for 100 GHz, we performed a 
modified version of the Equation \ref{equ:dip} fit, using the following model:

\begin{equation} \label{equ:quad}
\mathcal{I}_{100, p} = m\mathcal{I}_{857, p} + d\mathcal{D}_p + b + qQ_p
\end{equation}

We will refer to the additional term $qQ_p$ as the ``quadrupole correction''. 
The shape of the quadrupole template is given by:

\begin{equation}
Q_p = cos^2(\theta_p)
\end{equation}

Where $\theta_p$ is the angle of pixel $p$ relative to the Solar velocity 
vector. It is not clear physically why such a term should improve the 
correlation between $\mathcal{I}_{100}$ and $\mathcal{I}_{857}$. However, the 
scatter plots shown in Figure \ref{fig:quad} make it clear that such a term, 
with best-fit amplitude $q$=$-7.17$$\times$10$^{-6}$, dramatically improves the
quality of the linear correlation between the 100 GHz and 857 GHz channels. We 
tried re-running the fits at 143-545 GHz including the $Q_p$ quadrupole term, 
but in all cases found the best-fit quadrupole to be negligible, and yield no 
substantial tightening of the correlation versus 857 GHz.

% worth clarifying that, for quadrupole fit at 100 GHz, we still do the
% fit with uniform weights and iterative outlier rejection ?

% mention how offset/dipole values compare to those derived in the Planck
% team paper ?

\begin{figure}
\begin{center}
\epsfig{file=sed.eps, width=3.3in}
\caption{\label{fig:sed} Planck SED for a single nside=2048 pixel in the 
Polaris region. Note that the two lowest-frequency data points (\PLANCK~100, 
143 GHz) were not used in our fit, while the three lowest-frequency data
points were not used in the \cite{planckdust} fit.}
\end{center}
\end{figure}

% add error bars to this example SED plot

\begin{figure*}
\begin{center}
\epsfig{file=posteriors.eps, width=7.0in}
\caption{Gridded posterior PDFs for three nside=2048 pixels. Red crosses mark 
the best-fit parameters based on our Markov chain sampling of the posterior. 
The posterior distributions are in general extremely well-behaved, showing
no multimodality or other pathological qualities. Our MCMC parameter 
estimates coincide well with the peaks in the gridded posteriors. The 
colorscale is linear in $log(P)$, with black representing the maximum of 
$log(P)$ and white representing $max(log(P))-5$. Left: Low S/N pixel at 
high-latitude in Galactic north. Center: High S/N pixel in the Polaris region. 
Right: Low S/N pixel at high-latitude in the Galactic south.}
\end{center}
\end{figure*}

% overplot ellipses showing 1 sigma errors in posterior grayscales

\section{Dust Emission Model}
\label{sec:modeling}

% maybe a sentence to transition from preprocessing -> modeling
% e.g. With pre-processing completed, we turned to modeling the 
% thermal dust emission.


% mention that the two different emissivity power laws allow 
% the model to simultaneously explain narrow SED peak, and flattening
% of SED toward the millimeter
We adopt the thermal dust emission model of FDS99, which consists of two 
populations of dust grains emitting as MBBs, each with distinct 
emission/absorption characteristics. Importantly, the emissivity power law 
indices $\beta$ of the two species differ. The shape of this spectrum is given 
by:

\begin{equation}
M_{\nu} \propto f_{1}q_{1}(\frac{\nu}{\nu_{0}})^{\beta_1}B_{\nu}(T_1) + (1-f_{1})q_{2}(\frac{\nu}{\nu_0})^{\beta_2}B_{\nu}(T_2)
\end{equation}

% With $f_{1}$ = 0.0363, $\beta_1$ = 1.67, $\beta_2$ = 2.70, and $q_1/q_2$ = 
% 13.0.
 
In accordance with the conventions of FDS99, $\nu_0$ is chosen to be 3000 GHz.
In principle, one could imagine this two MBB model to be specified by seven 
free parameters for every line of sight: $T_1$, $T_2$, $\beta_1$, $\beta_2$, 
$f_1$, and $q_1$/$q_2$ and the normalization of $M_{\nu}$. However, as 
discussed in FDS99, $T_1$=$T_1$($T_2$, $\beta_1$, $\beta_2$, $q_1/q_2$) is 
fully specified by these other parameters under the assumption that the 
temperature of each species is determined by maintaining thermal equilibrium 
with the same interstellar radiation field (ISRF). $T_1$ is always related to 
$T_2$ via a simple power law, although the prefactor and exponent depend on the
 parameters $q_1/q_2$, $\beta_1$ and $\beta_2$. $q_1/q_2$ encodes the
relative emission/absorption cross-sections of the two dust species, while
$f_1$ dictates the relative amounts of each species present.

These considerations still leave us with six potentially free parameters per 
line of sight. Unfortunately, fitting this many parameters per spatial pixel is
not feasible for our full-resolution $6.1$$'$ fits, as these are constrained by
only five broadband intensity measurements. Hence, as in FDS99, we 
deem certain parameters to be ``global'', i.e. spatially constant over the 
entire sky. In our full-resolution five-band fits, we designate $\beta_1$, 
$\beta_2$, $f_1$ and $q_1/q_2$ to be spatially constant. This same approach 
was employed by FDS99, and the globally best-fit values obtained by FDS99 for 
these parameters are listed in the first row of Table \ref{tab:global}. With 
these global parameters, FDS99 found $T_2$$\approx$$16.2$K, 
$T_1$$\approx$$9.4$K to be typical at high-latitude. In $\S$\ref{sec:global}, 
we discuss the best-fit global parameters favored by the \PLANCK~HFI data; 
these are listed in the second row of Table \ref{tab:global}.

Fixing the four parameters of Table \ref{tab:global}, our full-resolution, 
five-band fits have two remaining free parameters per line of sight: the hot 
dust temperature $T_2$ determines the SED shape and the normalization of 
$M_{\nu}$ determines the SED amplitude. In the lower-resolution fits of 
$\S$\ref{sec:lores} which include all HFI bands, we will allow $f_1$ to be a 
third free parameter, still holding $\beta_1$, $\beta_2$, and $q_1/q_2$ fixed.

%explain that we take the ``hot'' dust temperature to be T2 rather than T1

% maybe point to the SED fit summary figure as an example of what
% 

% For the parameters thus far specified, the two dust temperatures are further 
% related by:

%\begin{equation}
%T_1 = 0.352T_2^{1.18}
%\end{equation}

% more important than the details of the above equation is the notion
% that, for any set of f1, q1/q2, beta1, beta2, the relationship between
% T1 and T2 is a simple power law

%the below statements seem like they might best be moved to a generic
%opening of the next section, stating our philosophy of breaking sky
%into pixels, each with 5 SED measurements, and that with this small
%number of constraints per pixel, it makes sense to have a baseline
% fit at full resolution that has only two free parameters

%FDS99 found $T_2 \approx 16.2$K, $T_1 \approx 9.4$K to be typical at 
%high-latitude. In the present formulation, there are only two free parameters 
%to describe an observed SED: the hot dust temperature $T_2$ determines the SED
%shape and the normalization of $M_{\nu}$ determines the SED amplitude. In 
%principle, we could allow additional parameters, such as $f_1$ and the ratio 
%$q_1/q_2$, to vary. However the number of free parameters in our baseline fit 
%at the full $6.1'$ resolution is limited by the fact that, for each sky 
%location, the observed SED contains only five independent intensity 
%measurements (\verb|i100| along with the four highest-frequency 
%\PLANCK~bands).

To calculate the optical depth in the context of this model, we assume
optically thin conditions, meaning that $\tau_{\nu}$ = $M_{\nu}/S_{\nu}$, where
$M_{\nu}$ is the appropriately scaled two-component intensity and the source
function is given by:

\begin{equation}
\label{eqn:source}
S_{\nu} = \frac{f_1q_1(\nu/\nu_0)^{\beta_1}B_{\nu}(T_1) + f_2q_2(\nu/\nu_0)^{\beta_2}B_{\nu}(T_2)}{f_1q_1(\nu/\nu_0)^{\beta_1}+f_2q_2(\nu/\nu_0)^{\beta_2}}
\end{equation}
%basically just finish clarifying here that we want to start simple by fitting
%only two parameters for each pixel since we only have 5 intensity
% measurements per pixel in our baseline fits at 6.1 arcmin resolution

\section{Global Model Parameters}
\label{sec:global}
%%%%

\begin{deluxetable}{rrrrrr} 
\tabletypesize{\scriptsize}
\tablecolumns{6} 
\tablewidth{0pc} 
\tablecaption{\label{tab:global} Global Model Parameters} 
\tablehead{
\colhead{Model} &
\colhead{Data} &
\colhead{$f_1$} &
\colhead{$q_1/q_2$} & 
\colhead{$\beta_1$} & 
\colhead{$\beta_2$}
}
\startdata
 1 & FIRAS+\verb|i100| & 0.0363 & 13.0 & 1.67 & 2.70 \\
 2 & \PLANCK+\verb|i100| & 0.0267 & 3.68 & 1.19 & 2.61 \\ [-2ex]
\enddata
\end{deluxetable}

% would be great to have uncertainties on each parameter as well (in the
% Planck case at least...) !!!!!!!!!!!!!!!

%%%%

In light of the \PLANCK~HFI data, it is possible to revisit the best-fit global
parameters of the two-component emission model. To derive the 
\{$\beta_1$, $\beta_2$, $f_1$, $q_1/q_2$\} that best represent the \PLANCK~HFI
maps, we employed a correlation slope analysis (see Figure XX). Specifically,
we correlated the observed \PLANCK~data against the appropriate FDS99 
prediction for each HFI band from 143-857 GHz. We omitted 100 GHz from 
consideration due to the sensitivity of the 100 GHz trend to our ad hoc
low order corrections (see Figure \ref{fig:quad}). All comparisons between
\PLANCK~and the FDS99 predictions were made after smoothing both to 
$1^{\circ}$, subtracting the SMICA CMB model, and ignoring SMICA inpainted 
pixels. For 100-857 GHz, the linearity between the FDS and \PLANCK~is
excellent.

%also emphasize that FDS99 predictions are broadband, not monochromatic
% (i think i can borrow a sentence about this from the absolute zero point 
% section above)

However, the slopes at low frequencies are not precisely unity, as would
be expected if the global two-component parameters derived from FIRAS by
FDS99 were also the best-fitting global parameters with respect to the 
\PLANCK~data set.

% maybe a table listing the correlation slopes and their uncertainties
% would be useful here

We then derive \{$\beta_1$, $\beta_2$, $f_1$, $q_1/q_2$\} as constrained by 
the correlation slopes and their uncertainties given in Table QQ by 
taking $T_2$=$16.2$K and running Markov chains to sample the posterior:

Figure ZZ shows a corner plot of the 150,000 samples. Clearly, there are
strong covariances, but even so several conclusions are clear: (1) a lower 
value of $q_1/q_2$ is favored by the \PLANCK~data than by the FIRAS data (2) 
the best-fit emissivity power law index of the hot dust component ($\beta_2$) 
is effectively identical to that found by FDS99 to within uncertainties and (3)
a somewhat lower $\beta_1$ is favored by \PLANCK~than by FIRAS, meaning the
\PLANCK-based SED is shallower toward the millimeter. Because $f_1$ is so 
strongly covariant with $\beta_1$, the difference of $f_1$ between the 
\PLANCK~and FIRAS best fits is not particularly significant.

% actually quantitatively assess, for a typical temperature (and frequency?)
% how the new f_1 value basically balances out the change in beta_1 ???

The MCMC fitting procedure described in $\S$\ref{sec:fitting} is performed
separately for both model 1 (FIRAS+\verb|i100|) and model 2 
(\PLANCK+\verb|i100|), each defined by their respective global parameters 
listed in Table \ref{tab:global}.

% does this statement accidentally give the impression that I do separate 
% MCMC fits, one in which i fit Planck/IRAS **data** and one in which
% i fit FIRAS/IRAS data ????

% multi-panel figure showing more scatterplots of Planck vs. FDS predictions

%\subsection{Hierarchical Approach?}

\section{Fitting Procedure}
\label{sec:fitting}

The following subsections detail our procedure for constraining the 
two-component dust emission model parameters which are permitted to vary
spatially. In the case of our full-resolution $6.1'$fits, these parameters
are the SED normalization and the hot dust temperature $T_2$. The same 
formalism will apply to our lower-resolution fits in which $f_1$ is also 
allowed to vary from one line of sight to another.

% sentence here summarizing what the following subsections will describe ?

\subsection{Pixelization}
\label{sec:pix}
For the purpose of fitting, we break the sky into $\sim$50 million 
non-overlapping pixels of angular size $\sim$1.72$'$, defined by the HEALPix 
pixelization in Galactic coordinates, with $N_{side}$=2048 \citep{healpix}. 
This pixelization is convenient because it is the same format in which the 
\PLANCK~HFI maps were released, and because it adequately samples the $6.1'$ 
FWHM maps under consideration in our full-resolution fits. Our procedure will 
fit the intensity measurements in each spatial pixel independently of those in 
neighboring pixels.

\subsection{Finite Bandpass Correction}
\label{sec:bpcorr}
The thermal dust emission model of $\S$\ref{sec:modeling} predicts the 
flux density $M_{\nu}$ in e.g. MJy/sr for any single frequency $\nu$. In 
practice, however, we wish to constrain our model using measurements in the 
broad \verb|i100|/\PLANCK~bandpasses, typically hundreds of GHz wide. We have 
therefore computed dimensionless finite bandpass correction factors to 
account for the MBB($T$, $\beta$) spectral shape and the transmission as a 
function of frequency:

\begin{equation} \label{equ:bpcorr}
b_{\nu_i}(T, \beta) = \frac{\int \mathcal{T}_{\nu_i}(\nu)\nu^{\beta}B_{\nu}(T) d\nu \bigg[\int \mathcal{T}_{\nu_i}(\nu) d\nu\bigg]^{-1}}{\nu_{i,c}^{\beta}B_{\nu_{i,c}}(T)}
\end{equation}

% integral limits ??? check planck paper to see what they did

% now i need to define what \nu_{i,c} is (it's the nominal central 
% frequency of band \nu_i) e.g. 857 GHz for Planck 857, etc..

% also need to define what \mathcal{T} is (relative transmission, 
% dimensionless)

Here $\nu_{i,c}$ is the nominal band center frequency of band $\nu_i$,  with 
$\nu_{i,c} \in$ \{100, 143, 217, 353, 545, 857, 3000\} GHz. 
$\mathcal{T}_{\nu_i}(\nu)$ represents the relative transmission as a function 
of frequency for band $\nu_i$. For the HFI maps, $\mathcal{T}_{\nu_i}(\nu)$ is 
given by the \PLANCK~transmission curves provided in the file 
\verb|HFI_RIMO_R1.10.fits| \citep{planckresponse}. For 3000 GHz, we have 
adopted the DIRBE transmission curve.

The broadband, two-component model prediction in band $\nu_i$, termed 
$\tilde{I}_{\nu_i}$, is then constructed as a linear combination of 
color-corrected MBB terms:

\begin{equation} \label{equ:broadband}
\tilde{I}_{\nu_i} \propto \sum_{k=1}^{2} b_{\nu_i}(T_k, \beta_k) f_k q_k (\nu_{i,c}/\nu_0)^{\beta_k} B_{\nu_{i,c}}(T_k)
\end{equation}

Note that, as in FDS99, $f_2$=(1$-$$f_1$) by convention. This finite bandpass 
correction approach allows us to predict the \textit{broadband} measurement 
$\tilde{I}_{\nu_i}$ by computing monochromatic flux densities at the central 
frequency $\nu_{i,c}$ and then multiplying by factors $b_{\nu_i}(T, \beta)$ 
which are interpolated off of a set of precomputed, one-dimensional lookup 
tables each listing $b_{\nu_i}(T, \beta)$ as a function of $T$. We avoided the 
need to interpolate in both $\beta$ and $T$ by computing only a small set of 
one dimensional correction factors for the particular set of $\beta$ values of 
interest ($\beta \in$ \{1.67, 2.7, 1.19, 2.61\}, see Table \ref{tab:global}).

This finite bandpass correction approach makes the MCMC sampling described in 
$\S$\ref{sec:mcmc} much more computationally efficient by circumventing the 
need to perform the integral in the numerator of Equation \ref{equ:bpcorr} 
on-the-fly for each proposed dust temperature. We have chosen to compute the 
bandpass corrections on a per-MBB basis because this approach is very 
versatile; all possible two-component models are linear combinations of MBBs, 
so we can apply all of our finite bandpass machinery even when we allow 
parameters other than temperature (e.g. $f_1$) to vary and thereby change the 
dust spectrum shape.

% We note that it is not clear which bandpass should properly be ascribed to 
% the \IRAS/DIRBE composite 100$\mu$m maps under consideration.

\subsection{Predicting the Observed SED}
With the formalism established in $\S$\ref{sec:modeling} and 
$\S$\ref{sec:bpcorr}, we can mathematically state the model we will use during 
MCMC sampling to predict the observed SED, which we will denote by 
\{$\tilde{I}_{\nu_i}$\}. The predicted broadband intensities are given by:

\begin{equation}
\label{eqn:inten}
\tilde{I}_{\nu_i} = \frac{\sum\limits_{k=1}^{2} b_{\nu_i}(T_k, \beta_k) f_k q_k (\nu_{i,c}/\nu_0)^{\beta_k} B_{\nu_{i,c}}(T_k) u_{\nu_i}^{-1}(T_k, \beta_k)}{\sum\limits_{k=1}^{2} b_{545}(T_k, \beta_k) f_k q_k (545 \textrm{GHz}/\nu_0)^{\beta_k} B_{545}(T_k)}\tilde{I}_{545}
\end{equation}

This equation is quite similar to Equation \ref{equ:broadband}, but with two 
important differences. First, the normalization of $\tilde{I}_{\nu_i}$ is now 
specified by $\tilde{I}_{545}$, which represents the broadband intensity 
measured in the \PLANCK~545 GHz band. The denominator serves to ensure that,
for the case of $\nu_i$=545 GHz, $\tilde{I}_{545}$ is self-consistent. Second, 
each term in the numerator is multiplied by a conversion factor 
$u_{\nu_i}^{-1}(T_k, \beta_k)$. This factor is necessary because some of the 
\PLANCK~maps of interest have units of $K_{CMB}$ (100-353 GHz), while the
remaining maps (545-3000 GHz) have units of MJy/sr. Complicating matters, the 
conversion from $K_{CMB}$ to MJy/sr depends on the spectrum under 
consideration; attempting to apply conversion factors to put all of the 
observed maps themselves into common units is not a particularly attractive 
option. Therefore, we adopt the strategy of predicting each band in its native 
units, whether MJy/sr or $K_{CMB}$. For this reason, we always evaluate 
$B_{\nu_{i,c}}$ in Equation \ref{eqn:inten} in MJy/sr and let $u_{\nu_i}$=1 
(dimensionless) for all $\nu_i$$\ge$545 GHz. For $\nu_i$$\le$353 GHz, 
$u_{\nu_i}(T_k, \beta_k)$ represents the appropriate conversion factor from 
$K_{CMB}$ to MJy/sr:

% equation for u goes here

\begin{equation}
u_{\nu_i}(T, \beta) = u_{\nu_i, IRAS} \times u_{\nu_i, IRAS \ to \ (T, \beta)}
\end{equation}

Here $u_{\nu_i, IRAS}$ gives the conversion factor from $K_{CMB}$ to MJy/sr
for band $\nu_i$ under the `\IRAS~convention', which corresponds to
a power law specrum with spectral index $\alpha$=$-1$. The values of 
$u_{\nu_i, IRAS}$ are listed in the second column of \cite{planckresponse} 
Table 6. $u_{\nu_i, IRAS \ to \ (T, \beta)}$ is a color correction factor 
relative to the \IRAS~convention for band $\nu_i$, assuming a MBB($T$, $\beta$)
spectrum. In practice, we retrieve $u_{\nu_i}(T, \beta)$ values in a fashion 
similar to that used for the $b_{\nu_i}(T, \beta)$ color corrections, using a 
set of precomputed one-dimensional look-up tables. To reiterate, for 
$\nu_i$$\le$353 GHz, $u_{\nu_i}(T_k, \beta_k)$ has units of (MJy/sr)/$K_{CMB}$,
and therefore $\tilde{I}_{\nu_{i}}$ will have different units depending on the 
band $\nu_i$, though $\tilde{I}_{\nu_{i}}$ always has the same units as the 
native band $\nu_i$ map.

%We will treat the SED normalization $\tilde{I}_{545}$ as a free 
%parameter in our fitting procedure. In our baseline fits, second free 
%parameter is $T_2$, which determines the shape of the SED through $M_{\nu}$ as
% well as $b_{\nu_i}(T_2)$.

% introduce nu_i notation in above paragraph with some mention of what it
% means

% check that RIMO file i've used is actually the most up-to-date version

\subsection{Sampling Parameters}
As discussed in $\S$\ref{sec:modeling}, our full-resolution fits
consider the ``global'' parameters $f_1$, $q_1/q_2$, $\beta_1$, $\beta_2$ to be
 fixed, while the dust spectrum normalization and dust temperature vary for
each line-of-sight. In order to predict the dust SED for a given pixel, we are 
therefore left with two remaining degrees of freedom, and must choose an 
appropriate set of two parameters to sample and thereby constrain via MCMC. To 
determine the SED normalization in each pixel, we draw samples in 
$\tilde{I}_{545}$, the broadband intensity in the 545 GHz bandpass, as defined 
in Equation \ref{eqn:inten}. With the four aforementioned global parameters 
fixed, the dust spectrum shape is determined entirely by the two dust 
temperatures, which are coupled. To constrain the dust temperatures,
 we sample in $T_2$, the hot dust temperature. For each sample in $T_2$, we 
compute the corresponding value of $T_1$, thereby specifying ($\beta$, $T$) for
both MBB components and thus fully specifying the SED shape.

For the lower resolution fits described in $\S$\ref{sec:lores}, we sample
in three parameters: $\tilde{I}_{545}$, $T_2$, and $f_1$.

% sentence or two mentioning reduced resolution fits where we sample in
% three dimensions (f_1, T_2, tilde{I}_{545})

\subsection{Markov Chains}
\label{sec:mcmc}

In our full-resolution fits, we use a MCMC approach to constrain the 
parameters $\tilde{I}_{545}$ and $T_2$. For each pixel, we run a 
Metropolis-Hastings Markov chain sampling the posterior probability as a 
function of the two parameters $\tilde{I}_{545}$ and $T_2$. More specifically, 
we are sampling the posterior given by:

% check whether I545 normalization parameter is the monochromatic
% 545 GHz intensity, or if it is for the Planck bandpass
% i think it should eventually be monochromatic, since that will be
% more useful to people who want SED predictions

\begin{equation}
\label{eqn:post}
P(\tilde{I}_{545}, T_2|\{I\}) \propto \mathcal{L}(\{I\}|\tilde{I}_{545}, T_2)P(T_2)
\end{equation}

% be careful about notational consistency with e.g. tildes, etc.. in this sect

Here \{$I$\} denotes the set of observed broadband intensities, $I_{217}$, 
$I_{353}$, etc. The likelihood function is given by:

\begin{equation}
\mathcal{L}(\{I\}|\tilde{I}_{545}, T_2) = \displaystyle\prod\limits_{i}\mathcal{N}(\tilde{I}_{\nu_{i}}|I_{\nu_{i}}, \sigma_{\nu_i})
\end{equation}

%be more careful about notation re: color corrections

Where the product runs over $\nu_i$ in $\{217,\ 353,\ 545,\ 857,\ 3000\}$ GHz
and $\tilde{I}_{\nu_i}$ is a function of $\tilde{I}_{545}$ and $T_2$ as 
specified by Equation \ref{eqn:inten}. For each pixel, the errors on each 
intensity measurement $\sigma_{\nu_i}$ are given by:

\begin{equation}
\sigma_{\nu_i}(p) = \sqrt{c^2_{\nu_i}I_{\nu_i}(p)^2 + c^2_{\nu_i}CMB(p)^2 + (\delta O_{\nu_i})^2 + n_{\nu_i}(p)^2}
\end{equation}

% is I_nu on RHS inclusive of offsets or is it the ``pure'' thermal dust
% intensity

This error budget is the same as that of \cite{planckdust} Equation 7. 
$c_{\nu_i}$ is the multiplicative calibration uncertainty for the intensity in 
each band. We have adopted the $c_{\nu_i}$ values listed in \cite{planckdust} 
Table 1. For the 217 GHz band, we have assigned $c_{217 GHz}$ = 0.54, from 
Table 11 of \cite{planckcalib}. $CMB$ represents the CMB intensity 
subtracted at the relevant pixel. $\delta O_{\nu_i}$ represents the uncertainty
in the offset used to tie \PLANCK~intensity to H\,\textsc{i}. We have adopted 
the  $\delta O_{\nu_i}$ values of \cite{planckdust} Table 1. $n_{\nu_i}$ 
represents the instrumental noise in the pixel of interest, and is taken to be 
the square root of the \verb|ii_cov| parameter that accompanies each 
\PLANCK~intensity map. In order to obtain reasonable fitting results at high 
latitude where the data is noisy, we include a prior on $T_2$:

%what about the offset uncertainty for 217 GHz

\begin{equation}
P(T_2) = \mathcal{N}(T_2|\bar{T}_2, \sigma_{\bar{T}_2})
\end{equation}

% maybe have a section justifying the T2 prior value of 16.2K, presumably using
% FIRAS as a cross-check?

With $\bar{T}_2$ = 16.2 K and $\sigma_{\bar{T}_2}$ = 1.4 K. In practice we 
always perform computations using logarithms of the relevant probabilities.
%mention something about the proposal distribution at some point?

\begin{figure}
\begin{center}
\epsfig{file=tcomparison.eps, width=3.3in}
\caption{\label{fig:comparison} Comparison of SFD temperature, two-component 
model $T_2$, and \cite{planckdust} temperature (labeled $T_{R1.20}$) for a
 $10.5^{\circ}\times8.3^{\circ}$  region centered about 
$(l,b) = (111.6^{\circ}, 20.3^{\circ})$. Note the differing colorscales. Both 
models incorporating \PLANCK~data clearly show a major improvement in angular 
resolution relative to SFD.}
\end{center}
\end{figure}

For each pixel, we initialize the Markov chain with parameters 
$\tilde{I}_{545}$ = $I_{545}$ and $T_2$ consistent with the FDS99 
DIRBE 100$\mu$m/240$\mu$m ratio map $\mathscr{R}$. We run 500 steps of burn-in 
and then 2000 steps during which we keep track of the proposed $T_{2, j}$ and 
$\tilde{I}_{545, j}$ values at the $j^{th}$ step since the end of burn-in. From
 these 2000 samples, we compute estimates of each parameter, 
$T_2$ = $\langle T_{2, j} \rangle$, $\tilde{I}_{545}$ = 
$\langle \tilde{I}_{545, j} \rangle$, and of each parameter's variance, 
$\sigma^2_{T_2}$ = $\langle T^2_{2, j} \rangle-\langle T_{2, j} \rangle ^2$ and
 $\sigma^2_{\tilde{I}_{545}}$=$\langle \tilde{I}^2_{545, j} 
\rangle-\langle \tilde{I}_{545, j} \rangle ^2$. We also compute the
 corresponding estimate of the monochromatic two-component intensity at 545 
GHz, $M_{545}$ = $\langle M_{545, j} \rangle$ = 
$\langle \tilde{I}_{545, j}/b_{545}(T_{2,j}) \rangle$  and its variance. 
$M_{545}$ is of more general utility in making foreground predictions because 
it provides the normalization for the two-component spectrum $M_{\nu}$, 
independent of the \PLANCK~545 GHz bandpass. Lastly, we compute the 545 GHz
optical depth as $\tau_{545}$ = $\langle \tau_{545, j} \rangle$ = 
$\langle M_{545, j}/S_{545, j} \rangle$ and its variance, with the source 
function $S_{545}$ calculated according to Equation \ref{eqn:source}.

%in the above need to mention/introduce the use of ``j'' to denothe the j'th
%step of the MH sampling **after** burn-in has finished

% actually we I think we want to keep track of $\tilde{I}_{545}$/b_{545}
% which is the monochromatic value, since that is most useful downstream
% when making foreground predictions that typically won't have anything
% to do with Planck bandpasses

\subsection{Low-resolution Fits}
\label{sec:lores}
%mention that the pixelization changes ---> to lower HEALpix nside
%mention that over some region of the sky with high signal and high
%S/N we try fitting t2, normalization and one of q1/q2, f_1
%using more bands (including 143, and maybe 100 GHz)

As mentioned previously, our baseline fits fix several parameters which
in principle could vary spatially. If we are willing to sacrifice angular
resolution in order to incorporate lower-frequency \PLANCK~bands and
enhance signal to noise, we can introduce additional free parameters
and run the Markov chains in higher dimensions.

We ran the Markov chains in three dimensions by allowing 
$f_1$ to be a third free parameter. We added to our standard posterior a 
term involving.

\section{Calibrating to Optical Reddening}
\label{sec:ebv}

%FILL THIS SECTION IN !!!!!

% scatterplot figure showing linear regression of EBV vs. tau
\begin{figure}
\begin{center}
\epsfig{file=calib_ebv.eps, width=3.3in}
\caption{\label{fig:calib} Linear fit of $E(B-V)_{SSPP}$ as function of
$\tau_{545}$.}
\end{center}
\end{figure}

% need 2d histogram figure that is as closely analogous as possible to
% Figure 2 of Eddie's draft at : 
% http://faun.rc.fas.harvard.edu/eschlafly/2dmap/2dmap.pdf

\section{Systematics}

\subsection{CIBA}
Perhaps the dominant limitation of our fits is the presence of cosmic
infrared background anisotropies (CIBA) bleeding through into our maps of 
Galactic dust temperature and optical depth in high-latitude regions 
\citep{ciba}.
\subsubsection{Simulations}

\subsubsection{Optimizing Relative Weights}

%not sure if below subsection is necessary, idea of giving zero weight to 217 
%GHz may belong in above subsection

%\subsubsection{Omitting 217 GHz}
%The \cite{planckdust} result used only 353, 545, 857, and 3000 GHz. When we 
%repeat our analysis ignoring 217 GHz, so that we are using the same
%set of frequencies, we find that no major changes arise. This actually
%should be done/investigated though.

\subsection{Zero Point}
Simulate the effect of getting zero points wrong to quantiy this
systematic a bit.

\subsection{$T_{2}$ Prior}
Use FIRAS data to justify $T_2$ prior. Could I use the DIRBE 100/240 micron
ratio to get a $1.3^{\circ}$ resolution prior on $T_2$?

\subsection{Other Limitations}
\IRAS~is missing in some parts of sky which will therefore not truly have 6.1' 
resolution.
%any other miscellaneous issues?

\section{Results}
\label{sec:results}

%\subsection{Calculating $\tau$}

% this subsection should probably be moved to the ``dust emission model'' 
% section of the paper eventually

%at some point in this should make statements re: the fact that in terms
%of predicting *emission* the conversion to tau is unnecessary and
%and actually the raw uncertainty on Ghz 545 amplitude is the relevant quantity

\subsection{Data Release}
We are releasing nside=2048 HEALPix maps in Galactic coordinates summarizing 
the results of of our two-component dust fits.

%software utilities
%provide maps at multiple different nsides ?
%   -> if so, actually smooth the maps before rebinning, or leave 
%      undersampled?
% provide uncertainty maps for tau and T2 ??


\subsection{Comparison to Planck Collaboration Dust Map}
%striping in IRAS/IRIS versus in SFD IRAS ... planck dust map obvious striping
An important difference between our maps and those of \cite{planckdust} is our 
use of SFD \verb|i100| instead of IRIS 100$\mu$m \citep{IRIS}. Residual 
\IRAS~striping at levels higher than that remaining in SFD \verb|i100| is 
clearly visible in the \cite{planckdust} maps.

%we actually have 6.1' arcminute resolution
It is true that we have smoothed the input intensity maps for this study 
to $6.1'$ FWHM, which is nominally lower resolution than the $5'$ 
\cite{planckdust} maps. However, as is apparent from figure 
\ref{fig:comparison}, the fact that \cite{planckdust} modeled $\beta$ at 
$0.5^{\circ}$ resolution has effectively blurred their dust temperature map.

\begin{figure}
\begin{center}
\epsfig{file=f1_1deg.eps, width=3.3in}
\caption{\label{fig:f1} The results of our low-resolution fit with $f_1$
allowed to vary, but with a strong prior 
$P(f_1)=\mathcal{N}(f_1|0.0363, 0.003)$.}
\end{center}
\end{figure}

\begin{figure}
\begin{center}
\epsfig{file=f1_lambert.eps, width=3.3in}
\caption{\label{fig:f1lambert} The results of our low-resolution fit with $f_1$
allowed to vary, shown in Lambert projection to highlight the salient features 
near the north Galactic pole.}
\end{center}
\end{figure}

%\subsection{Validation with FIRAS}

\section{Conclusion}

% should mention that one definitive improvement of this work relative
% to FDS99 is that there are uncertainty/covariances provided for
% the derived parameters !!!!

\label{sec:conclusion}

\begin{figure*} [ht]
\begin{center}
\epsfig{file=results.eps, width=7.0in}
\caption{Our best-fit $T_2$, binned to 27.5$'$ resolution}
\end{center}
\end{figure*}

This material is based upon work supported by the National Science Foundation 
Graduate Research Fellowship under Grant No.Based on observations obtained with
 Planck (http://www.esa.int/Planck), an ESA science mission with instruments 
and contributions directly funded by ESA Member States, NASA, and Canada. This 
research made use of the NASA Astrophysics Data System (ADS) and the IDL 
Astronomy User's Library at Goddard. \footnote{Available at 
\texttt{http://idlastro.gsfc.nasa.gov}}

\bibliographystyle{apj}
\bibliography{twocomp.bib}

\end{document}
