\documentclass{emulateapj} 
 
\usepackage[dvipdf]{epsfig} 
\usepackage[dvips]{rotating}
\usepackage{subfigure}
\usepackage{mathrsfs}

\newcommand{\IRAS}{{\it IRAS}}
\newcommand{\HERSCHEL}{{\it Herschel}}
\newcommand{\SPITZER}{{\it Spitzer}}
\newcommand{\PLANCK}{{\it Planck}}
\newcommand{\AKARI}{{\it Akari}}
\newcommand{\COBE}{{\it COBE}}
\newcommand{\WISE}{{\it WISE}}

\bibpunct{(}{)}{;}{a}{}{,} 
 
\shorttitle{\PLANCK~Dust Model}
 
\shortauthors{Meisner \& Finkbeiner} 

\begin{document}

\title{Two-component thermal dust emission model: application to the 
{\it PLANCK} HFI maps}
\author{Aaron M. Meisner\altaffilmark{1,2}}
\author{Douglas P. Finkbeiner\altaffilmark{1,2}}
\altaffiltext{1}{Department of Physics, Harvard University, 17 Oxford Street, 
Cambridge, MA 02138, USA; ameisner@fas.harvard.edu}
\altaffiltext{2}{Harvard-Smithsonian Center for Astrophysics, 60 Garden St, 
Cambridge, MA 02138, USA; dfinkbeiner@cfa.harvard.edu}

\begin{abstract}
We present full-sky, $6.1'$ resolution maps of dust optical depth and 
temperature derived by fitting the \cite{FDS99} two-component dust emission 
model to the \PLANCK~217, 353, 545, and 857 GHz maps along with  
\IRAS~100$\mu$m data. This parametrization of the far-infrared thermal dust 
spectrum as the sum of two modified blackbodies serves as an important 
alternative to the commonly adopted single modified blackbody dust emission 
model. Because our two-component model matches the dust spectrum near its 
peak, accounts for the flattening of the dust spectrum at millimeter 
wavelengths, and specifies dust temperature at 6.1$'$ FWHM, our model 
provides accurate, high-resolution thermal dust emission foreground predictions
from 100 GHz to 3000 GHz. We also calibrate the derived optical depth to 
reddening at visible wavelengths, and compare the resulting reddening estimates
with those based on stellar spectra, as well as those of \cite{SFD} and 
\cite{planckdust} based on infrared emission.

% might consider changing the word `specifies' in penultimate sentence

% specify early on that this is 'Galactic dust', 'dust' is kind of a vague
% term ... on the other hand the Planck dust paper is pretty lax about this

% mention more explicitly the enhanced temperature resolution relative to SFD??

\end{abstract}

\section{Introduction}
The presence of Galactic interstellar dust affects
astronomical observations over a wide range of wavelengths. In the mid-infrared
and far-infrared, Galactic dust emission contributes significantly
to the total observed sky intensity. At optical and ultraviolet (UV) 
wavelengths, extinction by foreground dust grains attenuates the signal from 
extragalactic sources over the entire sky. Observations of interstellar dust 
emission/absorption can improve our understanding of the composition and 
physical conditions that prevail in the interstellar medium (ISM), an 
environment which plays a crucial role in Galactic evolution and star 
formation. Equally, or perhaps even more important to the practice of 
astronomy, however, is accurately accounting for dust as a foreground which 
reddens optical/UV observations of stars/galaxies and superimposes Galactic 
emission on low-frequency observations of the cosmic microwave background 
(CMB). 

% mention re-radiation somewhere ?

Over the past decades, satellite observations have dramatically enhanced our
knowledge about infrared emission from the ISM. The \textit{Infrared Astronomy 
Satellite} (\IRAS), with its $\sim$4$'$ resolution, revolutionized the study of
 Galactic dust emission, first revealing the high-latitude ``infrared cirrus'' 
using 60$\mu$m and 100$\mu$m observations \citep{low84, wheelock94}. Later, the
Diffuse Infrared Background Experiment (DIRBE) aboard the \COBE~satellite 
provided complementary full-sky measurements at ten infrared wavelengths from 
1.25$\mu$m to 240$\mu$m, boasting a reliable zero point despite inferior 
$\sim$0.7$^{\circ}$ angular resolution \citep{boggess92}. \COBE/FIRAS 
\citep{firas} also provided full-sky infrared dust spectra at $7^{\circ}$ 
resolution in 213 narrow frequency bins between 30 GHz and 2850 GHz.

% DIRBE beam solid angles:
% http://lambda.gsfc.nasa.gov/product/cobe/dirbe_products.cfm

\citet[hereafter FDS99]{FDS99} used these FIRAS data to derive a globally 
best-fit model of dust emission applicable over a very broad range of 
frequencies. FDS99 showed that no model consisting of a single modified 
blackbody (MBB) could accurately match the FIRAS/DIRBE spectrum at both the 
Wien and Rayleigh-Jeans extremes. To fit the thermal dust spectrum between 100 
and 3000 GHz, FDS99 therefore proposed an emission model consisting of two 
MBBs, each with a different temperature and emissivity power law index. 
Physically, these two components might represent distinct dust grain species 
within the ISM, or they might simply provide a convenient fitting function. By 
combining this best-fit two-component model with a custom reprocessing of 
\IRAS~100$\mu$m, FDS99 provided widely used foreground predictions with $6.1'$ 
FHWM, limited largely by their $1.3^{\circ}$ resolution DIRBE-based temperature
correction.

% additional sentence suggested by doug : 
% The fact that the range of optical
% properties of small silicate particles measured in the lab (Agladze et
% al. 1996) spans the range needed by the FDS99 model makes plausible
% that the distinct species are real.

The \PLANCK~2013 data release \citep{planck2013} represents an important 
opportunity to revisit foreground predictions in light of \PLANCK's superb, 
relatively artifact-free broadband data covering the entire sky and a wide 
range of frequencies. \cite{planckdust} has recently conducted a study modeling
\PLANCK~353 GHz, 545 GHz, 857 GHz and \IRAS~100$\mu$m emission with a single
MBB spectrum. Here we investigate the FDS99 two-component dust emission model 
as an alernative parametrization for the dust spectral energy distribution 
(SED) composed  of \PLANCK~and \IRAS~data. In doing so, we obtain maps of dust 
temperature and optical depth, both at $6.1'$ resolution. Because we employ a 
model that has been validated with FIRAS data down to sub-mm wavelengths, we 
expect our best-fit parameters to be useful in constructing high-resolution 
predictions of dust emission over a very broad range of wavelengths. This 
includes low frequencies ($<$350 GHz), which \cite{planckdust} caution their 
model may not adequately fit, and also measurements near the peak of the dust 
SED, for example \AKARI~140-160$\mu$m \citep{akari}. We also anticipate our 
derived optical depth map will serve as a useful cross-check for extinction 
estimates based directly upon optical observations of stars
\citep[e.g.][]{schlafly14} and as a baseline for next-generation dust 
extinction maps incorporating high-resolution, full-sky infrared data sets such
as \WISE~\citep{wright10, meisner14} and \AKARI.

% might be more appropriate to cite the 2D dust map Eddie is making rather
% than the `methods' paper

In $\S$\ref{sec:data} we introduce the data used throughout this study. In 
$\S$\ref{sec:prepro} we describe our preprocessing of the \PLANCK~maps to 
isolate thermal emission from Galactic dust. In $\S$\ref{sec:modeling} we 
explain the dust emission model we fit to the \PLANCK~and \IRAS~data. In 
$\S$\ref{sec:global} we describe our constraints on this model's global 
parameters derived from the \PLANCK~High Frequency Instrument (HFI) maps. In 
$\S$\ref{sec:fitting} we detail the Markov chain Monte Carlo (MCMC) method with
which we have estimated the spatially varying parameters of our model. In 
$\S$\ref{sec:ebv} we calibrate our derived optical depth to reddening at 
optical wavelengths. In $\S$\ref{sec:release} we present the full-sky maps of 
dust temperature and optical depth we have obtained, and conclude in 
$\S$\ref{sec:conclusion}.

\section{Data}
\label{sec:data}
All \PLANCK~data products utilized throughout this work are drawn from the 
\PLANCK~2013 release \citep{planck2013}. Specifically, we have made use 
of all six of the zodiacal light corrected HFI intensity maps
\citep[\texttt{R1.10\_nominal\_ZodiCorrected},][]{planckzodi}. Our 
full-resolution (6.1$'$ FWHM) SED fits neglect the two lowest HFI frequencies, 
100 and 143 GHz, as these have FWHM of 9.66$'$ and 7.27$'$ respectively.
% in last sentence, reference Table 1 ?

To incorporate measurements on the Wien side of the dust emission spectrum, 
we include 100$\mu$m data in our SED fits. In particular, we use the 
\citet[henceforth SFD]{SFD} reprocessing of \IRAS~100$\mu$m, which we will 
refer to as \verb|i100|, and at times by frequency as 3000 GHz. The \verb|i100|
 map has angular resolution of $6.1'$, and was constructed so as to contain 
only thermal emission from Galactic dust, with compact sources and zodiacal 
light removed, and its zero level tied to H\,\textsc{i}. We use the \verb|i100|
map as is, without any custom modifications.

\section{Preprocessing}
\label{sec:prepro}

The following subsections detail the processing steps we have applied to 
isolate Galactic dust emission in the \PLANCK~maps in preparation for our joint
\PLANCK/\IRAS~SED fits. % reference fitting section at end of this sentence ?

\subsection{CMB Anisotropy Removal}
\label{sec:cmb}
We first addressed the CMB anisotropies before performing any of the 
interpolation/smoothing described in $\S$\ref{sec:ptsrc}/$\S$\ref{sec:smth}. 
The CMB anisotropies are effectively imperceptible upon visual inspection 
of \PLANCK~857 GHz, but can be perceived at a low level in \PLANCK~545 GHz, and
are prominent at 100-353 GHz relative to the Galactic emission
we wish to characterize, especially at high latitudes. To remove the CMB 
anisotropies, we have subtracted the Spectral Matching Independent Component 
Analysis \citep[SMICA,][]{smica} model from each of the \PLANCK~maps, 
applying appropriate unit conversions for the 545 and 857 GHz maps with native 
units of MJy/sr. Low-order corrections, particularly removal of the residual 
Solar dipole, are discussed in $\S$\ref{sec:zp}.

% NEED TO MENTION SMOOTHING OF THE SMICA MAP TO 6.1 ARMCIN !!!!!!

\subsection{Compact Sources}
\label{sec:ptsrc}
% this compact source section still needs a lot of work
After subtracting the SMICA CMB model, we interpolate over compact sources, 
including both point sources and resolved galaxies. Removing compact sources at
this stage is important as it prevents contamination of compact-source-free 
pixels in our downstream analyses which require smoothing of the \PLANCK~maps. 
SFD carefully removed point sources and galaxies from the \verb|i100| map 
everywhere outside of $|b|$$<$$5^{\circ}$. We do not perform any further 
modifications of the \verb|i100| map to account for compact sources. To mask 
compact sources in the \PLANCK~217-857 GHz maps, we use the SFD compact source 
mask. At 100, 143 GHz we use the compact source masks provided by the 
\PLANCK~collaboration in the file \verb|HFI_Mask_PointSrc_2048_R1.10.fits|. 
Given our pixelization (see $\S$\ref{sec:pix}), 1.56\% of pixels are masked 
at 217-857 GHz (1.05\%, 1.02\% at 100, 143 GHz) 

% best to refer to an appendix for details of the point source masking ?

\subsection{Smoothing}
\label{sec:smth}
For our full-resolution model, we wish to simultaneously fit \verb|i100| along 
with the four highest-frequency \PLANCK~bands. To properly combine these maps, 
they must have the same point spread function (PSF). \verb|i100|, with its 
$6.1'$ symmetric Gaussian beam, has the lowest angular resolution of the 
relevant maps. To match PSFs, we have therefore smoothed each of the 
\PLANCK~maps under consideration to \verb|i100| resolution by considering each 
native \PLANCK~map to have a symmetric Gaussian beam and smoothing by the 
appropriate symmetric Gaussian such that the resulting map has a  $6.1'$ FWHM. 
The FWHM values we assign to the native \PLANCK~maps are taken from 
\cite{planckbeam}, and are listed in Table \ref{table:offs}.

\subsection{Molecular Emission}
Because the FIRAS spectra consist of many narrow frequency bins, FDS99 were
able to discard the relatively small number of frequency intervals contaminated
by strong molecular line emission. Unfortunately, while the \PLANCK~data 
considered in this study are of high angular resolution, the broad 
\PLANCK~bandpasses do not allow us to adopt the same appraoch as FDS99 in 
dealing with line emission. Instead, we must subtract estimates of the 
molecular line contamination from each \PLANCK~band in order to best isolate 
the thermal continuum we wish to characterize. The most prominent molecular 
line emission in the \PLANCK~bands of interest arises from the three lowest CO 
rotational transitions: J=1$\rightarrow$0 at 115 GHz, J=2$\rightarrow$1 at 230 
GHz and J=3$\rightarrow$2 at 345 GHz, respectively affecting the \PLANCK~100, 
217 and 353 GHz bands. The J=1$\rightarrow$0 line also imparts a signal upon 
\PLANCK~143 GHz, but at a negligible level, $\sim$1000$\times$ fainter relative
to the dust continuum than J=1$\rightarrow$0 at 100 GHz.

To correct for molecular emission, we employ the \PLANCK~Type 3 CO data 
product, which boasts the highest S/N among the available full-sky CO maps 
based on the \PLANCK~HFI and Low Frequency Instrument (LFI) data 
\citep{planckco}. The native angular resolution of the Type 3 CO map is 5.5$'$.
We therefore begin by smoothing the raw Type 3 CO map to match the PSF of the 
smoothed \PLANCK~intensity maps we wish to correct for molecular emission.

We must apply the appropriate unit conversions to the Type 3 CO map before 
subtracting it from the \PLANCK~intensity maps, which have native units of 
$K_{CMB}$ at the frequencies of interest. The Type 3 CO map is provided in 
units of K$_{RJ}$ km/s of J=1$\rightarrow$0 emission. To convert this quantity 
to $K_{CMB}$, we assume that all of the CO emission arises from the $^{12}$CO 
isotope, and derive the \PLANCK-observed CO intensity in units of $K_{CMB}$ as
follows:

\begin{equation}
I_{CO, \nu_i, N, N-1} = I_{3}F_{12CO, \nu_i, N, N-1} R_{N, N-1}
\end{equation}

Where $I_{CO, \nu_i, N, N-1}$ is the intensity in $K_{CMB}$ in \PLANCK~band 
$\nu_i$ due to the CO transition from J=$N$ to J=$(N$$-$1). $I_3$ represents 
the appropriately smoothed Type 3 CO amplitude in  K$_{RJ}$ km/s of 
J=1$\rightarrow$0 emission. The $F_{12CO, \nu_i, N, N-1}$ are conversion 
factors between K$_{RJ}$ km/s and $K_{CMB}$ for particular band/transition 
pairs. The relevant values, calculated with the \textit{Unit Conversion and 
Colour Correction} software utilities (\verb|v1.2|), are:

\noindent
$F_{12CO, 100, 1, 0}$=1.478$\times$10$^{-5}$$K_{CMB}/(K_{RJ} \ km/s)$,

\noindent
$F_{12CO, 217, 2, 1}$=4.585$\times$$10^{-5}$$K_{CMB}/(K_{RJ} \ km/s)$, and 

\noindent
$F_{12CO, 353, 3, 2}$=1.751$\times$$10^{-4}$$K_{CMB}/(K_{RJ} \ km/s)$.

\noindent
$R_{N, N-1}$ represents the line ratio of the transition from J=$N$ to 
J=$(N$$-$1) relative to the J=1$\rightarrow$0. Thus, $R_{1,0}$=1, and we 
further adopt $R_{2,1}$=0.595 and $R_{3,2}$=0.297 based on \cite{planckco}. 
These line ratios are assumed to be constant over the entire sky. 

Formally, then, the CO contamination in band $\nu_i$ is given by:

% might consider quoting rounder values in text, most precise values are:
% 100 GHz, J=1->0 : 1.478e-5, 217 GHz, J=2->1 : 4.585e-5, 
% 353 GHz, J=3->2 : 1.751e-4

\begin{equation}
I_{CO, \nu_i} = \sum\limits_{N} I_{CO, \nu_i, N, N-1}
\end{equation}

It happens that, for each of the \PLANCK~bands in which CO emission is
non-negligible (100, 217 and 353 GHz), only a single $N$ contributes ($N$=1, 
$N$=2 and $N$=3, respectively). 

Unfortunately, the Type 3 CO map at $6.1'$ FWHM is rather noisy, and the vast
majority of the sky has completely negligible CO emission. Thus, in order to 
avoid adding unnecessary noise outside of molecular cloud complexes and at high
latitudes, we have zeroed out low-signal regions of the Type 3 CO map. We 
identify  low-signal regions as those with $\mathcal{I}_3$$<$1 K$_{RJ}$ km/s, 
where $\mathcal{I}_3$ is the Type 3 CO map smoothed to 0.25$^{\circ}$ FWHM. As 
a result of this cut, 90\% of the sky remains unaffected by our CO correction, 
particularly the vast majority of the high Galactic latitude sky.

% maybe a final sentence or two warning that, to whatever extent
% CO line ratios vary over the sky and/or other molecules make a substantial
% contribution to the observed Planck intensities, our molecular emission
% correction will be insufficient

% for CO line list see ~/wise/pro/planck_co_lines.pro

\subsection{Zero Level}
\label{sec:zp}

% for quantitative assessment of the zero-point difference due to
% flattening of 857 GHz vs. HI, see 
% ~/Desktop/dust/planck_hi/piecewise.pdf (on my Mac)

% -> at low HI, slope decreases by a factor of 1.88
%    the resulting difference in 857 GHz zero points is 0.373 MJy/sr

Although we wish to isolate and model thermal emission from Galactic dust, the
\PLANCK~maps contain additional components on large angular scales. At each 
frequency, there can exist an overall, constant offset that must be subtracted 
to set the zero level of Galactic dust by removing the mean cosmic IR 
background (CIB), as well as any instrumental offset. Additionally, faint 
residuals of the Solar dipole remain at low frequencies. We will address these 
issues by separately solving two sub-problems: first, fixing the absolute zero 
level of \PLANCK~857 GHz relative to external data, and second fitting 
100-545 GHz offsets and low order corrections by correlating these 
\PLANCK~bands against \PLANCK~857 GHz.

\subsubsection{Absolute Zero Level}
\label{sec:zp_abs}
In \cite{planckdust}, the absolute zero level of thermal dust emission was set 
by requiring that \PLANCK~infrared emission tends to zero when H\,\textsc{i} is
zero, assuming a linear correlation between these two measurements at low 
column density. However, this approach is somewhat unsatisfying in that there 
appear to be different slopes of \PLANCK~857 GHz versus H\,\textsc{i} for 
different ranges of H\,\textsc{i} intensity. In particular, \PLANCK~857 GHz 
appears to ``flatten out'' at very low H\,\textsc{i}. More quantitavely, we 
have found using the LAB H\,\textsc{i} data \citep{lab} for 
$-72$$<$$v_{LSR}$$<$$+25$ km/s that the best-fit slope for H\,\textsc{i}$<$70 K
km/s is a factor of $\sim$1.9 lower than the best fit slope for 110 K km/s 
$<$H\,\textsc{i}$<$200 K km/s, and as a result the implied zero-point offsets 
for \PLANCK~857 GHz differ by $\sim$0.37 MJy/sr.

% maybe state approximately what 0.37 MJy/sr at 857 GHz equates to in EBV
% --> did a quick calculation and it's only ~0.0118 mags ... thought it
%     would be more ??

\begin{figure}
\begin{center}
\epsfig{file=zp_857.eps, width=3.4in}
\caption{\label{fig:fdsref} Scatterplot of $\mathcal{F}_{857}$ versus 
$\mathcal{I}_{857}$, illustrating our absolute zero point determination by
comparison to the FDS99 prediction for \PLANCK~857 GHz.}
\end{center}
\end{figure}

\begin{figure}
\begin{center}
\epsfig{file=dipole_all_100.eps, width=3.4in}
\caption{\label{fig:dip}  Scatter plots of \PLANCK~100, 143, 217, 353, and 545 
GHz versus \PLANCK~857 GHz. Left: before applying zero-level offsets and 
additional low-order corrections. Right: after correcting for the best-fit 
offset and residual Solar dipole (143-545 GHz). 100 GHz is shown after applying
the spherical harmonic correction of Equation \ref{equ:harm}. The dashed red 
line shows the best-fit linear relationship in all cases.}
\end{center}
\end{figure}

\begin{figure*}
\begin{center}
\epsfig{file=harm_100.eps, width=6.5in}
\caption{\label{fig:harm} Summary of low-order corrections at 100 GHz. Left: 
prior to our low-order corrections, a $\sim$17$\mu$K zero-level offset is 
present and strong low-order problems reduce the linearity of the 100 GHz 
scatter versus 857 GHz. Center: scatter plot versus 857 GHz after applying the 
best-fit offset and residual Solar dipole corrections derived with Equation 
\ref{equ:dip} to \PLANCK~100 GHz. The correlation is strengthened, but remains
far less tight than for 143-545 GHz (see Figure \ref{fig:dip}, top four rows, 
right column). Right: after applying the spherical harmonic correction of
Equation \ref{equ:harm} to \PLANCK~100 GHz, the correlation versus 857 GHz is 
far more tightly linear than following the dipole correction.}
\end{center}
\end{figure*}

Because of these ambiguities in the relationship between 857 GHz and 
H\,\textsc{i} emission, we decided to instead constrain the 857 GHz zero level 
by comparison to the FDS99 857 GHz prediction. This renders the \PLANCK~857 GHz
absolute zero level tied indirectly to H\,\textsc{i} through the FDS99 
100$\mu$m and 240$\mu$m zero points. 

We perform a linear fit to the FDS99-predicted 857 GHz values as 
a function of \PLANCK~857 GHz. For this purpose, we employ a version of the 
\PLANCK~857 GHz map with zodiacal light and point sources removed and smoothed 
to 1$^{\circ}$ FWHM, which we will refer to as $\mathcal{I}_{857}$. We 
consider $\mathcal{I}_{857}$ to be the independent variable, as it has much 
higher S/N than the FDS99 prediction, henceforward referred to as 
$\mathcal{F}_{857}$. Note that $\mathcal{F}_{857}$ is not simply the FDS99
model evaluated at 857 GHz, but also incorporates the color correction factor 
of $\S$\ref{sec:bpcorr}, using the FDS99 temperature map to determine the dust 
spectrum shape. We then rebin to $N_{side}$=64 and restrict to pixels with 
$\mathcal{I}_{857}$$<$2.15 MJy/sr. Since \PLANCK~857 GHz smoothed to degree 
resolution has very high S/N, we can safely perform such a cut on 
$\mathcal{I}_{857}$. Figure \ref{fig:fdsref} shows a scatterplot of 
$\mathcal{I}_{857}$ versus $\mathcal{F}_{857}$, with a moving median and linear
fit overplotted. The linear fit was performed with uniform weights and 
iterative outlier rejection. The best-fit linear model is given by 
$\mathcal{F}_{857}$=0.991$\mathcal{I}_{857}$$-$0.018. It is encouraging that 
the slope is quite close to unity, and the offset near zero.

% i evaluated the `formal statistical error' using the covar keyword in
% hogg_iter_linfit

The formal statistical error on the best-fit 857 GHz offset is quite small, 
$\sim$0.002 MJy/sr. The systematics likely to dominate the actual uncertainty 
on our FDS-based zero level are imperfections in the \PLANCK/\verb|i100| 
zodiacal light models and the FDS99 temperature map. To quantify these 
systematic uncertainties, we split the sky into four quadrants, with boundaries
at $b$=0$^{\circ}$ and $l$=0$^{\circ}$, $l$=180$^{\circ}$. We again restricted 
to $\mathcal{I}_{857}$$<$2.15 MJy/sr, and repeated the regression in each 
quadrant. The rms of the per-quadrant slopes was found to be 0.0188, while the 
rms of the per-quadrant offsets was 0.0586 MJy/sr.

% in previous sentence, think it might be better to only quote rms, and
% omit the ranges of values for each parameter ?

% mention that all of this assumes that residual CMB dipole
% and other low order-corrections are negligible at 857 GHz

% what HI intensity does the 2.15 MJy/sr 857 GHz correspond to ?
%  ---> pretty high actually

\begin{deluxetable*}{rrrrrr} 
\tabletypesize{\scriptsize}
\tablecolumns{5} 
\tablewidth{0pc} 
\tablecaption{\label{table:offs} Planck/IRAS Map Properties \& Pre-processing} 
\tablehead{
\colhead{$\nu$ (GHz)} &
\colhead{Offset ($K_{CMB}$)} & 
\colhead{Dipole ($K_{CMB}$)} & 
\colhead{$c_{\nu}$} &
\colhead{FWHM ($'$)}
}
\startdata
100 & 1.69$\times$10$^{-5}$$\pm$3.61$\times$10$^{-7}$ & $-$1.08$\times$10$^{-5}$ &  0.0054 & 9.66 \\
143 & 3.58$\times$10$^{-5}$$\pm$7.58$\times$10$^{-7}$ & $-$1.08$\times$10$^{-5}$ & 0.0054 &  7.27 \\
217 & 7.79$\times$10$^{-5}$$\pm$2.60$\times$10$^{-6}$ & $-$1.40$\times$10$^{-5}$ & 0.0054 &  5.01 \\
353 & 2.76$\times$10$^{-4}$$\pm$1.95$\times$10$^{-5}$ & $-$3.08$\times$10$^{-5}$ & 0.012 & 4.86 \\
    & Offset (MJy/sr) & Dipole (MJy/sr) & & \\ \cline{2-3} \\ [-2ex]
545 & 7.27$\times$10$^{-2}$$\pm$1.99$\times$10$^{-2}$ & 1.63$\times$10$^{-2}$ & 0.10 & 4.84 \\
857 & 1.82$\times$10$^{-2}$$\pm$6.02$\times$10$^{-2}$ &  - & 0.10 & 4.63 \\
3000 & 0.0$\pm$4.3$\times$10$^{-2}$ & - & 0.10 & 6.1
\enddata
\end{deluxetable*}

\subsubsection{Relative Zero Level}
\label{sec:relzero}

In the course of this study we use not only \PLANCK~857 GHz, but also all of
the remaining \PLANCK~HFI bands, as well as \verb|i100|. To derive the 
zero level offsets that must be applied to each of the five lowest-frequency 
\PLANCK~bands, we perform a regression versus the \PLANCK~857 GHz map corrected
for the best-fit absolute zero level offset from $\S$\ref{sec:zp_abs}. We 
assume no offset need be applied to \verb|i100|, which already has its zero 
level tied to H\,\textsc{i}.

The need for additional low-order corrections beyond simple scalar offsets 
became evident upon inspecting the HFI maps at 100-545 GHz. In particular, we 
noticed the presence of a low-level dipole pattern, with an orientation 
consistent with that of the Solar dipole. Our strategy will be to 
simultaneously fit both this residual dipole and the zero-level offset 
amplitude for each band. To most precisely recover these amplitudes, it is 
necessary to have the highest available S/N in the independent variable of our 
regression. For this reason we have used \PLANCK~857 GHz as a reference for the
100-545 GHz bands, as opposed to the FDS99 predictions or H\,\textsc{i} data. 
In doing so, we assume \PLANCK~857 GHz contains no appreciable Solar dipole 
residual or other low-order problems. % grammar?

We perform one regression per HFI band (other than 857 GHz) to simultaneously 
fit for the zero level offset, the slope relative to 857 GHz, and the residual 
dipole amplitude. For each 100-545 GHz HFI band, we restrict to regions of low 
column density (H\,\textsc{i} $<$ 200 K\,km\,s$^{-1}$ for 
$-72$$<$$v_{LSR}$$<$$+25$ km\,s$^{-1}$) and fit the following model:

\begin{equation} \label{equ:dip}
\mathcal{I}_{\nu_i, p} = m\mathcal{I}_{857, p} + b + d\mathcal{D}_{p}
\end{equation}

With $p$ denoting a single nside=64 HEALPix pixel \citep{healpix} in the maps 
$\mathcal{I}_{857}$, $\mathcal{I}_{\nu_i}$, and $\mathcal{D}$. Here 
$\mathcal{I}_{857}$ is the \PLANCK~857 GHz map with zodiacal emission, compact 
sources, and the constant offset of $\S$\ref{sec:zp_abs} removed, smoothed to 
$1^{\circ}$ resolution. $\mathcal{I}_{\nu_i}$ is the corresponding $1^{\circ}$ 
resolution \PLANCK~HFI map with zodiacal emission, CMB anisotropies, and
compact sources removed. In the context of Equation \ref{equ:dip}, $\nu_i \in$ 
\{100, 143, 217, 353, 545\} GHz. Note that $\mathcal{I}_{\nu_i}$ is always in
the native units of the relevant \PLANCK~band. $\mathcal{D}$ is a scaling of 
the Solar dipole pattern oriented toward 
$(l, b) = (263.99^{\circ}, 48.26^{\circ})$, with unit amplitude. Because 
$\sim$18,000 pixels satisfy the low H\,\textsc{i} cut, we have an 
overconstrained linear model with three parameters: $m$, $d$, and $b$. $m$ 
represents the best-fit slope of \PLANCK~band $\nu_i$ versus \PLANCK~857 GHz 
assuming they are linearly related. $d$ is the residual Solar dipole amplitude,
and its best-fit value represents the scaling of the Solar dipole that makes 
the \PLANCK~band $\nu_i$ versus 857 GHz correlation most tightly linear. $b$ 
represents the constant offset that must be subtracted from the band $\nu_i$ 
map to make its zero level consistent with that of the 857 GHz map. For each 
band $\nu_i$, we obtain estimates of $m$, $d$, and $b$ by performing a linear 
least squares fit with uniform weights and iterative outlier rejection. Figure 
\ref{fig:dip} shows scatterplots of the band $\nu_i$ versus 857 GHz correlation
 before and after correcting for the best-fit offset and residual dipole, for 
each $\nu_i \in$ \{143, 217, 353, 545\} GHz. Not only are the tightened 
correlations striking in these scatterplots, but the residual dipole 
subtractions appear very successful in the two-dimensional band $\nu_i$ maps 
themselves. Before performing thermal dust fits, we therefore subtract the 
best-fit $b$ and $d\mathcal{D}$ from each band $\nu_i$ map. The best-fit 
offsets and residual dipole amplitudes are listed in Table \ref{table:offs},
along with other important per-band parameters, such as the fractional 
multiplicative calibration uncertainty $c_{\nu}$.

We found that a dipole correction alone could not sufficiently rectify the 
\PLANCK~100 GHz map (see Figure \ref{fig:harm}). Therefore, for 100 GHz, we 
performed a modified version of the Equation \ref{equ:dip} fit, using the 
following model:

\begin{equation} \label{equ:harm}
\mathcal{I}_{100, p} = m\mathcal{I}_{857, p} + b + \sum_{l=1}^{4} \sum_{m=-l}^{l} a_{lm}Y_{l}^{m}(\theta_p, \phi_p)
\end{equation}

Where $Y_{l}^{m}$ are the real spherical harmonics, and the $a_{lm}$ are their 
corresponding real coefficients. The angle $\phi_p$ is taken simply to be 
$l_{gal, p}$ and $\theta_p$=$(90^{\circ}-b_{gal, p})$. Thus, we have replaced 
the Solar dipole term with a sum of 24 spherical harmonic templates, which, 
when multiplied by the best-fit $a_{lm}$ coefficients and subtracted from 
\PLANCK~100 GHz make the relation between 100 GHz and 857 GHz most tightly 
linear. Figure \ref{fig:harm} illustrates the improved correlation of 100 GHz 
vs. 857 GHz when including the spherical harmonic corrections relative to the 
dipole-only correction. The spherical harmonic decomposition of Equation 
\ref{equ:harm} did not improve the correlations at higher frequencies 
sufficiently to warrant replacing the dipole-only correction in those cases.

% mention how offset/dipole values compare to those derived in the Planck
% team paper ?

\section{Dust Emission Model}
\label{sec:modeling}

% maybe a sentence to transition from preprocessing -> modeling
% e.g. With pre-processing completed, we turned to modeling the 
% thermal dust emission.


% mention that the two different emissivity power laws allow 
% the model to simultaneously explain narrow SED peak, and flattening
% of SED toward the millimeter
At sufficiently high frequencies, Galactic thermal dust emission can be 
adequately modeled as a single MBB with power-law emissivity 
\citep[e.g. SFD;][]{planckdust}. However, it has long been recognized, 
particularly in view of the FIRAS spectra, that the dust SED flattens toward 
the millimeter in a manner which is not consistent with a simple extrapolation 
of the single-MBB model to low frequencies. In the diffuse ISM, \cite{reach95} 
found an improved fit to the FIRAS data using an empirically motivated 
superposition of two $\beta$=2 MBBs, one representing a `hot' grain population 
($T$$\approx$16$-$21 K), the other a `cold' grain population 
($T$$\approx$4$-$7 K). FDS99 built a more physically motivated two-MBB model, 
in which different grain emission/absorption properties account for the 
differing temperatures of each population, and these temperatures are coupled 
by assuming thermal equilibrium with the same interstellar radiation field 
(ISRF).

The primary FDS99 analysis considered the intrinsic grain properties, for 
example the emissivity power law indices, to be constant over the sky, and 
performed a correlation slope analysis to constrain these parameters with FIRAS
and DIRBE observations. FDS99 also constructed a DIRBE 240$\mu$m/100$\mu$m 
ratio to account for temperature variation at $\sim$1.3$^{\circ}$ resolution. 
In this work we seek to apply FDS99-like emission models to the \PLANCK~data 
set, which offers a dramatic enhancement in angular resolution relative to the 
FIRAS spectra. The \PLANCK~data thereby allow us to derive an improved 
temperature correction at near-\IRAS~resolution, re-evaluate the best-fit 
global dust properties, and fit additional parameters of the two-component 
model as a function of position on the sky. % in this last sentence
% reference the relevant sections for each sub-goal mentioned

% kind of feel like i'm re-writing the introduction here ??

% could add a last sentence 'ultimately this allows us to make high-resolution
% foreground and extinction predictions' and/or even find a way to mention
% other predictions, such as polarization/R_v

% in the above paragraph, should I mention the Planck collaboration result
% of beta_mm vs. beta_fir being different, with (beta_fir-beta_mm) > 0

% should i also try to justify why it's worth doing this fit even
% if it's possible to just claim different beta's at different frequencies?

The shape of the two-component model spectrum we will consider is given by:

\begin{equation}
M_{\nu} \propto f_{1}q_{1}\Big(\frac{\nu}{\nu_{0}}\Big)^{\beta_1}B_{\nu}(T_1) + (1-f_{1})q_{2}\Big(\frac{\nu}{\nu_0}\Big)^{\beta_2}B_{\nu}(T_2)
\end{equation}

% With $f_{1}$ = 0.0363, $\beta_1$ = 1.67, $\beta_2$ = 2.70, and $q_1/q_2$ = 
% 13.0.
 
% re: following paragraph --- in the code nu_0 is actually 2997.92458
% because this is the frequency with respect to which DIRBE color correction
% K_100 (and my color correction as well) is referenced

% not sure if `Planck function' is the correct jargon

Where $B_{\nu}$ is the Planck function, $T_1$ is the `cold' dust temperature, 
$T_2$ is the `hot' dust temperature, and $\beta_1$ and $\beta_2$ are the
emissivity power-law indices of the cold and hot components respectively. 
$q_1/q_2$ encodes the relative emission/absorption cross-sections of the two 
dust species, while $f_1$ dictates the relative amounts of each species 
present. Following the convention of FDS99, $\nu_0$ is chosen to be 3000 GHz. 

Mathematically, this two-MBB model requires specification of seven 
parameters for every line of sight: $T_1$, $T_2$, $\beta_1$, $\beta_2$, $f_1$, 
$q_1$/$q_2$ and the normalization of $M_{\nu}$. However, under the 
assumption that the temperature of each species is determined by maintaining 
thermal equilibrium with the same ISRF, $T_1$=$T_1$($T_2$, $\beta_1$, 
$\beta_2$, $q_1/q_2$) is fully determined by these other parameters. $T_1$ is 
always related to $T_2$ via a simple power law, although the prefactor and 
exponent depend on the parameters $q_1/q_2$, $\beta_1$ and $\beta_2$ (see FDS99
Equation 14).

% Importantly, the emissivity power law indices $\beta$ of the two species 
% differ.

These considerations still leave us with six potentially free parameters per 
line of sight. Unfortunately, fitting this many parameters per spatial pixel is
not feasible for our full-resolution $6.1$$'$ fits, as these are constrained by
only five broadband intensity measurements. Hence, as in FDS99, we deem certain
 parameters to be ``global'', i.e. spatially constant over the entire sky. In 
our full-resolution five-band fits, we designate $\beta_1$, $\beta_2$, $f_1$ 
and $q_1/q_2$ to be spatially constant. This same approach was employed by 
FDS99, and the globally best-fit values obtained by FDS99 for these parameters 
are listed in the first row of Table \ref{tab:global}. With these global 
parameters, FDS99 found $T_2$$\approx$$16.2$K, $T_1$$\approx$$9.4$K to be 
typical at high-latitude. In $\S$\ref{sec:global}, we discuss the best-fit 
global parameters favored by the \PLANCK~HFI data; these are listed in the 
second row of Table \ref{tab:global}.

Fixing the four parameters of Table \ref{tab:global}, our full-resolution, 
five-band fits have two remaining free parameters per line of sight: the hot 
dust temperature $T_2$ determines the SED shape and the normalization of 
$M_{\nu}$ determines the SED amplitude. In the lower-resolution fits of 
$\S$\ref{sec:lores} which include all HFI bands, we will allow $f_1$ to be a 
third free parameter, still holding $\beta_1$, $\beta_2$, and $q_1/q_2$ fixed.

%explain that we take the ``hot'' dust temperature to be T2 rather than T1

% maybe point to the SED fit summary figure as an example of what
% 

% For the parameters thus far specified, the two dust temperatures are further 
% related by:

%\begin{equation}
%T_1 = 0.352T_2^{1.18}
%\end{equation}

% more important than the details of the above equation is the notion
% that, for any set of f1, q1/q2, beta1, beta2, the relationship between
% T1 and T2 is a simple power law

%the below statements seem like they might best be moved to a generic
%opening of the next section, stating our philosophy of breaking sky
%into pixels, each with 5 SED measurements, and that with this small
%number of constraints per pixel, it makes sense to have a baseline
% fit at full resolution that has only two free parameters

%FDS99 found $T_2 \approx 16.2$K, $T_1 \approx 9.4$K to be typical at 
%high-latitude. In the present formulation, there are only two free parameters 
%to describe an observed SED: the hot dust temperature $T_2$ determines the SED
%shape and the normalization of $M_{\nu}$ determines the SED amplitude. In 
%principle, we could allow additional parameters, such as $f_1$ and the ratio 
%$q_1/q_2$, to vary. However the number of free parameters in our baseline fit 
%at the full $6.1'$ resolution is limited by the fact that, for each sky 
%location, the observed SED contains only five independent intensity 
%measurements (\verb|i100| along with the four highest-frequency 
%\PLANCK~bands).

To calculate the optical depth in the context of this model, we assume
optically thin conditions, meaning that $\tau_{\nu}$ = $M_{\nu}/S_{\nu}$, where
$M_{\nu}$ is the appropriately scaled two-component intensity and the source
function is given by:

\begin{equation}
\label{eqn:source}
S_{\nu} = \frac{f_1q_1(\nu/\nu_0)^{\beta_1}B_{\nu}(T_1) + f_2q_2(\nu/\nu_0)^{\beta_2}B_{\nu}(T_2)}{f_1q_1(\nu/\nu_0)^{\beta_1}+f_2q_2(\nu/\nu_0)^{\beta_2}}
\end{equation}

\section{Predicting the Observed SED}
\label{sec:bpcorr}

The thermal dust emission model of $\S$\ref{sec:modeling} predicts the 
flux density per solid angle $M_{\nu}$ in e.g. MJy/sr for any single frequency 
$\nu$. In practice, however, we wish to constrain our model using measurements 
in the broad DIRBE/\PLANCK~bandpasses, each with $\Delta\nu/\nu\sim0.3$. 
Both the \PLANCK~and DIRBE data products quote flux density per solid 
angle in MJy/sr under the `IRAS convention'. More precisely, each value 
reported in the \PLANCK~maps gives the amplitude of a power-law spectrum 
with $\alpha$=$-1$, evaluated at the nominal band center frequency, such that 
this spectrum integrated against the transmission reproduces the 
bolometer-measured power. Because our model spectra do not conform to the 
$\alpha$=$-1$ convention, we have computed color correction factors to account 
for the MBB($T$, $\beta$) spectral shape and the transmission as a function of 
frequency:

% is delta(nu)/nu conventionally (nu_off-nu_on)/nu_c or 1/2 of that ?

\begin{equation} \label{equ:bpcorr}
b_{\nu_i}(T, \beta) = \frac{\int \nu^{\beta}B_{\nu}(T)\mathcal{T}_{\nu_i}(\nu) d\nu \bigg[\int (\nu_{i,c}/\nu)\mathcal{T}_{\nu_i}(\nu) d\nu\bigg]^{-1}}{\nu_{i,c}^{\beta}B_{\nu_{i,c}}(T)}
\end{equation}

% integral limits ??? check planck paper to see what they did
% -> planck spectral response paper just leaves the integrals indefinite

Here $\nu_{i,c}$ is the nominal band center frequency of band $\nu_i$,  with 
$\nu_{i,c} \in$ \{100, 143, 217, 353, 545, 857, 1249.1352, 2141.3747, 
2997.92458\} GHz. $\mathcal{T}_{\nu_i}(\nu)$ represents the relative 
transmission as a function of frequency for band $\nu_i$. For the HFI maps, 
$\mathcal{T}_{\nu_i}(\nu)$ is given by the \PLANCK~transmission curves provided
in the file \verb|HFI_RIMO_R1.10.fits| \citep{planckresponse}. For 3000 GHz and
DIRBE 140$\mu$m, 240$\mu$m, we have adopted the corresponding DIRBE 
transmission curves. Again, since DIRBE-quoted fluxes also follow the 
$\alpha$=$-1$ convention, Equation \ref{equ:bpcorr} remains applicable.

% what is the reference for the dirbe

%The color correction factors
%of Equation \ref{equ:bpcorr} are applied in the sense that 
%$b_{\nu_i}(T ,\beta)\times M_{\nu_{i,c}}$ is a prediction suitable for 
%direct comparison to the \PLANCK~data products.

The two-component model prediction in band $\nu_i$ under the IRAS convention, 
termed $\tilde{I}_{\nu_i}$, is then constructed as a linear combination of 
color-corrected MBB terms:

\begin{equation} \label{equ:iras}
\tilde{I}_{\nu_i} \propto \sum_{k=1}^{2} b_{\nu_i}(T_k, \beta_k) f_k q_k (\nu_{i,c}/\nu_0)^{\beta_k} B_{\nu_{i,c}}(T_k)
\end{equation}

Note that, as in FDS99, $f_2$=(1$-$$f_1$) by convention. Our color correction 
approach allows us to predict $\tilde{I}_{\nu_i}$ by computing monochromatic 
flux densities at the central frequency $\nu_{i,c}$ and then multiplying by 
factors $b_{\nu_i}(T, \beta)$ which are interpolated off of a set of 
precomputed, one-dimensional lookup tables each listing $b_{\nu_i}(T, \beta)$ 
as a function of $T$. We avoided the need to interpolate in both $\beta$ and 
$T$ by computing only a small set of one dimensional correction factors for the
particular set of $\beta$ values of interest (e.g. $\beta$=1.67, 2.7, 
1.63, 2.82 ..., see Table \ref{tab:global}).

% need some sort of clarification about the fact that, in the parts of the 
% analysis where beta floats, i actually do interpolate color corrections 
% off of a 2d grid in (T, beta)

This color correction approach makes the MCMC sampling described in 
$\S$\ref{sec:mcmc} much more computationally efficient by circumventing the 
need to perform the integral in the numerator of Equation \ref{equ:bpcorr} 
on-the-fly for each proposed dust temperature. We have chosen to compute the 
color corrections on a per-MBB basis because this approach is very 
versatile; all possible two-component (and single-MBB) models are linear 
combinations of MBBs, so we can apply all of our color correction machinery 
even when we allow parameters other than temperature (e.g. $f_1$) to vary and 
thereby modify the dust spectrum shape.

% We note that it is not clear which bandpass should properly be ascribed to 
% the \IRAS/DIRBE composite 100$\mu$m maps under consideration.

With these color corrections and the  formalism established in 
$\S$\ref{sec:modeling} in hand, we can mathematically state the model we will 
use e.g. during MCMC sampling to predict the observed SED, which we will 
denote by \{$\tilde{I}_{\nu_i}$\}. The predicted observations are given by:

\begin{equation}
\label{eqn:inten}
\tilde{I}_{\nu_i} = \frac{\sum\limits_{k=1}^{2} b_{\nu_i}(T_k, \beta_k) f_k q_k (\nu_{i,c}/\nu_0)^{\beta_k} B_{\nu_{i,c}}(T_k) u_{\nu_i}^{-1}(T_k, \beta_k)}{\sum\limits_{k=1}^{2} b_{545}(T_k, \beta_k) f_k q_k (545 \textrm{GHz}/\nu_0)^{\beta_k} B_{545}(T_k)}\tilde{I}_{545}
\end{equation}

This equation is quite similar to Equation \ref{equ:iras}, but with two 
important differences. First, the normalization of $\tilde{I}_{\nu_i}$ is now 
specified by $\tilde{I}_{545}$, which represents the flux density per solid
angle measured in the \PLANCK~545 GHz band. The denominator serves to ensure 
that, for the case of $\nu_i$=545 GHz, $\tilde{I}_{545}$ is self-consistent. 
Second, each term in the numerator is multiplied by a conversion factor 
$u_{\nu_i}^{-1}(T_k, \beta_k)$. This factor is necessary because some of the 
\PLANCK~maps of interest have units of $K_{CMB}$ (100-353 GHz), while the
remaining maps (545-3000 GHz) have units of MJy/sr. We have adopted the 
strategy of predicting each band in its native units, whether MJy/sr or 
$K_{CMB}$. For this reason, we always evaluate $B_{\nu_{i,c}}$ in Equation 
\ref{eqn:inten} in MJy/sr and let $u_{\nu_i}$=1 (dimensionless) for 
$\nu_i$$\ge$545 GHz. For $\nu_i$$\le$353 GHz, $u_{\nu_i}(T_k, \beta_k)$ 
represents the conversion factor from $K_{CMB}$ to MJy/sr, given by 
\cite{planckresponse} Equation 32.

%is the old sentence below worth reinstating ?

%To reiterate, for 
%$\nu_i$$\le$353 GHz, $u_{\nu_i}(T_k, \beta_k)$ has units of 
%(MJy/sr)/$K_{CMB}$,and therefore $\tilde{I}_{\nu_{i}}$ will have different 
%units depending on the band $\nu_i$, though $\tilde{I}_{\nu_{i}}$ always has 
%the same units as the native band $\nu_i$ map.

% check that RIMO file i've used is actually the most up-to-date version


% extremely abrupt transition here...
\section{Global Model Parameters}
\label{sec:global}
%%%%

\begin{deluxetable*}{llrrrrrrrrrr} 
\tabletypesize{\scriptsize}
\tablecolumns{11} 
\tablewidth{0pc} 
\tablecaption{\label{tab:global} Global Model Parameters} 
\tablehead{
\colhead{Number} &
\colhead{Model} &
\colhead{$f_1$} &
\colhead{$q_1/q_2$} & 
\colhead{$\beta_1$} & 
\colhead{$\beta_2$} &
\colhead{$T_2$} &
\colhead{$T_1$} &
\colhead{$n$} &
\colhead{D.O.F.} &
\colhead{$\chi^2$} &
\colhead{$\chi^2_{\nu}$}
}
\startdata
 1 & FDS99 best-fit  & 0.0363 & 13.0  & 1.67 & 2.70 & 15.72 &  9.15 & 1.018 & 7 & 23.9 & 3.41 \\
 2 & FDS99 general   & 0.0485 & 8.22  & 1.63 & 2.82 & 15.70 &  9.75 & 0.980 & 3 & 3.99 & 1.33 \\
 3 & single MBB      &  0.0   &  ...  &  ... & 1.59 & 19.63 &   ... & 0.999 & 6 & 33.9 & 5.65 \\ [-2ex]
\enddata
\end{deluxetable*}

% would be great to have uncertainties on each parameter as well (in the
% Planck case at least...) !!!!!!!!!!!!!!!

%%%%

\begin{figure*}
\begin{center}
\epsfig{file=dirbe_slopes.eps, width=6.5in}
\caption{\label{fig:dirbe_slopes} Linear fits of DIRBE 240$\mu$m (left), 
140$\mu$m (center), and 100$\mu$m (right) as a function of \PLANCK~857 GHz. The
red lines illustrate the DIRBE correlation slopes used in our global parameter 
analysis of $\S$\ref{sec:global}.}
\end{center}
\end{figure*}

While we ultimately aim to obtain \PLANCK-resolution maps of the spatially 
varying dust temperature and optical depth, we start by applying the 
machinery/formalism thus far developed to reassess the best-fit global 
two-component model parameters in light of the \PLANCK~HFI data.

FDS99 determined the best-fit values of the two-component model global 
parameters $\beta_1$, $\beta_2$, $q_1/q_2$ and $f_1$ via a correlation slope
analysis incorporating DIRBE and FIRAS data. Here we seek to estimate these 
same global parameters via an analogous correlation slope analysis in which we 
swap the \PLANCK~HFI maps for FIRAS at low frequencies, while still relying on 
DIRBE at higher frequencies. We also seek to determine via this correlation 
slope analysis whether or not the combination of \PLANCK+DIRBE data favors 
two-component models over single-MBB models in the same way that the 
FIRAS+DIRBE data did in the FDS99 analysis.

In the case of the two-component model, based on a spectrum of \PLANCK~and 
DIRBE correlations slopes, we wish to obtain estimates for six free parameters:
$\beta_1$, $\beta_2$, $q_1/q_2$, $f_1$, $T_2$ and the overall spectrum 
normalization $n$. The constraints we employ are the correlation slopes of each
of the \PLANCK~HFI bands, as well as DIRBE 100$\mu$m (3000 GHz), 140$\mu$m 
(2141 GHz) and 240$\mu$m (1250 GHz) relative to \PLANCK~857 GHz, i.e. 
$dI_{\nu_i}/dI_{857}$. We will refer to the slope for band $\nu_i$ relative to 
\PLANCK~857 GHz as $s_{i,857}$. The slopes for \PLANCK~100-545 GHz are taken 
to be those derived from the relative zero level fits of $\S$\ref{sec:relzero},
and are illustrated by the dashed red lines in the right-hand column plots of 
Figure \ref{fig:dip}. The 857 GHz slope is by definition unity.

At 1250, 2141 and 3000 GHz, we use the SFD98-reprocessed DIRBE maps. For each
DIRBE band, we determine $s_{i, 857}$ by performing a linear fit to DIRBE as a 
function of \PLANCK~857 GHz, after both have been zodiacal light subtracted and
smoothed to $1^{\circ}$ FWHM, also restricting to the low HI mask of 
$\S$\ref{sec:relzero} (see Figure \ref{fig:dirbe_slopes}).
% re-mention nside=64 or just bag it?
% restricted to same low HI mask as in Sect. 3.5.2 analysis
% sentence pointing to figure showing the regressions

Including 857 GHz, we thus have nine correlation slope constraints for six 
free parameters. Including DIRBE 140$\mu$m and 240$\mu$m is critical in making 
this problem sufficiently overconstrained, and also in providing information 
near the peak of the dust SED at $\sim$140$\mu$m, which is particularly 
sensitive to the presence of a single versus multiple MBB components. 

We assume an uncertainty of 2\% on each of the $s_{i, 857}$ and minimize the 
chi-squared given by:

\begin{equation} \label{equ:chi2corr}
\chi^2 = \sum_{i=0}^{8}\frac{\big[s_{i, 857}-n\frac{\tilde{I}_{\nu_i}(\beta_1, \beta_2, f_1, q_1/q_2, T_2)}{\tilde{I}_{857}(\beta_1, \beta_2, f_1, q_1/q_2, T_2)}\big]^2}{\sigma_{i}^2}
\end{equation}

Where $\nu_i$$\in$\{100, 143, 217, 353, 545, 857, 1250, 2141, 3000\} GHz. Note 
that this formula encompasses the general two-component case. In the single-MBB
case, we take $f_1$=0 and hence $q_1/q_2$, $\beta_1$ and $T_1$ are immaterial, 
but Equation \ref{equ:chi2corr} still applies. Note also that no `priors' have
been included to preferentially drag our results towards agreement with those 
of FDS99.

The results of our chi-squared minimization are listed in Table 
\ref{tab:global}. First (model 1), we fix $\beta_1$, $\beta_2$ ,$q_1/q_2$ and
$f_1$ to the best-fit values from the FDS99 analysis based on DIRBE+FIRAS. We
then allow $n$ and $T_2$ to vary so as to best match our 
DIRBE+\PLANCK~spectrum. This results in a reduced chi-squared of 
$\chi^2_{\nu}$=3.41. Reassuringly, $n$ is quite close to unity. It should be 
noted though that our best-fit $T_2$ is $\sim$0.5 K lower than that found by 
FDS99 for the same values of $\beta_1$, $\beta_2$ ,$q_1/q_2$ and $f_1$.

Next (model 2), we consider the fully-general two-component model, allowing all
 six model parameters to vary. In this case, the reduced chi-squared of the 
best fit is $\chi^2_{\nu}=$1.33, signifying that our introduction of four 
additional free parameters is justified. The best-fit $\beta_1$ and $\beta_2$ 
are both consistent with the corresponding FDS99 values to within 5\%. 
$q_1/q_2$=8.22 represents a $\sim$40\% lower value than found by FDS99, while
$f_1$=0.0458 represents a $\sim$25\% increase relative to FDS99. Again, our
best-fit high-latitude $T_2$ is $\sim$0.5 K lower than the typical value of 
$\langle T_2 \rangle$=16.2 K from FDS99.

Lastly, we calculate the optimal single-MBB fit to the \PLANCK+DIRBE 
correlation slope spectrum. The best-fit single MBB has $\beta$=1.59, 
$T$=19.63, and $\chi^2_{\nu}$=5.65, indicating a significantly worse fit to the
data than our best-fit two-component model (model 2). Thus, our \PLANCK+DIRBE 
correlation slope analysis has confirmed the main conclusion of FDS99 and 
others e.g. \cite{reach95}, that the FIR/submm dust SED prefers two MBBs to 
just one, but, for the first time, independent of FIRAS. Still, it is apparent 
that the improvement in $\chi^2_{\nu}$ for single-MBB versus double MBB models 
found here is substantially less dramatic ($\Delta\chi^2_{\nu}$=4.32) than that
found in FDS99 ($\Delta\chi^2_{\nu}$=29.2). This is likely attributable to the 
exquisite narrow-band frequency coverage of FIRAS, especially near the dust SED
peak, which makes FIRAS a better suited data set than \PLANCK~for a detailed 
analysis of the globally best-fit dust SED model. In $\S$\ref{sec:hier}, we
confirm the basic conclusions of this section via an MCMC-based approach in
which the dust temperature is allowed to vary spatially.

% also might be worth mentioning that the MCMC analysis involves constraints
% where the zero level matters, whereas the correlation slope analysis here
% renders the zero levels irrelevant

% maybe a table listing the correlation slopes and their uncertainties
% would be useful here

\section{MCMC Fitting Procedure}
\label{sec:fitting}

The following subsections detail our procedure for constraining the 
two-component dust emission model parameters which are permitted to vary
spatially. We perform two types of fits: (1) full-resolution 
$6.1'$ fits, in which just two parameters, the SED normalization and the hot 
dust temperature $T_2$, vary spatially and (2) lower-resolution fits in which 
$f_1$ is also allowed to vary from one line of sight to another.

\begin{figure}
\begin{center}
\epsfig{file=sed.eps, width=3.3in}
\caption{\label{fig:sed} Planck SED for a single nside=2048 pixel in the 
Polaris region. Note that the two lowest-frequency data points (\PLANCK~100, 
143 GHz) were not used in our fit, while the three lowest-frequency data
points were not used in the \cite{planckdust} fit.}
\end{center}
\end{figure}

% add error bars to this example SED plot

\begin{figure*}
\begin{center}
\epsfig{file=posteriors.eps, width=7.0in}
\caption{Gridded posterior PDFs for three nside=2048 pixels. Red crosses mark 
the best-fit parameters based on our Markov chain sampling of the posterior. 
The posterior distributions are in general extremely well-behaved, showing
no multimodality or other pathological qualities. Our MCMC parameter 
estimates coincide well with the peaks in the gridded posteriors. The 
colorscale is linear in $log(P)$, with black representing the maximum of 
$log(P)$ and white representing $max(log(P))-5$. Left: Low S/N pixel at 
high-latitude in Galactic north. Center: High S/N pixel in the Polaris region. 
Right: Low S/N pixel at high-latitude in the Galactic south.}
\end{center}
\end{figure*}

% overplot ellipses showing 1 sigma errors in posterior grayscales

\subsection{Pixelization}
\label{sec:pix}
For the purpose of fitting, we break the sky into $\sim$50 million pixels of 
angular size $\sim$1.72$'$, defined by the HEALPix pixelization in Galactic 
coordinates, with $N_{side}$=2048. This pixelization is convenient because it 
is the same format in which the \PLANCK~HFI maps were released, and because it 
adequately samples the $6.1'$ FWHM maps under consideration in our 
full-resolution fits. Our procedure will fit the intensity measurements in each
spatial pixel independently.

\subsection{Sampling Parameters}
\label{sec:samp}
As discussed in $\S$\ref{sec:modeling}, our full-resolution fits
consider the ``global'' parameters $f_1$, $q_1/q_2$, $\beta_1$, $\beta_2$ to be
 fixed, while the dust spectrum normalization and dust temperature vary for
each line-of-sight. In order to predict the dust SED for a given pixel, we are 
therefore left with two remaining degrees of freedom, and must choose an 
appropriate set of two parameters to sample and thereby constrain via MCMC. To 
determine the SED normalization in each pixel, we draw samples in 
$\tilde{I}_{545}$, the broadband intensity in the 545 GHz bandpass, as defined 
in Equation \ref{eqn:inten}. With the four aforementioned global parameters 
fixed, the dust spectrum shape is determined entirely by the two dust 
temperatures, which are coupled. To constrain the dust temperatures,
 we sample in $T_2$, the hot dust temperature. For each sample in $T_2$, we 
compute the corresponding value of $T_1$, thereby specifying ($\beta$, $T$) for
both MBB components and thus fully specifying the SED shape. In principle,
we could sample in either $T_1$ or $T_2$, but have chosen to sample in $T_2$ 
because emission from this component dominates in the relatively high frequency
bands which most strongly constrain the dust temperatures.

For the lower resolution fits described in $\S$\ref{sec:lores}, we sample
in three parameters: $\tilde{I}_{545}$, $T_2$, and $f_1$.

% sentence or two mentioning reduced resolution fits where we sample in
% three dimensions (f_1, T_2, tilde{I}_{545})

\subsection{Markov Chains}
\label{sec:mcmc}

In our full-resolution fits, we use a MCMC approach to constrain the 
parameters $\tilde{I}_{545}$ and $T_2$. For each pixel, we run a 
Metropolis-Hastings Markov chain sampling the posterior probability as a 
function of the two parameters $\tilde{I}_{545}$ and $T_2$. More specifically, 
we are sampling the posterior given by:

% check whether I545 normalization parameter is the monochromatic
% 545 GHz intensity, or if it is for the Planck bandpass
% i think it should eventually be monochromatic, since that will be
% more useful to people who want SED predictions

\begin{equation}
\label{eqn:post}
P(\tilde{I}_{545}, T_2|\{I\}) \propto \mathcal{L}(\{I\}|\tilde{I}_{545}, T_2)P(T_2)P(\tilde{I}_{545})
\end{equation}

% be careful about notational consistency with e.g. tildes, etc.. in this sect

Here \{$I$\} denotes the set of observed broadband intensities, $I_{217}$, 
$I_{353}$, etc. The likelihood function is given by:

\begin{equation} \label{equ:like}
\mathcal{L}(\{I\}|\tilde{I}_{545}, T_2) = \displaystyle\prod\limits_{i}\mathcal{N}(\tilde{I}_{\nu_{i}}|I_{\nu_{i}}, \sigma_{\nu_i})
\end{equation}

%be more careful about notation re: color corrections

Where the product runs over $\nu_i$ in $\{217,\ 353,\ 545,\ 857,\ 3000\}$ GHz
and $\tilde{I}_{\nu_i}$ is a function of $\tilde{I}_{545}$ and $T_2$ as 
specified by Equation \ref{eqn:inten}. For each pixel, the errors on each 
intensity measurement $\sigma_{\nu_i}$ are given by:

\begin{equation}
\sigma_{\nu_i}(p) = \sqrt{c^2_{\nu_i}I_{\nu_i}(p)^2 + c^2_{\nu_i}CMB(p)^2 + (\delta O_{\nu_i})^2 + n_{\nu_i}(p)^2}
\end{equation}

% is I_nu on RHS inclusive of offsets or is it the ``pure'' thermal dust
% intensity

This error budget is the same as that of \cite{planckdust} Equation 7. 
$c_{\nu_i}$ is the multiplicative calibration uncertainty for the intensity in 
each band. We have adopted the $c_{\nu_i}$ values listed in \cite{planckdust} 
Table 1. For the 217 GHz band, we have assigned $c_{217 GHz}$ = 0.54, from 
Table 11 of \cite{planckcalib}. $CMB$ represents the CMB intensity 
subtracted at the relevant pixel. $\delta O_{\nu_i}$ represents the uncertainty
in the offset used to tie \PLANCK~intensity to H\,\textsc{i}. We have adopted 
the  $\delta O_{\nu_i}$ values of \cite{planckdust} Table 1. $n_{\nu_i}$ 
represents the instrumental noise in the pixel of interest, and is taken to be 
the square root of the \verb|ii_cov| parameter that accompanies each 
\PLANCK~intensity map.

In order to obtain reasonable fitting results at high latitude where the data 
is noisy, we include the prior on $T_2$:

%what about the offset uncertainty for 217 GHz

\begin{equation} \label{equ:t2prior}
P(T_2) = \mathcal{N}(T_2|\bar{T}_2, \sigma_{\bar{T}_2})
\end{equation}

% maybe have a section justifying the T2 prior value of 16.2K, presumably using
% FIRAS as a cross-check?

With $\bar{T}_2$ = 16.2 K and $\sigma_{\bar{T}_2}$ = 1.4 K. In principle, there
can also be an informative prior on $\tilde{I}_{545}$. However, we have chosen 
to assume a uniform prior on the SED normalization and, as a matter of 
notation, will omit $P(\tilde{I}_{545})$ henceforward. In practice we always 
perform computations using logarithms of the relevant probabilities.
%mention something about the proposal distribution at some point?

\begin{figure}
\begin{center}
\epsfig{file=tcomparison.eps, width=3.3in}
\caption{\label{fig:comparison} Comparison of SFD temperature, two-component 
model $T_2$, and \cite{planckdust} temperature (labeled $T_{R1.20}$) for a
 $10.5^{\circ}\times8.3^{\circ}$  region centered about 
$(l,b) = (111.6^{\circ}, 20.3^{\circ})$. Note the differing colorscales. Both 
models incorporating \PLANCK~data clearly show a major improvement in angular 
resolution relative to SFD.}
\end{center}
\end{figure}

For each pixel, we initialize the Markov chain with parameters 
$\tilde{I}_{545}$ = $I_{545}$ and $T_2$ consistent with the FDS99 
DIRBE 100$\mu$m/240$\mu$m ratio map $\mathscr{R}$. We run 500 steps of burn-in 
and then 2000 steps during which we keep track of the proposed $T_{2, j}$ and 
$\tilde{I}_{545, j}$ values at the $j^{th}$ step since the end of burn-in. From
 these 2000 samples, we compute estimates of each parameter, 
$T_2$ = $\langle T_{2, j} \rangle$, $\tilde{I}_{545}$ = 
$\langle \tilde{I}_{545, j} \rangle$, and of each parameter's variance, 
$\sigma^2_{T_2}$ = $\langle T^2_{2, j} \rangle-\langle T_{2, j} \rangle ^2$ and
 $\sigma^2_{\tilde{I}_{545}}$=$\langle \tilde{I}^2_{545, j} 
\rangle-\langle \tilde{I}_{545, j} \rangle ^2$. We also compute the
 corresponding estimate of the monochromatic two-component intensity at 545 
GHz, $M_{545}$ = $\langle M_{545, j} \rangle$ = 
$\langle \tilde{I}_{545, j}/b_{545}(T_{2,j}) \rangle$  and its variance. 
$M_{545}$ is of more general utility in making foreground predictions because 
it provides the normalization for the two-component spectrum $M_{\nu}$, 
independent of the \PLANCK~545 GHz bandpass. Lastly, we compute the 545 GHz
optical depth as $\tau_{545}$ = $\langle \tau_{545, j} \rangle$ = 
$\langle M_{545, j}/S_{545, j} \rangle$ and its variance, with the source 
function $S_{545}$ calculated according to Equation \ref{eqn:source}.

%in the above need to mention/introduce the use of ``j'' to denothe the j'th
%step of the MH sampling **after** burn-in has finished

% actually we I think we want to keep track of $\tilde{I}_{545}$/b_{545}
% which is the monochromatic value, since that is most useful downstream
% when making foreground predictions that typically won't have anything
% to do with Planck bandpasses

\subsection{Low-resolution Fits}
\label{sec:lores}
%mention that the pixelization changes ---> to lower HEALpix nside
%mention that over some region of the sky with high signal and high
%S/N we try fitting t2, normalization and one of q1/q2, f_1
%using more bands (including 143, and maybe 100 GHz)

As explained in $\S$\ref{sec:modeling} and $\S$\ref{sec:samp}, our 
full-resolution fits fix the parameters $f_1$, $q_1/q_2$, $\beta_1$, $\beta_2$ 
globally so that only the dust temperatures and SED normalization vary on a 
per-pixel basis. The reason for fixing so many parameters is that, at $6.1'$
FWHM, only five SED measurements are available per pixel.

% need some sort of add on to last sentence or another sentence entirely
% to the effect of ``staying in overconstrained regime''

However, sacrificing angular resolution in order to incorporate 
the \PLANCK~100, 143 GHz bands can allow us to introduce an additional free 
parameter and run our Markov chains in three-dimensional parameter space.

We ran the Markov chains in three dimensions by allowing 
$f_1$ to be a third free parameter. In these chains, we sampled the posterior
given by:

\begin{equation}
\label{eqn:f1post}
P(\tilde{I}_{545}, T_2, f_1|\{I\}) \propto \mathcal{L}(\{I\}|\tilde{I}_{545}, T_2, f_1)P(T_2)P(f_1)
\end{equation}

We still employ the likelihood function of Equation \ref{equ:like}, the only
difference being that $f_1$ can vary from proposal to proposal within each 
individual pixel when predicting \{$\tilde{I}_{\nu_i}$\} with Equation 
\ref{eqn:inten}. The prior $P(T_2)$ from Equation \ref{equ:t2prior} remains 
unchanged. The prior on $f_1$ is:

\begin{equation} \label{equ:f1prior}
P(f_1) = \mathcal{N}(f_1|\bar{f}_1, \sigma_{\bar{f}_1})
\end{equation}

With $\bar{f}_1$=0.0363 and $\sigma_{\bar{f}_1}$=0.003. This is a fairly
stringent prior, but we must restrict the fit from wandering with too much
freedom, as we are attempting to constrain three parameters using an SED
with only seven intensity measurements. To boost S/N, we have performed
the fits using maps smoothed to $1^{\circ}$ resolution.

\begin{figure}
\begin{center}
\epsfig{file=f1_1deg.eps, width=3.3in}
\caption{\label{fig:f1} The results of our low-resolution fit with $f_1$
allowed to vary.}
\end{center}
\end{figure}

\begin{figure}
\begin{center}
\epsfig{file=f1_lambert.eps, width=3.3in}
\caption{\label{fig:f1lambert} The results of our low-resolution fit with $f_1$
allowed to vary, shown in Lambert projection to highlight the salient features 
near the north Galactic pole.}
\end{center}
\end{figure}

The resulting full-sky map of $f_1$ is shown in Figure \ref{fig:f1}. Some of 
the most striking features occur near the north Galactic pole, and to 
highlight these regions of unusually high $f_1$, we display the northern
Galactic hemisphere in Lambert projection in Figure \ref{fig:f1lambert}.
 
% add comment about high f_1 regions near NGP
% also add sentences referring reader to the two f_1 figures

\subsection{Global Parameters Revisited}
\label{sec:hier}

The posterior sampling framework thus far described also affords us an
opportunity to evaluate the goodness-of-fit for competing dust SED models, and 
thereby cross-check the conclusions of our correlation slope analysis in
$\S$\ref{sec:global}. The basic idea will be to continue sampling the posterior
of Equation \ref{eqn:post}, but at low resolution ($N_{side}$=64), including 
all HFI bands as well as DIRBE 100$\mu$m, 140$\mu$m and 240$\mu$m, and 
switching to a uniform prior on $T_2$. Under these circumstances, the 
chi-squared corresponding to the best-fit parameters for each pixel, termed 
$\chi^2_p$, is simply $-2 \times log(P_{max})$. We will refer to the
per-pixel chi-squared per degree of freedom as $\chi^2_{p, \nu}$.

\begin{figure}
\begin{center}
\epsfig{file=chi2_dirbe.eps, width=3.3in}
\caption{\label{fig:chi2_dirbe} Comparison of goodness-of-fit, 
$\chi^2_{\nu}$=$\langle \chi^2_{p, \nu} \rangle$ for various dust SED models. 
For single-MBB models with spatially constant $\beta$, we varied $\beta$ 
between 1 and 2 (x axis), achieving reduced chi-squared $\chi^2_{\nu}$ shown by
the black line, with $\beta$=1.57 providing the best fit. Horizontal lines 
indicate $\chi^2_{\nu}$ for other dust SED models considered, including the 
FDS99 best-fit model and our best-fit \PLANCK+DIRBE model from 
$\S$\ref{sec:global}. The minimum $\chi^2_{\nu}$ was achieved with
two-componenent `model 2' from Table \ref{tab:global} (horizontal magenta 
line).}
\end{center}
\end{figure}

Because we seek to compare the goodness-of-fit for various dust SED models in 
the diffuse ISM, we restrict to a set of $\sim$10,800 pixels ($\sim$22\% of the
sky), with $|b|>30^{\circ}$ and $|\beta|>10^{\circ}$. We also avoid the SMICA 
inpainting mask, pixels with appreciable CO contamination, and compact sources.
The goodness-of-fit `objective function' we employ to judge the quality of 
a particular dust SED model is $\langle \chi^2_{p, \nu} \rangle$, where the 
average is taken over the aforementioned set of $\sim$10,800 pixels. 
$\langle \chi_{p, \nu}^2 \rangle$ is also equivalent to the reduced 
chi-squared, $\chi^2_{\nu}$, when considering the total number of free 
parameters to be the number of pixels multiplied by the number of free 
parameters per pixel (and similarly for the total number of constraints), and
taking $\chi^2$=$\sum\chi^2_{p}$.

We calculate $\chi^2_{\nu}$ for various dust SED models, independently 
minimizing each $\chi^2_p$ by finding pixel $p$'s best-fitting dust 
temperature and normalization, then evaluating $\langle \chi_{p}^2 \rangle$. 
First, we consider single-MBB models with $\beta$ spatially constant (see the
black line in Figure \ref{fig:chi2_dirbe}). $\beta$=1.57 yields the best fit, 
with $\chi^2_{\nu}$=2.51. This result is in excellent agreement with that of 
$\S$\ref{sec:global}, where we found the best-fit single-MBB model to have 
$\beta$=1.59. 

We also evaluated $\chi^2_{\nu}$ for single-MBB models in which
$\beta$ varies spatially. In these cases, we adopted the 0.5$^{\circ}$ 
resolution $\beta$ map from \cite{planckdust}. We started by calculating 
$\chi^2_{\nu}$ using the \cite{planckdust} temperature map, finding 
$\chi^2_{\nu}$=4.68. Note that in this case no per-pixel chi-squared 
minimization was involved, as we simply evaluated $\chi^2_{p}$ for each pixel 
based on the fully-specified \cite{planckdust} emission model. Next, we tested 
a single-MBB model for which we adopted the \cite{planckdust} $\beta$ map, but 
allowed the per-pixel temperature and normalization to vary so as to minimize 
$\chi^2_p$. In this case, we found $\chi^2_{\nu}$=2.51, effectively identical 
to the value found for the spatially constant $\beta$=1.57 single-MBB model. 
This is perhaps unsurprising, as the average $\beta$ value from 
\cite{planckdust} over the mask in question is $\langle\beta\rangle$=1.58. 
This result does suggest, however, that the half-degree variations in $\beta$ 
are not materially improving the goodness-of-fit over the full frequency range 
100-3000 GHz relative to a model with appropriately chosen spatially constant 
$\beta$.

% mention that when t floats, we find a lower T by ~ 1 degree than the Planck 
%team?

We move on to evaluate two-component models, first calculating $\chi^2_{\nu}$ 
with the FDS99 global parameters (Table \ref{tab:global}, model 2). We find
$\chi^2_{\nu}$=2.33, a slight improvement relative to the best-fitting
single-MBB models. Finally, calculate $\chi^2_{\nu}$ for Table 
\ref{tab:global} model 3, the two-component model favored by our \PLANCK+DIRBE 
correlation slopes. In this case, we achieve the best goodness-of-fit out of 
all the models we have tested, with $\chi^2_{\nu}$=2.11.

Thus, our degree-resolution goodness-of-fit analysis has generally confirmed
the conclusions of $\S$\ref{sec:global}. We find the single-MBB $\beta$ value 
favored by the combination of \PLANCK~and DIRBE to be nearly identical here 
($\beta$=1.57) versus in $\S$\ref{sec:global} ($\beta$=1.59). As in 
$\S$\ref{sec:global}, we also find that the \PLANCK+FIRAS and \PLANCK+DIRBE 
best-fit two-component models from Table \ref{tab:global} outperform single-MBB
alternatives, though only by a relatively small margin. The agreement between 
our correlation slope analysis and the present goodness-of-fit analysis is 
especially encouraging for two main reasons: (1) in the present analysis, dust 
temperature has been allowed to vary on degree scales, whereas in 
$\S$\ref{sec:global} we assumed a single global dust temperature (2) in the
present analysis, our zero-level offsets factor into the dust temperature,
whereas in $\S$6 this was not the case, meaning the former and latter analyses 
agree in spite of their potential to be affected by rather different 
systematics.

% say that, becuase such little improvement for two-component versus
% single MBB in this analysis, we include a single-MBB beta = 1.57 fit in
% our data release ???

% could compare temperatures from correlation slope analysis versus here, but 
% masks are different, so i guess it's just not worth it

\section{Optical Reddening}
\label{sec:ebv}

While the temperature and optical depth maps thus far derived are
useful for making thermal dust emission foreground predictions, estimating 
optical reddening/extinction is another important application of the 
$\tau_{545}$ map. 

% mention that this is especially true because Planck gives such a huge
% increase in temperature resolution relative to SFD (and possibly reference
% temperature comparison figure as well)

\subsection{Reddening Calibration Procedure}

% scatterplot figure showing linear regression of EBV vs. tau
\begin{figure}
\begin{center}
\epsfig{file=calib_ebv.eps, width=3.3in}
\caption{\label{fig:calib} Linear fit of $E(B-V)_{SSPP}$ as function of
$\tau_{545}$.}
\end{center}
\end{figure}

We calibrate to reddening empirically rather than derive
a relationship between $\tau_{545}$ and reddening by introducing additional 
assumptions about the dust grain physics and size distribution. To achieve this
empirical calibration, we must adopt a set of calibrator objects for which true
optical reddening is known. There are various possibilities at our disposal. 
\cite{planckdust} calibrated their radiance and $\tau_{353}$ maps to $E(B-V)$ 
using broadband Sloan Digital Sky Survey \citep[SDSS;][]{sdss} photometry for a
set of $\sim$10$^5$ quasars. The SFD calibration was originally tied to a 
sample of 384 elliptical galaxies, but was later revised by 
\citet[hereafter SF11]{schlafly11} based on $\sim$260,000 stars with both 
spectroscopy and broadband photometry available from SDSS.

% maybe note the possibility in the future of calibrating to a 3D dust
% map type data product rather than directly to a set of individual 
% stars/galaxies 

% make argument that, in models with f1 and the beta's constant, the choice
% of frequency at which tau is evaluated does not matter

\begin{figure*}
\begin{center}
\epsfig{file=ebv_resid.eps, width=6.6in}
\caption{\label{fig:resid} (top left) Residuals of $E(B-V)_{2comp}$ relative to
$E(B-V)_{SSPP}$ as a function of $E(B-V)_{SFD}$. The grayscale represents the 
conditional probability within each $E(B-V)_{SFD}$ bin. The central black line 
shows the moving median. The upper and lower black lines represent the moving 
75th and 25th percentiles respectively. (bottom left) Residuals of 
$E(B-V)_{2comp}$ relative to $E(B-V)_{SSPP}$ as a function of hot dust 
temperature $T_2$. (top right) Same as top left, but illustrating the residuals
of $E(B-V)_{mbb}$, our calibration of the \cite{planckdust} $\tau_{353}$ to 
$E(B-V)_{SSPP}$. (bottom right)  Same as bottom left, but showing the 
$E(B-V)_{mbb}$ residuals as a function of the single-MBB dust temperature from 
\cite{planckdust}. The temperature axes always range from the 
0.4$^{th}$ percentile to 99.6$^{th}$ percentile temperature values.}
\end{center}
\end{figure*}

To calibrate our two-component optical depth to reddening, we make use of the 
stellar sample from SF11. Given a library of model stellar atmospheres, the 
spectral lines of these stars can be used to predict their intrinsic optical 
broadband colors. The `true' reddening is then simply the difference between 
the observed $g-r$ color and the $g-r$ color predicted from the spectral lines.
Applying a color transformation then yields `true' $E(B-V)$ values for 
$\sim$260,000 lines of sight. Throughout our SSPP calibration analysis, we 
restrict to the $\sim$230,000 lines of sight with $|b|$$>$20$^{\circ}$ in order
to avoid stars which may not lie behind the full dust column. The calibration 
of two-component optical depth to $E(B-V)$ is performed as a linear regression 
of $E(B-V)_{SSPP}$ versus $\tau_{545}$. $\tau_{545}$ is considered to be the 
independent variable in this regression, as we ultimately wish to predict 
$E(B-V)$ as a function of optical depth, and $\tau_{545}$ has much higher S/N 
than the SSPP $E(B-V)$ estimates.

% MENTION THAT THE TAU TEMPLATE IS INDEPENDENT OF FREQUENCY FOR 
% TWO-COMPONENT MODELS WITH SPATIALLY CONSTANT BETAS, Q1/Q2 AND F1

% need to explain SSPP acronym somewhere ?
% cite Green appendix for details of deriving EBV_SSPP from Schlafly 11 data ?

This regression is illustrated in Figure \ref{fig:calib}. As expected, there 
is a strong linear correlation between $E(B-V)_{SSPP}$ and $\tau_{545}$. The 
conversion factor from $\tau_{545}$ to $E(B-V)$ is 2.46$\times$10$^{3}$. 
Reassuringly, the best-fit offset is quite close to zero, $\sim$0.6 mmag.

Figure \ref{fig:resid} shows the residuals of our $\tau_{545}$-based reddening 
predictions, $E(B-V)_{2comp}$, relative to the corresponding SF11 reddening 
measurements, $E(B-V)_{SSPP}$, as a function of SFD reddening, $E(B-V)_{SFD}$, 
(top left panel) and as a function of hot dust temperature (bottom left panel).
For comparison, the right panels show analogous residual plots, but with 
respect to reddening predictions based on our calibration of the 
\cite{planckdust} 353 GHz optical depth to $E(B-V)_{SSPP}$ , using the same 
regression procedure employed to calibrate $E(B-V)_{2comp}$. We refer to these 
reddening predictions based on the \cite{planckdust} single-MBB model and 
calibrated to the SF11 measurements as $E(B-V)_{mbb}$.

All four residual plots in Figure 12 show systematic problems at some level. 
The most striking systematic trend is the `bending' behavior of the reddening
residuals versus $E(B-V)_{SFD}$ (top panels), with the median residual 
bottoming out near $-10$ mmag at $E(B-V)_{SFD}$$\approx$0.15 mag. This behavior
is common to both $E(B-V)_{2comp}$ and $E(B-V)_{mbb}$, and in fact was first 
noted in the residuals of $E(B-V)_{SFD}$ itself relative to $E(B-V)_{SSPP}$ by 
SF11 (see their Figure 6). Such a bending behavior is troubling because it 
could indicate a nonlinearity common to many FIR reddening predictions based on
column densities inferred from dust emission. Alternatively, because the SF11 
stars are distributed over the sky in a highly non-uniform manner, the bend 
could arise from aliasing of discrepancies particular to certain sky regions 
(e.g. inner vs. outer Galaxy) on to the $E(B-V)_{SFD}$ axis.

% could add a sentence praising the two-component reddening prediction for
% only having issues at the ~<10 mmag level for EBV_SFD ~< 0.4 mag, since
% it could be a lot worse ...

The obvious culprit for any potential nonlinearity in FIR-based reddening
estimates is a faulty temperature correction. For this reason, we have included
the bottom panels of Figure \ref{fig:resid}, to check for the presence of
a temperature dependence of the reddening residuals. Indeed, in both the
two-component and single-MBB cases there exists some systematic dependence of
the reddening residuals on temperature. For $T_{mbb}$$\gtrsim$19 K, the 
median residual is reasonably flat, but at lower temperatures (the lowest 
temperature $\sim$20\% of the sky), the median shows trends at the $\sim$10 
mmag level. On the other hand, the median residual in the two-component case 
trends downward with increasing $T_2$ over the entire $T_2$ range shown, with a
peak-to-peak amplitude of $\sim$20 mmag.

\subsection{Rectifying Reddening Residuals}
In this section we describe our attempts to eliminate the systematic
problems in the two-component reddening residuals shown in the left column
of Figure \ref{fig:resid}. We employed two main strategies: (1) recomputing
the two-component $\tau_{545}$ by re-running our Markov chains after modifying 
the input maps and/or changing the particular two-component model adopted and
(2) making spatial cuts to isolate sky regions in which the 
residuals are especially pristine (or especially problematic).

The following is a list of dust model modifications we tested, but which
proved to have little impact on the reddening residual trends as a function 
of either $E(B-V)_{SFD}$ or $T_2$:

\begin{itemize}
\item Varying each of the global two-component model parameters $\beta_1$, 
$\beta_2$, $q_1/q_2$ and $f_1$ individually while holding the others fixed.
\vspace{-3mm}
\item Allowing $f_1$ to vary spatially as in the fits of $\S$\ref{sec:lores}.
\vspace{-2mm}
\item Changing the mean and/or variance of the $T_2$ prior.
\vspace{-6mm}
\item Varying multiple global parameters at a time e.g. both $f_1$ and 
$q_1/q_2$, restricting to regions of parameter space favored by our 
goodness-of-fit analyses described in $\S$\ref{sec:global} and 
$\S$\ref{sec:hier}.
\end{itemize}

We additionally investigated the following spatial cuts which did not resolve 
the dominant problems noted in the reddening residuals:
\begin{itemize}
\item Separating Celestial north and south.
\vspace{-3mm}
\item Separating Galactic north and south.
\vspace{-3mm}
\item Separating inner and outer Galaxy.
\vspace{-3mm}
\item Combining the above two sets of cuts i.e. separating the Galaxy into
quadrants. 
\vspace{-3mm}
\item Combining these spatial cuts with the dust model changes of the previous
list.
\end{itemize}

However, we found that changing the zero level offsets of the input maps 
had a significant effect on the strength of the anticorrelation between median 
reddening residual and $T_2$. In particular, we experimented with perturbing 
the zero level offset of \PLANCK~857 GHz while correspondingly changing the 
zero levels of the remaining \PLANCK~maps based on the prescription of 
$\S$\ref{sec:relzero}. We also experimented with changing the zero level of SFD
\verb|i100|, independent of the other zero levels. Unfortunately, completely
flattening the reddening residual dependence on $T_2$ required unreasonably 
large zero level modifications. For example, flattening the $T_2$ residual 
required adding $\gtrsim$0.6 MJy/sr to the \verb|i100| map. Such an offset is 
implausible, being an order of magnitude larger than the nominal \verb|i100| 
zero level uncertainty quoted by SFD98, and comparable to the entire 3000 GHz 
CIB monopole signal. Furthermore, we note that even these large zero 
level modifications had virtually no effect in eliminating the reddening 
residual `bend' versus $E(B-V)_{SFD}$. Thus, changing the zero level 
offsets showed hints of promise in rectifying the reddening residual 
temperature dependence, but cannot completely resolve the systematic trends in 
reddening residuals.
% last few words here are kind of bad, need better terminology

The only solution we have been able to identify that both removes the `bend' 
vs. $E(B-V)_{SFD}$ and simultaneously reduces the temperature dependence of the
reddening residuals is cutting out the ecliptic plane by restricting to 
$|\beta|>20^{\circ}$. In this case, we completely eliminated the bending 
behavior of the residual versus $E(B-V)_{SFD}$, and significantly reduced the 
$T_2$ dependence to a peak-to-peak amplitude of only $\sim$10 mmag (see Figure 
\ref{fig:resid_ecl}). Figure \ref{fig:resid_ecl} still includes the single-MBB 
plots (right column), to show that the bend versus $E(B-V)_{SFD}$ is eliminated
by the $|\beta|$ cut, even for the single-MBB model. However, the single-MBB 
residuals still differ systematically from zero for $T$$\lesssim$$19$ K. 
Perhaps the improvements in the two-component reddening residuals after 
restricting to high ecliptic latitude should come as no surprise, given that 
the ecliptic plane is the most obvious systematic problem with our temperature 
map (see the full-sky results shown in Figure \ref{fig:results}).

\begin{figure*}
\begin{center}
\epsfig{file=ebv_resid_ecl.eps, width=6.6in}
\caption{\label{fig:resid_ecl} Same as Figure \ref{fig:resid}, but 
restricting to high ecliptic latitude, $|\beta|>20^{\circ}$. The top left and 
top right plots both show that the bending of the reddening residuals as a 
function of $E(B-V)_{SFD}$ as seen in Figure \ref{fig:resid} has been 
eliminated. Further, the two-component reddening residual temperature 
dependence (bottom left) has been significantly reduced relative to the
corresponding trend shown in Figure \ref{fig:resid}, bottom left panel.}
\end{center}
\end{figure*}

After cutting the ecliptic plane, we found that only small zero level
perturbations were required to fully flatten the temperature residuals,
while still maintaining flat residuals versus $E(B-V)_{SFD}$. The optimal
offsets we found were $\pm$0.08 MJy/sr to \verb|i100| and 857 GHz respectively
(see Figure \ref{fig:resid_offs}). These offsets are well within reason, 
given the nominal zero level uncertainties quoted in Table \ref{table:offs}.

\begin{figure}
\begin{center}
\epsfig{file=resid_offs.eps, width=3.3in}
\caption{\label{fig:resid_offs} $E(B-V)_{2comp}$ residuals as a function of
SFD $E(B-V)$, including a cut on ecliptic latitude, $|\beta|>20^{\circ}$, 
and zero level perturbations of $\pm$0.08 MJy/sr to i100 and 
857 GHz respectively. The grayscale represents the conditional probability 
within each bin in SFD $E(B-V)$. The central black line shows the moving 
median. The upper and lower black lines represent the moving 75th and 25th 
percentiles respectively.}
\end{center}
\end{figure}

% but then we tried ECL cuts, and in combination with modest offsets, that 
% worked


%%\section{Systematics}
%%
%%\subsection{CIBA}
%%Perhaps the dominant limitation of our fits is the presence of cosmic
%%infrared background anisotropies (CIBA) bleeding through into our maps of 
%%Galactic dust temperature and optical depth in high-latitude regions 
%%\citep{ciba}.
%%\subsubsection{Simulations}

%%\subsubsection{Optimizing Relative Weights}

%%\subsection{Zero Point}
%%Simulate the effect of getting zero points wrong to quantiy this
%%systematic a bit.

%%\subsection{$T_{2}$ Prior}
%%Use FIRAS data to justify $T_2$ prior. Could I use the DIRBE 100/240 micron
%%ratio to get a $1.3^{\circ}$ resolution prior on $T_2$?

%%\subsection{Other Limitations}
%%\IRAS~is missing in some parts of sky which will therefore not truly have 
%%6.1' resolution.
%%any other miscellaneous issues?

\section{Low-frequency Emission Predictions}
% compare 100-217 GHz predictions using my model versus Planck Collaboration
% single-MBB model

% is this the section where, potentially, plots of the residuals might go?

%striping in IRAS/IRIS versus in SFD IRAS ... planck dust map obvious striping
An important difference between our maps and those of \cite{planckdust} is our 
use of SFD \verb|i100| instead of IRIS 100$\mu$m \citep{IRIS}. Residual 
\IRAS~striping at levels higher than that remaining in SFD \verb|i100| is 
clearly visible in the \cite{planckdust} maps.

%we actually have 6.1' arcminute resolution
It is true that we have smoothed the input intensity maps for this study 
to $6.1'$ FWHM, which is nominally lower resolution than the $5'$ 
\cite{planckdust} maps. However, as is apparent from figure 
\ref{fig:comparison}, the fact that \cite{planckdust} modeled $\beta$ at 
$0.5^{\circ}$ resolution has effectively blurred their dust temperature map.

%\subsection{Validation with FIRAS}

\section{Data Release}
\label{sec:release}
We are releasing $N_{side}$=2048 HEALPix maps in Galactic coordinates 
summarizing the results of of our two-component dust fits.

% if we're not going to be claiming to be superseding SFD in terms of 
% reddening, then do we even want to include anything extinction-related
% (data files and/or code) in this data release ?

%at some point in this should make statements re: the fact that in terms
%of predicting *emission* the conversion to tau is unnecessary and
%and actually the raw uncertainty on Ghz 545 amplitude is the relevant quantity

%software utilities
%provide maps at multiple different nsides ?
%   -> if so, actually smooth the maps before rebinning, or leave 
%      undersampled?
% provide uncertainty maps for tau and T2 ??

\begin{figure*} [ht]
\begin{center}
\epsfig{file=results.eps, width=7.0in}
\caption{\label{fig:results} Our best-fit $T_2$, binned to 27.5$'$ resolution}
\end{center}
\end{figure*}

\section{Conclusions}
\label{sec:conclusion}
% should mention that one definitive improvement of this work relative
% to FDS99 is that there are uncertainty/covariances provided for
% the derived parameters !!!!


This material is based upon work supported by the National Science Foundation 
Graduate Research Fellowship under Grant No. We acknowledge support of NASA 
grant NNX12AE08G for this research. Based on observations obtained with Planck 
(http://www.esa.int/Planck), an ESA science mission with instruments and 
contributions directly funded by ESA Member States, NASA, and Canada. This 
research made use of the NASA Astrophysics Data System (ADS) and the IDL 
Astronomy User's Library at Goddard. \footnote{Available at 
\texttt{http://idlastro.gsfc.nasa.gov}}

\bibliographystyle{apj}
\bibliography{twocomp.bib}

\end{document}
