\documentclass{emulateapj} 
 
\usepackage[dvipdf]{epsfig} 
\usepackage[dvips]{rotating}
\usepackage{subfigure}

\newcommand{\IRAS}{{\it IRAS}}
\newcommand{\HERSCHEL}{{\it Herschel}}
\newcommand{\SPITZER}{{\it Spitzer}}
\newcommand{\PLANCK}{{\it Planck}}
\newcommand{\AKARI}{{\it Akari}}

\bibpunct{(}{)}{;}{a}{}{,} 
 
\shorttitle{\PLANCK~Dust Model}
 
\shortauthors{Meisner \& Finkbeiner} 

\begin{document}

\title{Two-component thermal dust emission model: application to the {\it PLANCK} HFI maps}
\author{Aaron M. Meisner\altaffilmark{1,2}}
\author{Douglas P. Finkbeiner\altaffilmark{1,2}}
\altaffiltext{1}{Department of Physics, Harvard University, 17 Oxford Street, 
Cambridge, MA 02138, USA; ameisner@fas.harvard.edu}
\altaffiltext{2}{Harvard-Smithsonian Center for Astrophysics, 60 Garden St, 
Cambridge, MA 02138, USA; dfinkbeiner@cfa.harvard.edu}

\begin{abstract}
We present full-sky maps of dust optical depth and temperature derived by 
applying the \cite{FDS99} two-component dust emission model fitted to 
\PLANCK~217, 353, 545, 857 GHz and \cite{SFD} \IRAS~100$\mu$m data. Both the 
optical depth and temperature maps have angular resolution of $6.1'$. Such maps
 serve as an important alternative to and comparison for other dust 
emission models, especially the single modified blackbody (MBB) model in which 
$\beta$ and $T$ both vary spatially. We also expect the derived optical 
depth to provide valuable cross-checks for other dust-column related data 
products, specifically the \cite{SFD} map, which assumed 
$\beta = 2$ single-component dust emission and included a temperature map of 
inferior $\sim$$1.3^{\circ}$ resolution. We briefly compare our 
best-fit two-component spectral energy distributions to those of the 
\cite{planckdust} single-component parametrization.
%mention the enhanced temperature resolution relative to SFD
%Our optical depth map will also provide a useful reference for dust maps 
%based on stellar colors rather than infrared emission.
\end{abstract}

\section{Introduction}
%why dust important in astronomy/astrophysics
The presence of Galactic interstellar dust affects
astronomical observations over a wide range of wavelengths. At mid-infrared
and far-infrared wavelengths, Galactic dust emission is a major contributor
to the observed sky intensity. In the optical and ultraviolet, extinction
by foreground dust grains attenuates the signal from extragalactic sources
over the entire sky. Understanding the properties of dust in the interstellar 
medium (ISM) is valuable in its own right, as the ISM is the site of star 
formation, thereby playing a crucial role in galaxy evolution. Equally, or 
perhaps even more important to the practice of astronomy, however, is 
accurately accounting for dust as a foreground which reddens optical/UV o
bservations and superimposes Galactic emission on low-frequency observations 
of the cosmic microwave background (CMB). 
%specify ``low frequency more precisely eventually
% somewhere mention re-radiation

%basic background about interstellar dust emission
Over the past decades, satellite observations have dramatically enhanced our
knowledge about infrared emission from the ISM. IRAS, with its $\sim$6$'$ 
resolution, revolutionized the study of Galactic dust emission, first revealing
 the high-latitude ``infrared cirrus'' using \IRAS~60$\mu$m and 100$\mu$m 
observations \citep{low84, wheelock94}. Beginning in 1989, DIRBE  mapped the 
full sky at ten infrared wavelengths from 1.25$\mu$m to 240$\mu$m with a 
reliable zero point, but inferior 0.7$^{\circ}$ angular resolution 
\citep{boggess92}.mention FIRAS specifically, since it is important to FDS99.


FDS99 used FIRAS \citep{firas} to make a globally best-fit model of dust 
emission appropriate over a very broad range of frequencies. They showed that 
a model consisting of a single MBB with power law index $\beta$ 
could not fit the IRAS/FIRAS/DIRBE spectrum at both the Wien and 
Rayleigh-Jeans extremes. To fit both ends of the spectrum, FDS99 proposed
a spectrum consisting of two MBBs, each with a different
temperature and power law index. Physically, these two components
represent distinct dust grain species within the ISM, specifically 
silicates and carbonaceous grains.


With the advent of the Planck mission, we have an important opportunity to
revisit foreground predictions in light of their superb data. Brief Planck
background. \cite{planckdust} has recently conducted a study modeling
\PLANCK~353 GHz, 545 GHz, 857 GHz and IRAS 100$\mu$m emission with a single
MBB spectrum. Here we investigate the FDS99 two-component dust emission model 
as an alernative model for the dust SED composed of \PLANCK~and \IRAS~data. In
doing so, we obtain $6.1'$ maps of dust temperature and optical depth. Because
we are using a model that has been validated using FIRAS data down to sub-mm
wavelengths, this model should be useful in constructing predictions of dust
emission over a very broad range of wavelengths. This includes low frequencies,
which \cite{planckdust} admit their model does not adequately fit, and also
data near the peak of the dust SED, for example \AKARI. We also anticipate 
the optical depth map we derive will be useful as a comparison for extinction
estimates based directly upon optical/NIR observations of stars.

%would be nice if i had a citation here for Eddie/Greg pan-starrs dust stuff

In $\S$\ref{sec:data} we discuss the data used in throughout this study. In 
$\S$\ref{sec:prepro} we describe our preprocessing of the data to isolate 
thermal emission from Galactic dust. In $\S$\ref{sec:modeling} we describe the 
dust emission model we fit to the \PLANCK and \IRAS~data. In 
$S$\ref{sec:fitting} we provide details of the Markov Chain Monte Carlo method 
we use to estimate the parameters of our model. In $\S$\ref{sec:results} we 
present the full-sky maps of dust temperature and optical depth we have 
obtained, and conclude and $\S$\ref{sec:conclusion}


%how what we're doing fits in
%     where FDS99 fit in
%     Planck data qualities that improve on data used in FDS99
%     benefits of using Planck + FDS99 model
%utility/applications of what we're doing

%list of figures : 
% posterior surface for some pixel, with samples from markov chain overplotted
%    --> also overplot output best-fit (T,I545) and unctertainty estimate
% plot for some set of pixels of SED overlaid with best-fit two-component model
%    --> done in Python instead of IDL
%    --> overlay 
% plot of resulting T2 and tau
% plots of full-sky residuals relative to our model ?
%  ---> and for lower-frequency bands to emphasize that FDS99 is better for
%       extrapolating towards mm wavelengths
% zoom-in of region with interesting dust showing :
%  SFD T map, two-component T map, Planck team one-component T map
%      good spot is in polaris, centered on (lgal,bgal) ~ (112, 20)

\section{Data}
\label{sec:data}
All \PLANCK~data products utilized throughout this work are drawn from the 
\PLANCK~2013 data release \citep{planck2013}. Specifically, we have made use 
of the the 217 GHz, 353 GHz, 545 GHz and 857 GHz intensity maps. We use the 
versions of these maps which have been corrected for zodiacal light 
\citep[\texttt{R1.10\_nominal\_ZodiCorrected},][]{planckzodi}.

%TODO: actually smooth all of the maps to a common PSF of 6.1 arcmin.

To incorporate information about the dust emission on the Wien side of its 
spectrum, we make use of 100$\mu$m data. In particular, we use the SFD 
reprocessing of IRAS 100$\mu$m, which we will refer to as \verb|i100|, and at 
times by frequency as 3000 GHz. \verb|i100| has angular resolution of $6.1'$. 
%list other %important characteristics of i100 e.g. that it has been 
%zodi-corrected and relies on dirbe on scales larger than ???

\section{Pre-processing}
\label{sec:prepro}

\subsection{Smoothing}
We wish to simultaneously fit \verb|i100| along with four \PLANCK~bands. To 
properly combine these maps, they must have the same point spread function 
(PSF). \verb|i100|, with its $6.1'$ symmetric Gaussian PSF, is the 
lowest-resolution map we have included. To match PSFs, we have therefore 
smoothed each of the \PLANCK~maps under consideration to $6.1'$ resolution by 
considering each native map to have a symmetric Gaussian beam and smoothing by 
the appropriate symmetric Gaussian such that the resulting map has a FWHM of 
$6.1'$. The FWHM we assign to the native \PLANCK~maps is taken from 
\cite{planckbeam}, specifically $5.01'$ for 217 GHz, $4.86'$ for 353 GHz, 
$4.84'$ for 545 GHz, and $4.63'$ for 857 GHz.

\subsection{Unit Conversions}

Some of the maps in native format are provided in thermodynamic $\delta T$, 
while others are provided in units of MJy/sr. Before performing any fits, 
we convert all measurements to MJy/sr following the guidelines of \cite{planckresponse}.

%i think it's table 5 that I specifically want to cite within the ``spectral
%response'' paper, but i need to check that those values are consistent
% with the ones i actually used from the other document ``HFI-unit conversion 
%and colour correction instructions v1.2''

\subsection{CMB Removal}
The CMB is effectively imperceptible at 857 GHz, but can be noticed at a 
low level at 545 GHz, and is prominent at 217 GHz and 353 GHz relative 
to the diffuse Galactic emission we wish to characterize, especially at high
latitudes. To remove the CMB anisotropies, we have subtracted the SMICA model
from each of our Planck maps \citep{smica}, after appropriate unit conversions.

\subsection{Finite Bandpass Correction}
The thermal dust emission model of $\S$\ref{sec:modeling} predicts the 
intensity $M_{\nu}$ in MJy/sr for any single frequency $\nu$. But in reality
we are comparing to broad bands typically hundreds of GHz wide. The shape of 
model spectrum $M_{\nu}$ is only dependent upon one parameter: the hot
dust temperature $T_2$. We have therefore computed bandpass correction factors
$b_{\nu_i}$ such that $M_{\nu_i}b_{\nu_i}$ is the predicted observation for the
\PLANCK~or \IRAS~band of interest. We have used the \PLANCK~transmission 
curves provided in the file \verb|HFI_RIMO_R1.10.fits| \citep{planckresponse}. 
For 3000 GHz, we have used the \IRAS~bandpass, following the treatment of
\cite{planckdust}. We note that it is not clear which bandpass should
properly be considered for the IRAS/DIRBE composite 100$\mu$m maps used in 
this analysis (here SFD \verb|i100|) and that of \cite{planckdust}.

% check that RIMO file i've used is actually the most up-to-date version

\subsection{Correlation versus HI}
The raw \PLANCK~maps we have downloaded show offsets relative to HI, in the
sense that the FIR emission does not go to zero when HI emission goes to
zero. Since HI correlates linearly with far-IR emission in the diffuse ISM, 
when the optical depth is low and molecular effects are negligible, we modeled
far IR emission as a constant times HI emission, plus some offset. We then
subtracted the best-fit offset such that 

\subsection{Molecular Emission}
One advantage of the FIRAS data used by FDS99 was its many narrow bins in 
frequency, which allowed those frequency ranges contaminated by strong 
molecular line emission to be discarded from the analysis. Unfortunately, with 
the broad \PLANCK~and \IRAS~bandpasses we do not have such a luxury. This means
 that we cannot truly separate out the thermal continuum of dust emission we 
wish to study from the molecular line emission. The most prominent molecular 
line emission in the Planck bands of interest arises from CO lines in the XX, 
YY bands. We investigated the possibility of subtracting out CO emission based 
on the template of \cite{planckco}. However, we found that the CO emission
rarely amounted to more than XX\% of the total intensity, and following
the treatment of \cite{planckdust}, $\S$2.1, we do not attempt to remove CO
emission.

\subsection{Compact Sources}
To mask compact sources, we use the SFD compact source mask, which was 
originally constructed to mask point sources and resolved compact objects
that are not part of the Galactic cirrus at 3000 GHz. Given our pixelization 
(see $\S$\ref{sec:pix}), 1.56\% of pixels are masked. We have inspected all of 
the maps used in this analysis and do not see any evidence that this compact 
source mask fails to adequately mask compact sources in the \PLANCK~data. In 
practice, we perform our dust SED fits in all pixels, even those flagged by 
the point source mask. Compact source affected pixels can be ignored or 
interpolated over downstream.
%maybe say that we're releasing a version of the map with point sources
%interpolated over by us as well


\section{Dust Emission Model}
\label{sec:modeling}
Our model is that of FDS99, which consists of two populations of dust 
grains emitting as MBBs, but with different temperatures
and different emissivity power law indices $\beta$. The shape of this spectrum 
is given by:

\begin{equation}
M_{\nu} \propto f_{1}q_{1}(\frac{\nu}{\nu_{0}})^{\beta_1}B_{\nu}(T_1) + (1-f_{1})q_{2}(\frac{\nu}{\nu_0})^{\beta_2}B_{\nu}(T_2)
\end{equation}

With $\nu_0$ = 3000 GHz, $f_{1}$ = 0.0363, $\beta_1$ = 1.67, $\beta_2$ = 2.70, 
and $q_1/q_2$ = 13.0. For the parameters thus far specified, the two dust 
temperatures are further related by:

\begin{equation}
T_1 = 0.352T_2^{1.18}
\end{equation}

At each sky location, we have five independent intensity measurements (four ) 
FDS99 found typical values for $T_2$ and 

\section{Fitting Procedure}
\label{sec:fitting}

\subsection{Pixelization}
\label{sec:pix}
For the purpose of fitting, we break the sky into independent pixels of 
angular size $\sim$1.72$'$, defined by the HEALPix pixelization with nside=2048
\citep{healpix}. This is convenient because the Planck maps are provided in 
this format and this pixelization adequately samples the $6.1'$ resolution maps
under consideration. We reiterate that each pixel is fit independently of 
neighboring pixels.

\subsection{Markov Chains}
As explained in $\S$\ref{sec:modeling}, the two free parameters in our 
per-pixel dust emission SED are the hot dust temperature $T_2$, which 
determinesthe shape of the SED, and the spectrum normalization. In practice,
the normalization is controlled via $\tilde{I}_{545}$, the amplitude of the 
model SED in MJy/sr at $\nu$ = 545 GHz. For each pixel, we run a 
Metropolis-Hastings Markov chain sampling the posterior probability as a 
function of the two parameters $T_2$ and $\tilde{I}_{545}$. For each pixel, we 
are sampling the posterior given by:

\begin{equation}
P(\tilde{I}_{545}, T_2|\{I\}) \propto \mathcal{L}(\{I\}|\tilde{I}_{545}, T_2)P(T_2)
\end{equation}

The likelihood function is given by:
\begin{equation}
\mathcal{L}(\{I\}|\tilde{I}_{545}, T_2) = \displaystyle\prod\limits_{i}\mathcal{N}(M_{\nu_{i}}(\tilde{I}_{545}, T_2)|I_{\nu_{i}}, \sigma_{\nu_i})
\end{equation}

%be more careful about notation re: color corrections

Where the product runs over $\nu_i$ in $\{217,\ 353,\ 545,\ 857,\ 3000\}$ GHz. 
For each pixel, the errors on each intensity measurement $\sigma_{\nu_i}$ are 
given by:

\begin{equation}
\sigma_{\nu_i} = \sqrt{c^2_{\nu_i}I^2_{\nu_i} + c^2_{\nu_i}CMB^2_{\nu_i} + (\delta O_{\nu_i})^2 + n^2_{\nu_i}}
\end{equation}

This error budget is the same as that of \cite{planckdust} Equation 7. 
$c_{\nu_i}$ is the multiplicative calibration uncertainty for the intensity in 
each band. We have adopted the $c_{\nu_i}$ values listed in \cite{planckdust} 
Table 1. For the 217 GHz band, we have assigned $c_{217 GHz}$ = 0.54, from 
Table 11 of \cite{planckcalib}. $CMB$ represents the CMB intensity 
subtracted at the relevant pixel. $\delta0_{\nu_i}$ represents the uncertainty
in the offset used to tie \PLANCK~intensity to HI. We have adopted the 
 $\delta0_{\nu_i}$ values of \cite{planckdust} Table 1. $n_{\nu_i}$ represents 
the instrumental noise in the pixel of interest, and is taken to be the square 
root of the \verb|ii_cov| parameter that accompanies each \PLANCK~intensity 
map. In order to obtain reasonable fitting results at high latitude where the 
data is noisy, we include a prior on $T_2$:

%what about the offset uncertainty for 217 GHz

\begin{equation}
P(T_2) = \mathcal{N}(T_2|\bar{T}_2, \sigma_{\bar{T}_2})
\end{equation}

With $\bar{T}_2$ = 16.2 K and $\sigma_{\bar{T}_2}$ = 1.4 K. In practice we 
always perform computations using the logarithm of relevant probabilities.
%mention something about the proposal distribution at some point?

For each pixel, we initialize the Markov chain to have $T_2$ = $\bar{T}_2$ and
. We run 500 steps of burn-in and then 2000 steps during which we keep track
of the proposed $T_{2, j}$ and $I_{545, j}$ values. We use these 2000 values 
for each parameter to compute estimates of each parameter,  $T_2$ = 
$\langle T_{2, j} \rangle$, $I_{545}$ = $\langle I_{545, j} \rangle$, and of 
each parameter's variance, $\sigma^2_{T_2}$ = $\langle T^2_{2, j} \rangle-\langle T_{2, j} \rangle ^2$ and $\sigma^2_{I_{545}}$=$\langle I^2_{545, j} 
\rangle-\langle I_{545, j} \rangle ^2$

\section{Results}
\label{sec:results}

\begin{figure}
\begin{center}
\epsfig{file=tcomparison.eps, width=3.3in}
\caption{\label{fig:comparison} Comparison of SFD temperature, two-component 
model $T_2$, and \cite{planckdust} temperature. Note the differing colorscales.
Both models incorporating \PLANCK~data clearly show a major improvement in 
resolution relative to SFD.}
\end{center}
\end{figure}

\subsection{Calculating $\tau$}

\subsection{Data Release}
We are releasing nside=2048 HEALPix maps in Galactic coordinates summarizing 
the results of of our two-component dust fits.

%provide maps at multiple different nsides ?
%   -> if so, actually smooth the maps before rebinning, or leave 
%      undersampled?
% provide uncertainty maps for tau and T2 ??

\subsection{Limitations}
Perhaps the dominant limitation of our fits is the presence of cosmic
infrared background anisotropies (CIBA) bleeding through into our maps of 
Galactic dust temperature and optical depth in high-latitude regions 
\citep{ciba}.
% CIBA !!!
% IRAS missing so that parts of sky not truly 6.1' resolution map

\subsection{Comparison to Planck Collaboration Dust Map}
%striping in IRAS/IRIS versus in SFD IRAS ... planck dust map obvious striping
An important difference between our maps and those of \cite{planckdust} is our 
use of SFD \verb|i100| instead of IRIS 100$\mu$m \citep{IRIS}. Residual 
\IRAS~striping at levels higher than that remaining in SFD \verb|i100| is 
clearly visible in the \cite{planckdust} maps.

%we actually have 6.1' arcminute smoothing
It is true that we have smoothed the input intensity maps for this study 
to $6.1'$ FWHM, which is nominally lower resolution than the $5'$ 
\cite{planckdust} maps. However, as is apparent from figure 
\ref{fig:comparison}, the fact that \cite{planckdust} modeled $\beta$ at 
$0.5^{\circ}$ resolution has effectively blurred their dust temperature map.


\subsubsection{Omitting 217 GHz}
The \cite{planckdust} result used only 353, 545, 857, and 3000 GHz. When we 
repeat our analysis ignoring 217 GHz, so that we are using the same
set of frequencies, we find that no major changes arise. This actually
should be done/investigated though.

\section{Conclusion}
\label{sec:conclusion}

\begin{figure*} [ht]
\begin{center}
\epsfig{file=results.eps, width=7.0in}
\caption{Our best-fit $T_2$, binned to 27.5$'$ resolution}
\end{center}
\end{figure*}

This material is based upon work supported by the National Science Foundation 
Graduate Research Fellowship under Grant No.Based on observations obtained with
 Planck (http://www.esa.int/Planck), an ESA science mission with instruments 
and contributions directly funded by ESA Member States, NASA, and Canada. This 
research made use of the NASA Astrophysics Data System (ADS) and the IDL 
Astronomy User's Library at Goddard. \footnote{Available at 
\texttt{http://idlastro.gsfc.nasa.gov}}

\bibliographystyle{apj}
\bibliography{twocomp.bib}

\end{document}
