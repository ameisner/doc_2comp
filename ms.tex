\documentclass{emulateapj} 
 
\usepackage[dvipdf]{epsfig} 
\usepackage[dvips]{rotating}
\usepackage{subfigure}

\newcommand{\IRAS}{{\it IRAS}}
\newcommand{\HERSCHEL}{{\it Herschel}}
\newcommand{\SPITZER}{{\it Spitzer}}

\bibpunct{(}{)}{;}{a}{}{,} 
 
\shorttitle{Planck Dust Model}
 
\shortauthors{Meisner \& Finkbeiner} 

\begin{document} 

\title{Two-component thermal dust emission model: application to the Planck HFI maps}
\author{Aaron M. Meisner\altaffilmark{1,2}}
\author{Douglas P. Finkbeiner\altaffilmark{1,2}}
\altaffiltext{1}{Department of Physics, Harvard University, 17 Oxford Street, 
Cambridge, MA 02138, USA; ameisner@fas.harvard.edu}
\altaffiltext{2}{Harvard-Smithsonian Center for Astrophysics, 60 Garden St, 
Cambridge, MA 02138, USA; dfinkbeiner@cfa.harvard.edu}

\begin{abstract}
We have fit the FDS99 two-component dust emission model over
the full sky, based on dust SEDs comprised of  Planck HFI and IRAS 100$\mu$m. 
As a result, we have obtained maps of optical depth $\tau_{545}$ and dust 
temperatures $T_1$ and $T_2$,  with angular resolution  $\sim$6$'$. These maps 
serve as an interesting comparison to alternative models, for example 
one-component models with varying $T$ and $\beta$. We also expect the derived 
optical depth to be useful as a cross-check for other extinction estimates, 
such as the SFD98 map, which assumed $\beta = 2$ single-component dust 
emission, as well as forthcoming dust maps based on observations of stars in 
the optical rather than infrared emission.
\end{abstract}

\section{Introduction}

\section{Data}

We have made use of the Planck 2013 data release, specifically the 217 GHz, 
353 GHz, 545 GHz and 857 GHz channels Planck Collaboration (2013a). We use the 
versions of these maps which have been corrected for zodiacal light by the 
Planck team, using the zodiacal light model described in Planck Collaboration 
(2013b). For each pixel on the sky, Planck thus provides us with four data 
points of the dust emission SED. The angular resolution of these maps is 
$5.5$. TODO: actually smooth all of the maps to a common PSF of XX arcmin.

To add information about the dust emission on the Wien side of its spectrum,
we include IRAS 100$\mu$m. We use the SFD reprocessing of IRAS 100$\mu$m, which
we will refer to as \verb|i100| (Schegel et al. 1998). \verb|i100|
\section{Pre-processing}

\subsection{Unit Conversions}

Some of the maps in native format are provided in thermodynamic $\delta T$, 
while others are provided in units of MJy/sr. Before performing any fits, 
we convert all measurements to MJy/sr following the guidelines of the document
``HFI-unit conversion and colour correction instructions v1.2''.

\subsection{Zodiacal Light Removal}
Cite the appropriate reference for the Planck ZL model.

\subsection{Smoothing to common resolution}

\subsection{CMB Subtraction}
The CMB is effectively imperceptible at 857 GHz, but can be noticed at a 
low level at 545 GHz, and is extremely bright at 217 GHz and 353 GHz relative 
to the diffuse Galactic emission we wish to characterize.


\subsection{Finite Bandpass Correction}

\subsection{Correlation against HI}
The raw Planck maps we have downloaded show offsets relative to HI, in the
sense that the FIR emission does not go to zero when HI emission goes to
zero. Since HI correlates linearly with far-IR emission in the diffuse ISM, 
when the optical depth is low and molecular effects are negligible, we modeled
far IR emission as a constant times HI emission, plus some offset. We then
subtracted the best-fit offset such that 

\subsection{Molecular Emission}

\section{Model}
Our model is that of FDS99, which consists of two populations of dust 
grain emitting as modified blackbodies, but with different temperatures
and different power law emissivities $\beta$.


\subsection{Fitting Methodology}

\begin{figure*}[ht]
\begin{center}
\epsfig{file=t2_0128.eps, width=7.0in}
\caption{Our best-fit $T_2$, binned to 27.5$'$ resolution}
\end{center}
\end{figure*}

\end{document}
