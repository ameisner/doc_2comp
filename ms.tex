\documentclass{emulateapj} 
 
\usepackage[dvipdf]{epsfig} 
\usepackage[dvips]{rotating}
\usepackage{subfigure}

\newcommand{\IRAS}{{\it IRAS}}
\newcommand{\HERSCHEL}{{\it Herschel}}
\newcommand{\SPITZER}{{\it Spitzer}}
\newcommand{\PLANCK}{{\it Planck}}

\bibpunct{(}{)}{;}{a}{}{,} 
 
\shorttitle{\PLANCK~Dust Model}
 
\shortauthors{Meisner \& Finkbeiner} 

\begin{document} 

\title{Two-component thermal dust emission model: application to the {\it PLANCK} HFI maps}
\author{Aaron M. Meisner\altaffilmark{1,2}}
\author{Douglas P. Finkbeiner\altaffilmark{1,2}}
\altaffiltext{1}{Department of Physics, Harvard University, 17 Oxford Street, 
Cambridge, MA 02138, USA; ameisner@fas.harvard.edu}
\altaffiltext{2}{Harvard-Smithsonian Center for Astrophysics, 60 Garden St, 
Cambridge, MA 02138, USA; dfinkbeiner@cfa.harvard.edu}

\begin{abstract}
We present full-sky maps of dust optical depth and temperature derived from 
the \cite{FDS99} two-component dust emission model fitted to \PLANCK~217, 353, 
545, 857 GHz and \cite{SFD} \IRAS~100$\mu$m data. Both the optical 
depth and temperature maps have angular resolution of $6.1'$. Such maps serve 
as an important alternative to and comparison for other dust 
emission models, especially the single modified blackbody model in which 
$\beta$ and $T$ both vary spatially. We also expect the derived optical 
depth to provide valuable cross-checks for other dust-column related data 
products, specifically the \cite{SFD} map, which assumed 
$\beta = 2$ single-component dust emission and included a temperature map of 
inferior $\sim$$1.3^{\circ}$ resolution. We briefly compare our 
best-fit two-component spectral energy distributions to those of the 
\cite{planckdust} single-component parametrization.
%mention the enhanced temperature resolution relative to SFD
%Our optical depth map will also provide a useful reference for dust maps 
%based on stellar colors rather than infrared emission.
\end{abstract}

\section{Introduction}
%why dust important in astronomy/astrophysics
The presence of Galactic interstellar dust affects
astronomical observations over a wide range of wavelengths. At mid-infrared
and far-infrared wavelengths, Galactic dust emission is a major contributor
to the observed sky intensity. In the optical and ultraviolet, extinction
by foreground dust grains attenuates the signal from extragalactic sources
over the entire sky. Understanding the properties of dust in the interstellar 
medium (ISM) is valuable in its own right, as the ISM is the site of star 
formation, thereby playing a crucial role in galaxy evolution. Equally if not 
more important to the practice of astronomy, however, is accounting for dust 
as a foreground which reddens optical/UV observations and superimposes 
Galactic emission on low-frequency observations of the 
cosmic microwave background (CMB). %specify ``low frequency more precisely eventually

%basic background about interstellar dust emission

%maybe say instead ``over the past 30/40 years...''
%Barnard first recognized evidence for dust in astronomical observations, and
%ever since it has been mapped in increasing detail. 
Over the past decades, satellite observations have dramatically enhanced our
knowledge about infrared emission from the ISM. High-latitude complexes of 
resolved infrared sources were first identified by \cite{low84} using 
\IRAS~60$\mu$m and 100$\mu$m observations \citep{wheelock94}. This so-called 
``infrared cirrus,'' attributable to thermal emission from large dust grains in
the interstellar medium, has since been probed in greater detail by a variety 
of instruments and detected over virtually the entire sky. Beginning in 1989, 
DIRBE  mapped the full sky at ten infrared wavelengths from 1.25$\mu$m to 
240$\mu$m with a reliable zero point, but inferior 0.7$^{\circ}$ angular 
resolution \citep{boggess92}.mention FIRAS specifically, since it is important 
to FDS99.

FDS99 used FIRAS to make a globally best-fit model of dust emission 
appropriate over a very broad range of frequencies. They showed that a 
model consisting of a single modified blackbody with power law index $\beta$ 
could not fit the IRAS/FIRAS/DIRBE spectrum at both the Wien and 
Rayleigh-Jeans extremes. To fit both ends of the spectrum, FDS99 proposed
a spectrum consisting of two modified black bodies, each with a different
temperature and power law index. Physically, these two components
represent distinct dust grain species within the ISM, specifically 
silicates and carbonaceous grains.


%how what we're doing fits in
%     where FDS99 fit in
%     Planck data qualities that improve on data used in FDS99
%     benefits of using Planck + FDS99 model
%utility/applications of what we're doing

%list of figures : 
% posterior surface for some pixel, with samples from markov chain overplotted
%    --> also overplot output best-fit (T,I545) and unctertainty estimate
% plot for some set of pixels of SED overlaid with best-fit two-component model
%    --> done in Python instead of IDL
%    --> overlay 
% plot of resulting T2 and tau
% plots of full-sky residuals relative to our model ?
%  ---> and for lower-frequency bands to emphasize that FDS99 is better for
%       extrapolating towards mm wavelengths

\section{Data}

All \PLANCK~data products utilized throughout this work are drawn from the 
\PLANCK~2013 data release \citep{planck2013}. Specifically, we have made use 
of the the 217 GHz, 353 GHz, 545 GHz and 857 GHz intensity maps. We use the 
versions of these maps which have been corrected for zodiacal light 
\citep[\texttt{R1.10\_nominal\_ZodiCorrected},][]{planckzodi}.

%For each pixel on the sky, Planck thus provides us with four data points of 
%the dust emission SED. The angular resolution of these maps is $5.5$.

%TODO: actually smooth all of the maps to a common PSF of 6.1 arcmin.

To incorporate information about the dust emission on the Wien side of its 
spectrum, we make use of 100$\mu$m data. In particular, we use the SFD 
reprocessing of IRAS 100$\mu$m, which we will refer to as \verb|i100|. \verb|i100| has angular resolution of $6.1'$. %list other
%important characteristics of i100 e.g. that it has been zodi-corrected
% and relies on dirbe on scales larger than ???

\section{Pre-processing}

\subsection{Smoothing}


\subsection{Unit Conversions}

Some of the maps in native format are provided in thermodynamic $\delta T$, 
while others are provided in units of MJy/sr. Before performing any fits, 
we convert all measurements to MJy/sr following the guidelines of the document
``HFI-unit conversion and colour correction instructions v1.2''.

%\subsection{Zodiacal Light}
%As mentioned above, we rely upon the \cite{planckzodi} zodiacal light 
%correction. 

\subsection{CMB Removal}
The CMB is effectively imperceptible at 857 GHz, but can be noticed at a 
low level at 545 GHz, and is prominent at 217 GHz and 353 GHz relative 
to the diffuse Galactic emission we wish to characterize, especially at high
latitudes. To remove the CMB anisotropies, we have subtracted the SMICA model
from each of our Planck maps \citep{smica}, after appropriate unit conversions.

\subsection{Finite Bandpass Correction}


\subsection{Correlation versus HI}
The raw \PLANCK~maps we have downloaded show offsets relative to HI, in the
sense that the FIR emission does not go to zero when HI emission goes to
zero. Since HI correlates linearly with far-IR emission in the diffuse ISM, 
when the optical depth is low and molecular effects are negligible, we modeled
far IR emission as a constant times HI emission, plus some offset. We then
subtracted the best-fit offset such that 

\subsection{Molecular Emission}
One advantage of the FIRAS data used by FDS99 was its many narrow bins in 
frequency, which allowed those frequency ranges contaminated by strong 
molecular line emission to be discarded from the analysis. Unfortunately, with 
the broad \PLANCK~and \IRAS~bandpasses we do not have such a luxury. This means
 that we cannot truly separate out the thermal continuum of dust emission we 
wish to study from the molecular line emission. The most prominent molecular 
line emission in the Planck bands of interest arises from CO lines in the XX, 
YY bands. We investigated the possibility of subtracting out CO emission based 
on the template of \cite{planckco}. However, we found that the CO emission
rarely amounted to more than XX\% of the total intensity, and following
the treatment of \cite{planckdust}, $\S$2.1, we do not attempt to remove CO
emission.

\subsection{Compact Sources}
To mask compact sources, we use the SFD compact source mask, which was 
originally constructed to mask point sources and resolved compact objects
that are not part of the Galactic cirrus at 3000 GHz. Given our pixelization 
(see $\S$\ref{sec:pix}), 1.56\% of pixels are masked. We have inspected all of 
the maps used in this analysis and do not see any evidence that this compact 
source mask fails to adequately mask compact sources in the \PLANCK~data. In 
practice, we perform our dust SED fits in all pixels, even those flagged by 
the point source mask. Compact source affected pixels can be ignored or 
interpolated over downstream.
%maybe say that we're releasing a version of the map with point sources
%interpolated over by us as well


\section{Dust Emission Model}
\label{sec:modeling}
Our model is that of FDS99, which consists of two populations of dust 
grains emitting as modified blackbodies, but with different temperatures
and different emissivity power law indices $\beta$. The shape of this spectrum 
is given by:

\begin{equation}
I_{\nu} \propto f_{1}q_{1}(\nu/\nu_{0})^{\beta_1}B_{\nu}(T_1) + (1-f_{1})q_{2}(\nu/\nu_0)^{\beta_2}B_{\nu}(T_2)
\end{equation}

With $\nu_0$ = 3000 GHz, $f_{1}$ = 0.0363, $\beta_1$ = 1.67, $\beta_2$ = 2.70, 
and $q_1/q_2$ = 13.0. For the parameters thus far specified, the two dust temperatures are further related by:

\begin{equation}
T_1 = 0.352T_2^{1.18}
\end{equation}

At each sky location, we have five independent intensity 
measurements (four ) FDS99 found typical values for $T_2$ and 

\subsection{Fitting Methodology}

\section{Pixelization}
\label{sec:pix}
For the purpose of fitting, we break the sky into independent pixels of 
angular size $\sim$1.72$'$, defined by the HEALPix pixelization with nside=2048 \citep{healpix}. 
This is convenient because the Planck maps are provided in this format and 
this pixelization adequately samples the $6.1'$ resolution maps under 
consideration. We reiterate that each pixel is fit independently of 
neighboring pixels.

\section{Markov Chains}
As explained in $\S$\ref{sec:modeling}, the two free parameters in our 
per-pixel dust emission SED are the hot dust temperature $T_2$, which 
determinesthe shape of the SED, and the spectrum normalization. In practice,
the normalization is controlled via $\tilde{I}_{545}$, the amplitude of the 
model SED in MJy/sr at $\nu$ = 545 GHz. For each pixel, we run a 
Metropolis-Hastings Markov chain sampling the posterior probability as a function of the two parameters $T_2$ and $\tilde{I}_{545}$. To be clear, we are sampling the posterior
given by:

\begin{equation}
P(\tilde{I}_{545}, T_2) \propto \mathcal{L}(\{I_i\}|\tilde{I}_{545}, T_2)P(T_2)
\end{equation}

The likelihood function is given by:

Where the errors on each intensity measurement, $\sigma_{\nu}$ are given by:

In order to obtain reasonable fitting results at high latitude where the
data is noisy, we include a prior on $T_2$

%mention something about the proposal distribution at some point?

\section{Results}

\begin{figure*}
\begin{center}
\epsfig{file=results.eps, width=7.0in}
\caption{Our best-fit $T_2$, binned to 27.5$'$ resolution}
\end{center}
\end{figure*}

\subsection{Calculating $\tau$}

\subsection{Data Release}
We are releasing nside=2048 HEALPix maps in Galactic coordinates summarizing 
the results of of our two-component dust fits. 

\section{Conclusion}

This material is based upon work supported by the National Science Foundation 
Graduate Research Fellowship under Grant No.Based on observations obtained with
 Planck (http://www.esa.int/Planck), an ESA science mission with instruments 
and contributions directly funded by ESA Member States, NASA, and Canada. This 
research made use of the NASA Astrophysics Data System (ADS) and the IDL 
Astronomy User's Library at Goddard. \footnote{Available at 
\texttt{http://idlastro.gsfc.nasa.gov}}

\bibliographystyle{apj}
\bibliography{twocomp.bib}

\end{document}
